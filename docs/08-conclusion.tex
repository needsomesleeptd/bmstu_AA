\chapter*{Заключение}
\addcontentsline{toc}{chapter}{Заключение}

В результате исследования было определено, что стандартный алгоритм умножения матриц проигрывает по времени алгоритму Винограда примерно в 1.38 раз из-за того, что в алгоритме Винограда часть вычислений происходит заранее, а также сокращается часть сложных операций - операций умножения, поэтому предпочтение следует отдавать алгоритму Винограда. 
Но лучшие показатели по времени выдает оптимизированный алгоритм Винограда -- он примерно в 1.58 раза быстрее алгоритма Винограда на размерах матриц свыше 100 из-за замены операций равно и плюс на операцию плюс-равно, а также за счёт замены операции умножения операцией сдвига, что дает проводить часть вычислений быстрее. 
Поэтому при выборе самого быстрого алгоритма предпочтение стоит отдавать оптимизированному алгоритму Винограда. 
Также стоит упомянуть, что алгоритм Винограда работает на чётных размерах матриц примерно в 1.1 раза быстрее, чем на нечётных, что связано с тем, что нужно произвести часть дополнительных вычислений для крайних строк и столбцов матриц, поэтому алгоритм Винограда лучше работает чётных размерах матриц.

Цель, которая была поставлена в начале лабораторной работы была достигнута, а также в ходе выполнения лабораторной работы были решены следующие задачи.
\begin{enumerate}[label={\arabic*)}]
	\item Описаны два алгоритмы умножения матриц;
	\item Создано программное обеспечение, реализующее следующие алгоритмы:
	\begin{itemize}[label=---]
		\item классический алгоритм умножения матриц;
		\item алгоритм Винограда;
		\item алгоритм Штрассена;
		\item оптимизированный алгоритм Винограда.
	\end{itemize}
	\item Оценены трудоемкости реализаций алгоритмов;
	\item Проведен анализ затрат работы программы по времени, выяснены влияющие на них характеристики;
	\item Проведен сравнительный анализ между алгоритмами.
\end{enumerate}