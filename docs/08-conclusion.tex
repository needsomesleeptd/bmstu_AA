\chapter*{ЗАКЛЮЧЕНИЕ}
\addcontentsline{toc}{chapter}{ЗАКЛЮЧЕНИЕ}

В результате исследования было определено, что лучшие показатели по времени дают итеративные реализации алгоритмов, вычисляющих расстояния Левенштейна и Дамерау---Левенштейна, а также его рекурсивная реализация с кешем. Рекурсивный алгоритм вычисления расстояния Дамерау---Левенштейна оказался в 65 577 раз хуже итеративного алгоритма по времени для длины строк 10. При этом реализации с использованием матрицы занимает довольно много памяти при большой длине строк.

Цель данной лабораторной работы были достигнуты, а именно описание и исследование алгоритмов, вычисляющих расстояния Левенштейна и Дамерау---Левенштейна.

Для достижения поставленной целей были выполнены следующие задачи.
\begin{enumerate}
	\item Изучены алгоритмы поиска расстояния Левенштейна и \newline Дамерау---Левенштейна;
	\item Создано программное обеспечение, реализующее следующие алгоритмы.
	\begin{itemize}
		\item нерекурсивный метод поиска расстояния Левенштейна;
		\item нерекурсивный метод поиска расстояния Дамерау---Левенштейна;
		\item рекурсивный метод поиска расстояния Дамерау---Левенштейна;
		\item рекурсивный с кешированием метод поиска расстояния Дамерау---Левенштейна.
	\end{itemize}
	\item Выбраны инструменты для реализации алгоритмов и замера процессорного времени их выполнения.
	\item Проведен анализ затрат алгоритмов по времени и по памяти. 
\end{enumerate}