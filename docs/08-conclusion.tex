\chapter*{Заключение}
\addcontentsline{toc}{chapter}{Заключение}

В результате исследования было определено, что реализация стандартного алгоритма умножения матриц проигрывает по времени реализации алгоритма Винограда  в 1.38 раз из-за того, что в алгоритме Винограда часть вычислений происходит заранее, а также сокращается часть сложных операций - операций умножения, поэтому предпочтение следует отдавать алгоритму Винограда. 
Но лучшие показатели по времени выдает реализация оптимизированного алгоритма Винограда -- он примерно в 1.58 раза быстрее реализации алгоритма Винограда на размерах матриц свыше 100 из-за замены операций равно и плюс на операцию плюс-равно, а также за счёт замены операции умножения операцией сдвига, что дает проводить часть вычислений быстрее. 
Поэтому при выборе самого быстрого алгоритма предпочтение стоит отдавать оптимизированному алгоритму Винограда. 

Реализация алгоритма Штрассена требует больших временных ресурсов, чем иные исследуемые реализации, при размерностях матриц больше 100 требует в 127 раз больше времени, чем реализация алгоритма Винограда и в 91 раз больше времени, чем стандартная реализация алгоритма умножения матриц. Это обусловлено необходимостью увлечения размерности матрицы, также в реализации данного алгоритма выделяется дополнительная память для разбиения матрицы на подматрицы, что также тратит временные ресурсы.

Цель, которая была поставлена в начале лабораторной работы была достигнута: описать, реализовать и исследовать алгоритмы умножения матриц.
Были решены все поставленные задачи:
\begin{enumerate}[label={\arabic*)}]
	\item описаны два алгоритмы умножения матриц;
	\item создано программное обеспечение, реализующее следующие алгоритмы:
	\begin{itemize}[label=---]
		\item классический алгоритм умножения матриц;
		\item алгоритм Винограда;
		\item алгоритм Штрассена;
		\item оптимизированный алгоритм Винограда.
	\end{itemize}
	\item оценены трудоемкости реализаций алгоритмов;
	\item проведен анализ затрат работы программы по времени, выяснены влияющие на них характеристики;
	\item проведен сравнительный анализ алгоритмов.
\end{enumerate}