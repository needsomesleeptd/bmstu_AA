\chapter{Аналитический раздел}


В данном разделе рассматриваются классический алгоритм умножения матриц и алгоритм Винограда, а также его оптимизированная версия.

\subsection{Определение матрицы}

\textbf{Матрица} \cite{book_matrix} представляет собой таблицу чисел $a_{ik}$ следующего вида:
\begin{equation}
	\begin{pmatrix}
		a_{11} & a_{12} & \ldots & a_{1n}\\
		a_{21} & a_{22} & \ldots & a_{2n}\\
		\vdots & \vdots & \ddots & \vdots\\
		a_{m1} & a_{m2} & \ldots & a_{mn}
	\end{pmatrix},
\end{equation}

где $m$ - количество строк, а $n$ - количество столбцов. Элементы $a_{ik}$ называются элементами матрицы.

Если $A$ - матрица, то $A_{i,j}$ обозначает элемент матрицы, расположенный на \textit{i-й} строке и \textit{j-м} столбце.

Существуют различные операции, выполняемые с матрицами:
\begin{enumerate}
	\item Сложение матриц одинакового размера;
	\item Вычитание матриц одинакового размера;
	\item Умножение матриц, 
	если количество столбцов в первой матрице равно количеству строк во второй матрице.
	Результатом является матрица, у которой количество строк равно количеству строк первой матрицы, 
	а количество столбцов - количеству столбцов второй матрицы. 
\end{enumerate}

\textit{Замечание:} в общем случае операция умножения матриц, 
когда в первой матрице число столбцов не равно числу строк во второй не определена.
Также стоит отметить что операция умножения матриц не коммутативна ($A \cdot B$ не всегда равно $B \cdot A$).

\clearpage

\subsection{Классический алгоритм}

Допустим, у нас есть две матрицы

\begin{equation}
	\label{eq:an_matrixies}
	A_{lm} = \begin{pmatrix}
		a_{11} & a_{12} & \ldots & a_{1n}\\
		a_{21} & a_{22} & \ldots & a_{2n}\\
		\vdots & \vdots & \ddots & \vdots\\
		a_{m1} & a_{m2} & \ldots & a_{mn}
	\end{pmatrix},
	\quad
	B_{mn} = \begin{pmatrix}
		b_{11} & b_{12} & \ldots & b_{1n}\\
		b_{21} & b_{22} & \ldots & b_{2n}\\
		\vdots & \vdots & \ddots & \vdots\\
		b_{m1} & b_{m2} & \ldots & b_{mn}
	\end{pmatrix},
\end{equation}

В таком случае умножение происходит по принципу <<строка на столбец>>, результом умножения матриц, заданных в
\ref{eq:an_matrixies} ($A \cdot B$), является матрица $C$, расчитанная по формуле (\ref{eq:an_def_mat_mul}).


\begin{equation}
	\label{eq:an_def_mat_mul}
	c_{ij} =
	\sum_{r=1}^{m} a_{ir}b_{rj} \quad (i=\overline{1,l}; j=\overline{1,n})
\end{equation}


\subsection{Алгоритм Винограда}

В 1987 году Дон Копперсмит и Виноград опубликовали метод, 
содержащий асимптотическую сложность $O(n^{2,3755})$ и улучшили 
его в 2011 до $O(n^{2,373})$, где $n$ -- размер сторон матрицы \cite{book_vinograd}.
В основной идее алгоритма были заложены следующие рассуждения:
Если посмотреть на результат умножения двух матриц, 
то видно, 
что каждый элемент в нем представляет собой 
скалярное произведение соответствующих строки и 
столбца исходных матриц. 
Можно заметить также, что такое умножение допускает предварительную 
обработку, позволяющую часть работы выполнить заранее.
Рассмотрим два вектора $V = (v_1, v_2, v_3, v_4)$ и $W = (w_1, w_2, w_3, w_4)$. Их скалярное произведение равно:
$V \cdot W = v_1w_1 + v_2w_2 + v_3w_3 + v_4w_4$.
Это равенство можно переписать в виде~(\ref{eq:vin_algo}).
\begin{equation}
	\label{eq:vin_algo}
	V \cdot W = (v_1 + w_2)(v_2 + w_1) + (v_3 + w_4)(v_4 + w_3) - v_1v_2 - v_3v_4 - w_1w_2 - w_3w_4.	
\end{equation}

Пусть матрицы $A, B, C$, ранее определенных размеров. Скалярное произведение (\ref{eq:vin_algo}), по замыслу Винограда, можно использовать при расчете 
произведения матриц~(\ref{eq:vin_mul_matrix}).
\begin{equation} 
	\label{eq:vin_mul_matrix}
	C_{ij} = \sum_{k=1}^{q/2}(a_{i,2k-1} + b_{2k,j})(a_{i,2k} + b_{2k-1,j}) - \sum_{k=1}^{q/2} a_{i,2k-1}a_{i,2k} - \sum_{k=1}^{q/2} b_{2k-1,j}b_{2k,j}
\end{equation}


Заметим, что в выражении (\ref{eq:vin_algo}), операнды со знаком $-$ могут быть вычислены заранее для каждой
строки и каждого столбца исходных матриц, а затем переиспользовать полученные значения что позволяет на практике заметно ускорить расчеты.
Аналгично с (\ref{eq:vin_mul_matrix}), единожды вычислив $\sum_{k=1}^{q/2}a_{i,2k-1}a_{i,2k}$ для строки, можно использовать данное
значение для получения значения любого произведения с операндами данной строки. При  предварительном расчете 
значений столбцов используется аналогичная идея.



\subsection{Оптимизированный алгоритм Винограда}

При программной реализации рассмотренного выше алгоритма Винограда можно сделать следующие оптимизации:
\begin{enumerate}
	\item значение $\frac{N}{2}$, используемое как ограничения цикла подсчёта предварительных данных, можно кэшировать;
	\item операцию умножения на 2 программно эффективнее реализовывать как побитовый сдвиг влево на 1;
	\item операции сложения и вычитания с присваиванием следует реализовывать при помощи соответствующего оператора $+=$ или $-=$ (при наличии данных операторов в выбранном языке программирования).
\end{enumerate}

\textbf{Вывод}
В данном разделе было введено понятие матрицы, определены несколько операций на ней, 
а также рассмотрена операция умножения матриц.
В данном разделе мы изучили матрицы и операции, 
которые можно выполнять с ними, включая умножение. 
Было рассмотрено  два алгоритма умножения матриц: классический и алгоритм Винограда.
Алгоритм Винограда выделяется своей эффективностью благодаря использованию предварительных вычислений. 
Также были введены  различные оптимизации, которые можно учесть при реализации алгоритма Винограда для 
достижения более высокой производительности.