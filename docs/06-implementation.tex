\chapter{Технологический раздел}

В данной части работы будут описаны средства реализации программы, а также листинги, модульные и функциональные тесты.

\section{Средства реализации}
Алгоритмы для данной лабораторной работы были реализованы на языке C++, при использовании компилятора gcc версии 10.5.0, так как в стандартной библиотеке приведенного языка
присутствует функция \texttt{clock\_gettime}, которая (при использовании макропеременной \texttt{CLOCK\_THREAD\_CPUTIME\_ID}) позволяет рассчитать процессорное время конкретного потока \cite{cpp-time}.



\section{Реализация алгоритмов}

Стоит отметить, что все используемые выше алгоритмы реализованы как метода класса $Matrix$, рекурсивные части алгоритмов были вынесены в отдельные функции.
Листинги исходных кодов программ  \ref{lst:std}--\ref{lst:stras} приведены в приложении.

\section{Тестирование}
Функциональные  тесты рассмотрены в таблицах \ref{t:unit_tests_def} -- \ref{t:unit_tests_strassen}.
\begin{table}[ht]
	\small
	\begin{center}
		\begin{threeparttable}
			\caption{Функциональные тесты для классического алгоритма умножения матриц}
			\label{t:unit_tests_def}
			\begin{tabular}{|c|c|c|c|c|}
				\hline
				\multicolumn{2}{|c|}{\bfseries Входные данные}
				& \multicolumn{2}{c|}{\bfseries Результат для классического алгоритма} \\
				\hline 
				\bfseries Матрица 1
				& \bfseries Матрица 2
				& \bfseries Ожидаемый результат
				& \bfseries Фактический результат \\
				\hline
				$\begin{pmatrix}
					1 & 2 & 3\\
					2 & 3 & 4\\
					4 & 6 & 7
				\end{pmatrix}$ 
				&  
				$\begin{pmatrix}
					1 & 0 & 0\\
					0 & 1 & 0\\
					0 & 0 & 1
				\end{pmatrix}$
				&
				$\begin{pmatrix}
					1 & 2 & 3\\
					2 & 3 & 4\\
					4 & 6 & 7
				\end{pmatrix}$ 
				&
				$\begin{pmatrix}
					1 & 2 & 3\\
					2 & 3 & 4\\
					4 & 6 & 7
				\end{pmatrix}$  \\ 
				\hline
				$\begin{pmatrix}
					1 & 2 & 3\\
					2 & 3 & 4\\
					4 & 6 & 7
				\end{pmatrix}$ 
				&  
				$\begin{pmatrix}
					1 & 2 & 3\\
					0 & 1 & 4\\
					5 & 0 & 1
				\end{pmatrix}$
				&
				$\begin{pmatrix}
					16 & 4 & 14\\
					22 & 7 & 22\\
					39 & 14 & 43
				\end{pmatrix}$
				&
				$\begin{pmatrix}
					16 & 4 & 14\\
					22 & 7 & 22\\
					39 & 14 & 43
				\end{pmatrix}$ \\ 
				\hline
				$\begin{pmatrix}
				 1& 2& 3 & \\
				 2& 3& 4 & \\
				 4& 6& 7 & \\
				 1& 4& -2 
				\end{pmatrix}$ 
				& 
				$\begin{pmatrix}
				 1& 2& 3& 0& 7& 8 \\
				 0& 1& 4& 0& 5& 6 \\
				 5& 0& 1& 0& 3& 4 
				\end{pmatrix}$
				&
				$\begin{pmatrix}
					16& 4& 14& 0& 26& 32 \\
					22& 7& 22& 0& 41& 50 \\
				 	39& 14& 43& 0& 79& 96 \\
					-9& 6& 17& 0& 21& 24 
				\end{pmatrix}$
				&
				$\begin{pmatrix}
					16& 4& 14& 0& 26& 32 \\
					22& 7& 22& 0& 41& 50 \\
					39& 14& 43& 0& 79& 96 \\
					-9& 6& 17& 0& 21& 24 
				\end{pmatrix}$ \\ 
				\hline
				$\begin{pmatrix}
					3 & 5\\
					2 & 1\\
					9 & 7\\
				\end{pmatrix}$
				&
				$\begin{pmatrix}
					1 & 2 & 3\\
					4 & 5 & 6 \\
				\end{pmatrix}$
				&
				\text{Сообщение об ошибке} 
				&
				\text{Сообщение об ошибке} \\ 
				\hline
				$\begin{pmatrix}
					10
				\end{pmatrix}$
				&
				$\begin{pmatrix}
					35
				\end{pmatrix}$
				&
				$\begin{pmatrix}
					350
				\end{pmatrix}$ 
				&
				$\begin{pmatrix}
					350
				\end{pmatrix}$ \\ 
				\hline
			\end{tabular}
		\end{threeparttable}
	\end{center}
\end{table}

\begin{table}[ht]
	\small
	\begin{center}
		\begin{threeparttable}
			\caption{Функциональные тесты для умножения матриц по алгоритму Винограда}
			\label{t:unit_tests_vin}
			\begin{tabular}{|c|c|c|c|c|}
			\hline
			\multicolumn{2}{|c|}{\bfseries Входные данные}
			& \multicolumn{2}{c|}{\bfseries Результат для классического алгоритма} \\
			\hline 
			\bfseries Матрица 1
			& \bfseries Матрица 2
			& \bfseries Ожидаемый результат
			& \bfseries Фактический результат \\
			\hline
			$\begin{pmatrix}
				1 & 2 & 3\\
				2 & 3 & 4\\
				4 & 6 & 7
			\end{pmatrix}$ 
			&  
			$\begin{pmatrix}
				1 & 0 & 0\\
				0 & 1 & 0\\
				0 & 0 & 1
			\end{pmatrix}$
			&
			$\begin{pmatrix}
				1 & 2 & 3\\
				2 & 3 & 4\\
				4 & 6 & 7
			\end{pmatrix}$ 
			&
			$\begin{pmatrix}
				1 & 2 & 3\\
				2 & 3 & 4\\
				4 & 6 & 7
			\end{pmatrix}$  \\ 
			\hline
			$\begin{pmatrix}
				1 & 2 & 3\\
				2 & 3 & 4\\
				4 & 6 & 7
			\end{pmatrix}$ 
			&  
			$\begin{pmatrix}
				1 & 2 & 3\\
				0 & 1 & 4\\
				5 & 0 & 1
			\end{pmatrix}$
			&
			$\begin{pmatrix}
				16 & 4 & 14\\
				22 & 7 & 22\\
				39 & 14 & 43
			\end{pmatrix}$
			&
			$\begin{pmatrix}
				16 & 4 & 14\\
				22 & 7 & 22\\
				39 & 14 & 43
			\end{pmatrix}$ \\ 
			\hline
			$\begin{pmatrix}
				1& 2& 3 & \\
				2& 3& 4 & \\
				4& 6& 7 & \\
				1& 4& -2 
			\end{pmatrix}$ 
			& 
			$\begin{pmatrix}
				1& 2& 3& 0& 7& 8 \\
				0& 1& 4& 0& 5& 6 \\
				5& 0& 1& 0& 3& 4 
			\end{pmatrix}$
			&
			$\begin{pmatrix}
				16& 4& 14& 0& 26& 32 \\
				22& 7& 22& 0& 41& 50 \\
				39& 14& 43& 0& 79& 96 \\
				-9& 6& 17& 0& 21& 24 
			\end{pmatrix}$
			&
			$\begin{pmatrix}
				16& 4& 14& 0& 26& 32 \\
				22& 7& 22& 0& 41& 50 \\
				39& 14& 43& 0& 79& 96 \\
				-9& 6& 17& 0& 21& 24 
			\end{pmatrix}$ \\ 
			\hline
			$\begin{pmatrix}
				3 & 5\\
				2 & 1\\
				9 & 7\\
			\end{pmatrix}$
			&
			$\begin{pmatrix}
				1 & 2 & 3\\
				4 & 5 & 6 \\
			\end{pmatrix}$
			&
			\text{Сообщение об ошибке} 
			&
			\text{Сообщение об ошибке} \\ 
			\hline
			$\begin{pmatrix}
				10
			\end{pmatrix}$
			&
			$\begin{pmatrix}
				35
			\end{pmatrix}$
			&
			$\begin{pmatrix}
				350
			\end{pmatrix}$ 
			&
			$\begin{pmatrix}
				350
			\end{pmatrix}$ \\ 
			\hline
		\end{tabular}
		\end{threeparttable}
	\end{center}
\end{table}

\begin{table}[ht]
	\small
	\begin{center}
		\begin{threeparttable}
			\caption{Функциональные тесты для реализации алгоритма умножения матриц Штрассена}
			\label{t:unit_tests_strassen}
			\begin{tabular}{|c|c|c|c|c|}
			\hline
			\multicolumn{2}{|c|}{\bfseries Входные данные}
			& \multicolumn{2}{c|}{\bfseries Результат для классического алгоритма} \\
			\hline 
			\bfseries Матрица 1
			& \bfseries Матрица 2
			& \bfseries Ожидаемый результат
			& \bfseries Фактический результат \\
			\hline
			$\begin{pmatrix}
				1 & 2 & 3\\
				2 & 3 & 4\\
				4 & 6 & 7
			\end{pmatrix}$ 
			&  
			$\begin{pmatrix}
				1 & 0 & 0\\
				0 & 1 & 0\\
				0 & 0 & 1
			\end{pmatrix}$
			&
			$\begin{pmatrix}
				1 & 2 & 3\\
				2 & 3 & 4\\
				4 & 6 & 7
			\end{pmatrix}$ 
			&
			$\begin{pmatrix}
				1 & 2 & 3\\
				2 & 3 & 4\\
				4 & 6 & 7
			\end{pmatrix}$  \\ 
			\hline
			$\begin{pmatrix}
				1 & 2 & 3\\
				2 & 3 & 4\\
				4 & 6 & 7
			\end{pmatrix}$ 
			&  
			$\begin{pmatrix}
				1 & 2 & 3\\
				0 & 1 & 4\\
				5 & 0 & 1
			\end{pmatrix}$
			&
			$\begin{pmatrix}
				16 & 4 & 14\\
				22 & 7 & 22\\
				39 & 14 & 43
			\end{pmatrix}$
			&
			$\begin{pmatrix}
				16 & 4 & 14\\
				22 & 7 & 22\\
				39 & 14 & 43
			\end{pmatrix}$ \\ 
			\hline
			$\begin{pmatrix}
				1& 2& 3 & \\
				2& 3& 4 & \\
				4& 6& 7 & \\
				1& 4& -2 
			\end{pmatrix}$ 
			& 
			$\begin{pmatrix}
				1& 2& 3& 0& 7& 8 \\
				0& 1& 4& 0& 5& 6 \\
				5& 0& 1& 0& 3& 4 
			\end{pmatrix}$
			&
			$\begin{pmatrix}
				16& 4& 14& 0& 26& 32 \\
				22& 7& 22& 0& 41& 50 \\
				39& 14& 43& 0& 79& 96 \\
				-9& 6& 17& 0& 21& 24 
			\end{pmatrix}$
			&
			$\begin{pmatrix}
				16& 4& 14& 0& 26& 32 \\
				22& 7& 22& 0& 41& 50 \\
				39& 14& 43& 0& 79& 96 \\
				-9& 6& 17& 0& 21& 24 
			\end{pmatrix}$ \\ 
			\hline
			$\begin{pmatrix}
				3 & 5\\
				2 & 1\\
				9 & 7\\
			\end{pmatrix}$
			&
			$\begin{pmatrix}
				1 & 2 & 3\\
				4 & 5 & 6 \\
			\end{pmatrix}$
			&
			\text{Сообщение об ошибке} 
			&
			\text{Сообщение об ошибке} \\ 
			\hline
			$\begin{pmatrix}
				10
			\end{pmatrix}$
			&
			$\begin{pmatrix}
				35
			\end{pmatrix}$
			&
			$\begin{pmatrix}
				350
			\end{pmatrix}$ 
			&
			$\begin{pmatrix}
				350
			\end{pmatrix}$ \\ 
			\hline
		\end{tabular}
		\end{threeparttable}
	\end{center}
\end{table}

Все тесты были успешно пройдены.



\section*{Вывод}

В данной части работы были рассмотрены средства реализации, проведено тестирование, а также были реализованы рассмотренные ранее алгоритмы.
