\chapter{Исследовательский раздел}

В данном разделе будут приведены примеры работы программ, постановка эксперимента и сравнительный анализ алгоритмов на основе полученных данных.


\section{Технические характеристики}

Технические характеристики устройства, на котором выполнялись замеры по времени, представлены далее.

\begin{enumerate}
	\item Процессор	Intel(R) Core(TM) i7-9750H CPU @ 2.60GHz, 2592 МГц, ядер: 6, логических процессоров: 12;
	\item Оперативная память: 16 ГБайт;
	\item Операционная система: Майкрософт Windows 10 Pro \cite{windows};
	\item Использованная подсистема: WSL2 \cite{WSL2}.
\end{enumerate}

При замерах времени ноутбук был включен в сеть электропитания и был нагружен только системными приложениями.
\section{Пример работы программы}
На рисунке \ref{img:prog-work}, представлен пример работы программы. Были введены размерности матриц и выбран метод их умножения, после чего был выведен результат умножения.
\includeimage
{prog-work} % Имя файла без расширения (файл должен быть расположен в директории inc/img/)
{f} % Обтекание (без обтекания)
{H} % Положение рисунка (см. figure из пакета float)
{1\textwidth} % Ширина рисунка
{Пример работы программы} % Подпись рисунка



\section{Временные характеристики}

Результаты исследования замеров по времени приведены в таблице \ref{t:timings}.
Введем следующие обозначения для чтения таблиц:
\begin{enumerate}
	\item n --- размерность умножаемых матриц;
	\item CМ --- реализация стандартного алгоритма умножения матриц;
	\item ВМ --- реализация алгоритма умножения матриц Винограда;
	\item ВМО ---реализация алгоритма умножения матриц Винограда (оптимизированный);
	\item ШМ --- реализация алгоритма умножения матриц Штрассена.
\end{enumerate}

Для таблиц \ref{t:timings} и \ref{t:timings_even} замеры  производились с различным шагом, для каждой пары матриц производилось 100 замеров времени, после чего результаты замеров усреднялись. Все результаты вычислений приведены в миллисекундах.

\begin{table}[!ht]
	\centering
	\caption{Полученная таблица замеров по времени различных реализаций алгоритмов умножения матриц нечетной размерности}
	\begin{tabular}{|c|c|c|c|c|}
		\hline
		n   & СМ (мс)  & ШМ (мс)  & \multicolumn{1}{c|}{ВМ (мс)} & ВМО (мс) \\ \hline
		1   & 0.002159 & 0.004304 & 0.001817                     & 0.001758 \\ \hline
		11  & 0.025625 & 7.3367   & 0.023301                     & 0.020161 \\ \hline
		21  & 0.14005  & 46.559   & 0.11752                      & 0.1004   \\ \hline
		31  & 0.43789  & 46.262   & 0.34888                      & 0.29771  \\ \hline
		41  & 0.99899  & 326.63   & 0.78522                      & 0.66993  \\ \hline
		51  & 1.9746   & 332.43   & 1.5321                       & 1.3118   \\ \hline
		61  & 3.3842   & 329.97   & 2.5529                       & 2.1928   \\ \hline
		71  & 5.1593   & 2312.2   & 4.0072                       & 3.4496   \\ \hline
		81  & 7.5757   & 2279.9   & 5.763                        & 4.9773   \\ \hline
		91  & 10.703   & 2290.7   & 7.9208                       & 6.878    \\ \hline
		101 & 14.504   & 2300.3   & 11.025                       & 9.4659   \\ \hline
		111 & 19.166   & 2285.8   & 14.579                       & 12.439   \\ \hline
		121 & 25.184   & 2287.3   & 18.741                       & 16.192   \\ \hline
	\end{tabular}
	\label{t:timings}
\end{table}

\begin{table}[!ht]
	\centering
	\caption{Полученная таблица замеров по времени различных реализаций алгоритмов умножения матриц четной размерности}
	\begin{tabular}{|c|c|c|c|c|}
		\hline
		n   & СМ (мс)  & ШМ (мс)  & ВМ (мс)  & ВМО (мс  \\ \hline
		2   & 0.002283 & 0.024897 & 0.002821 & 0.003045 \\ \hline
		12  & 0.043452 & 9.2957   & 0.037144 & 0.032815 \\ \hline
		22  & 0.19104  & 51.76    & 0.14542  & 0.12715  \\ \hline
		32  & 0.5097   & 48.932   & 0.39907  & 0.34554  \\ \hline
		42  & 1.0893   & 331.24   & 0.83759  & 0.72743  \\ \hline
		52  & 1.9912   & 323.34   & 1.509    & 1.3167   \\ \hline
		62  & 3.4132   & 328.08   & 2.6004   & 2.2481   \\ \hline
		72  & 5.264    & 2286.4   & 3.9626   & 3.4397   \\ \hline
		82  & 8.0663   & 2332.5   & 5.9899   & 5.2073   \\ \hline
		92  & 11.053   & 2294     & 8.3101   & 7.1779   \\ \hline
		102 & 15.142   & 2286.2   & 11.272   & 9.8261   \\ \hline
		112 & 19.793   & 2273.9   & 14.748   & 12.853   \\ \hline
		122 & 25.798   & 2268.4   & 19.095   & 16.768   \\ \hline
	\end{tabular}
	\label{t:timings_even}
\end{table}



\includeimage
{all-matrix-cmp} % Имя файла без расширения (файл должен быть расположен в директории inc/img/)
{f} % Обтекание (без обтекания)
{H} % Положение рисунка (см. figure из пакета float)
{1\textwidth} % Ширина рисунка
{Сравнение реализаций исследуемых алгоритмов по временри исполнения с использованием логарифмической шкалы с матрицами нечетных размерностей} % Подпись рисунка

\includeimage
{vins-matrix-cmp} % Имя файла без расширения (файл должен быть расположен в директории inc/img/)
{f} % Обтекание (без обтекания)
{H} % Положение рисунка (см. figure из пакета float)
{1\textwidth} % Ширина рисунка
{Сравнение различных версий реализаций алгоритма Винограда с матрицами нечетных размерностей} % Подпись рисунка

\includeimage
{std-vin-matrix-cmp} % Имя файла без расширения (файл должен быть расположен в директории inc/img/)
{f} % Обтекание (без обтекания)
{H} % Положение рисунка (см. figure из пакета float)
{1\textwidth} % Ширина рисунка
{Сравнение реализаций алгоритма Винограда и классического алгоритма умножения матриц с нечетными размерностями} % Подпись рисунка


\includeimage
{all-matrix-cmp-even} % Имя файла без расширения (файл должен быть расположен в директории inc/img/)
{f} % Обтекание (без обтекания)
{H} % Положение рисунка (см. figure из пакета float)
{1\textwidth} % Ширина рисунка
{Сравнение реализаций исследуемых алгоритмов по времени при исполнения с использованием логарифмической шкалы с четными размерностями матриц} % Подпись рисунка

\includeimage
{vins-matrix-cmp-even} % Имя файла без расширения (файл должен быть расположен в директории inc/img/)
{f} % Обтекание (без обтекания)
{H} % Положение рисунка (см. figure из пакета float)
{1\textwidth} % Ширина рисунка
{Сравнение различных версий реализаций алгоритма Винограда с четными размерностями матриц} % Подпись рисунка

\includeimage
{std-vin-matrix-cmp-even} % Имя файла без расширения (файл должен быть расположен в директории inc/img/)
{f} % Обтекание (без обтекания)
{H} % Положение рисунка (см. figure из пакета float)
{1\textwidth} % Ширина рисунка
{Сравнение реализаций алгоритма Винограда и классического алгоритма умножения матриц с четными размерностями матриц} % Подпись рисунка







\section*{Вывод}
В результате анализа таблицы \ref{t:timings},\ref{t:timings_even}, что при больших размерах матриц (свыше 100) нечетной размерности, реализация алгоритма Винограда тратит в  1.38 раза , а реализация  оптимизированного алгоритма Винограда  в 1.56 раз меньше времени, чем реализация стандартного алгоритма умножения матриц.


Реализация алгоритма Штрассена требует больших временных ресурсов, чем иные исследуемые реализации, при размерностях матриц больше 100 требует в 127 раз больше времени, чем реализация алгоритма Винограда и в 91 раз больше времени, чем стандартная реализация алгоритма умножения матриц, данное обусловлено необходимостью увеличения размерности матриц для использования алгоритма.

