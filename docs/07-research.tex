\chapter{Исследовательский раздел}

В данном разделе будут приведены примеры работы программ, постановка эксперимента и сравнительный анализ алгоритмов на основе полученных данных.


\section{Технические характеристики}

Технические характеристики устройства, на котором выполнялись замеры по времени, представлены далее.

\begin{enumerate}
	\item Процессор	Intel(R) Core(TM) i7-9750H CPU @ 2.60GHz, 2592 МГц, ядер: 6, логических процессоров: 12;
	\item Оперативная память: 16 ГБайт;
	\item Операционная система: Майкрософт Windows 10 Pro \cite{windows};
	\item Использованная подсистема: WSL2 \cite{WSL2}.
\end{enumerate}

При замерах времени ноутбук был включен в сеть электропитания и был нагружен только системными приложениями.


\section{Временные характеристики}

Результаты исследования замеров по времени приведены в таблице \ref{t:timings}.
Введем следующие обозначения для чтения таблиц:
\begin{enumerate}
	\item n --- размерность умножаемых матриц;
	\item CМ --- реализация стандартного алгоритма умножения матриц;
	\item ММ --- реализация алгоритма умножения матриц Винограда;
	\item ММО ---реализация алгоритма умножения матриц Винограда (оптимизированный);
	\item ШМ --- реализация алгоритма умножения матриц Штрассена.
\end{enumerate}

%Заметим, что некоторые поля в данной таблице
%имеют значение <<$\sim$>>, это обусловлено тем, что дальнейший расчет %значений столбца <<ДР>> окажется слишком
%долгим, полученных данных достаточно для проведения исследования.
%Написать про таблицы.


\section{Характеристики по памяти}

\label{memory}

Введем следующие обозначения:
\begin{itemize}
	\item $Mat1$ --- длина строки $S_1$;
	\item $Mat2$ --- длина строки $S_2$;
	\item $size()$ -- функция, вычисляющая размер в байтах;
	\item $int$ -- целочисленный тип данных;
	\item $string$ -- строковый тип данных.
\end{itemize}

Т.~к. алгоритмы, вычисляющие расстояния Левенштейна и Дамерау---Левенштейна, не отличаются по использованию памяти, то достаточно рассмотреть итеративную, рекурсивную и рекурсивную с кешированием реализации алгоритмов вычисления расстояния Дамерау---Левенштейна.


Использование памяти при итеративной реализации теоритически рассчитывется по формул \eqref{eq:iter_mem}.
\begin{equation}
	\label{eq:iter_mem}
	(n + 1) * (m + 1) * size(int) + 2 * size(string) + 2 * size(int),
\end{equation}
где 
\begin{itemize}
	\item $ (n + 1) * (m + 1) * size(int) $ -- хранение матрицы;
	\item $ 2 * size(string) $ -- хранение двух строк;
	\item $ 2 * size(int) $ -- адрес возврата и возвращаемое значение.
\end{itemize}


Максимальная глубина стека вызовов при рекурсивной реализации
нахождения расстояния Дамерау---Левенштейна равна сумме входящих строк,
соответственно, максимальный расход памяти рассчитывается по \eqref{eq:rec_mem}.

\begin{equation}
	\label{eq:rec_mem}
	(n + m) * (2 * size(string) + 3 * size(int)),
\end{equation}
где 
\begin{itemize}
	\item $ (n + m) $ -- максимальная глубина стека вызовов;
	\item $ 2 * size(string) $ -- хранение двух строк;
	\item $ 2 * size(int) $ -- адрес возврата и возвращаемое значение;
	\item $ size(int) $ -- временная переменная.
\end{itemize}

Для алгоритма, использующего кеширование требуется дополнительно память под кеш и 4 временных переменных \eqref{eq:req_cash_mem}.

\begin{equation}
	\label{eq:req_cash_mem}
	(n + m) * (2 * size(string) + 6 * size(int)) + (n + 1) * (m + 1) * size(int),
\end{equation}
где 
\begin{itemize}
	\item $ (n + m) $ -- максимальная глубина стека вызовов;
	\item $ 2 * size(string) $ -- хранение двух строк;
	\item $ 2 * size(int) $ -- адрес возврата и возвращаемое значение;
	\item $ 4 * size(int) $ -- временные переменные;
	\item $ (n + 1) * (m + 1) * size(int) $ -- хранение кеша.
\end{itemize}

По расходу памяти итеративные алгоритмы проигрывают рекурсивным: максимальный размер используемой памяти в итеративном растет
как произведение длин строк, в то время как у рекурсивного алгоритма --
как сумма длин строк.


\section*{Вывод}

В данном разделе было произведено сравнение количества затраченного времени и памяти алгоритмов поиска расстояний Левенштейна и
Дамерау---Левенштейна. Наименее затратным по времени оказался итеративный алгоритм нахождения расстояния Левенштейна.

По таблице \ref{tbl:time_measurements} видно, что рекурсивный алгоритм в 65577 раз проигрывает итеративному при длине строк 10. Поэтому рекурсивные алгоритмы следует использовать лишь при малых длинах строк.

При этом как было замечено в пункте \ref{memory}, рекурсивные алгоритмы занимают меньше памяти, чем итеративные алгоритмы.

Рекурсивная реализация алгоритма поиска расстояния Дамерау---Левенштейн будет более затратным по времени по сравнению с итеративной реализацией алгоритма поиска расстояния Дамерау---Левенштейна, но менее затратным по памяти по отношению к итеративному алгоритму Дамерау---Левенштейна. При этом рекурсивные алгоритм с кешированием проигрывает по памяти и по времени итеративному.