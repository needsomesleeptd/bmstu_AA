\chapter*{ВВЕДЕНИЕ}
\addcontentsline{toc}{chapter}{ВВЕДЕНИЕ}


В данной лабораторной работе будет исследованы алгоритмы умножения матриц. Как и в математике, в программировании матрицы часто играют важную роль. Они находят широкое применение в разных областях, включая физику, где они используются для вывода формул, таких как:
\begin{enumerate}
	\item градиент;
	\item дивергенция;
	\item ротор. \cite{book_matrix}
\end{enumerate}

Также матрицы неизбежно встречаются при выполнении различных операций, таких как сложение, возведение в степень и умножение. При работе над разными задачами размеры матриц могут быть значительными, поэтому оптимизация операций с матрицами становится важным аспектом в программировании. В данной лабораторной работе мы сосредоточимся на оптимизации умножения матриц.

Цель этой лабораторной работы включает в себя описание, реализацию и исследование алгоритмов умножения матриц.

Для достижения этой цели необходимо выполнить следующие задачи:
\begin{enumerate}
	\item описать два алгоритма умножения матриц;
	\item разработать программное обеспечение, которое реализует следующие алгоритмы:
	\begin{enumerate}
		\item классический алгоритм умножения матриц;
		\item алгоритм Винограда;
		\item оптимизированный алгоритм Винограда;
		\item алгоритм Штрассена.
	\end{enumerate}
	\item оценить трудоемкость умножения матриц;
	\item провести анализ временных затрат работы программы и выявить важные факторы, влияющие на эти затраты;
	\item сравнить алгоритмы между собой.
\end{enumerate}
