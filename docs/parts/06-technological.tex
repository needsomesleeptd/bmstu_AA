\chapter{Технологическая часть}
В данном разделе будут описаны средства реализации, модули программы, а также листинги, модульные и функциональные тесты.

\section{Средства реализации}
Алгоритмы для данной лабораторной работы были реализованы на языке C++, при использовании компилятора gcc версии 10.5.0, так как в стандартной библиотеке приведенного языка
присутствует функция \texttt{clock\_gettime}, которая (при использовании макропеременной \texttt{CLOCK\_THREAD\_CPUTIME\_ID}) позволяет рассчитать процессорное время конкретного потока \cite{cpp-time}.



\section{Листинги реализаций алгоритмов}

Стоит отметить, что все используемые выше алгоритмы реализованы как метода класса $Matrix$, рекурсивные части алгоритмов были вынесены в отдельные функции.
Листинги исходных кодов программ  \ref{lst:lev_matr}--\ref{lst:damer_rec_meth} приведены в приложении.


\section{Тестирование}
При написании тестов использовалась библиотека $gtest$, позволяющая писать модульные тесты, которые очень удобны 
в данном случае. Данные тесты приведены в листинге  \ref{lst:unit_tests}. При реализации данных тестов вычисленные значения
сравниваются со значениями, заранее известными для соответствующих входных данных, с помощью приведенной библиотеки.


Также данные тесты рассмотрены в таблице \ref{t:unit_tests}.

\begin{table}[!ht]
	
	\begin{center}
		\small
		\begin{threeparttable}
		\caption{Модульные тесты}
        \label{t:unit_tests}
		\begin{tabular}{|c|c|c|c|c|c|}
			\hline
			\multicolumn{2}{|c|}{\bfseries Входные данные}
			& \multicolumn{4}{c|}{\bfseries Расстояние и алгоритм} \\ 
			\hline 
			&
			& \multicolumn{1}{c|}{\bfseries Левенштейна} 
			& \multicolumn{3}{c|}{\bfseries Дамерау---Левенштейна} \\ \cline{3-6}
			
			\bfseries Строка 1 & \bfseries Строка 2 & \bfseries Итеративный & \bfseries Итеративный
			
			& \multicolumn{2}{c|}{\bfseries Рекурсивный} \\ \cline{5-6}
			& & & & \bfseries Без кеша & \bfseries С кешом \\
			\hline
			wwwwwwc & bbbbbbc & 6 & 6 & 6 & 6 \\
			\hline
			AB & BA & 2 & 1 & 1 & 1 \\
			\hline
			KAABKA & AKAAK & 3 & 3 & 3 & 3 \\
			\hline
			ABC & BCA & 2 & 2 & 2 & 2 \\
			\hline
			ВФА & АВФ & 2 & 2 & 2 & 2 \\
			\hline
			ADF & ABFDSADADF & 7 & 7 & 7 & 7 \\
			\hline
		\end{tabular}	
		\end{threeparttable}
	\end{center}
\end{table}

Все тесты были успешно пройдены.