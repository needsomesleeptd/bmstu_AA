\chapter{Конструкторская часть}
В данной части работы будут рассмотрены схемы муравьиного алгоритма и алгоритма полного перебора.

\section{Требования к программному обеспечению}
Программа должна предоставлять следующие возможности:
\begin{itemize}
	\item выбор файла с матрицей расстояний;
	\item ввод параметров $\alpha$, $\beta$ и $days$ для муравьиного алгоритма;
	\item вывод на экран найденного каждым из алгоритмов кратчайшего пути и его длины;
	\item измерение времени получения результатов каждого из рассматриваемых алгоритмов.
\end{itemize}


\section{Схемы алгоритмов}
На рисунке~\ref{img:FullPerm} представлен алгоритм перебора всех возможных путей.
На рисунках~\ref{img:ants_1_saved}--\ref{img:choose_path}, представлен муравьиный алгоритм и схемы алгоритмов расчета данных для муравьиного алгоритма соотвественно.


\includeimage
{FullPerm} % Имя файла без расширения (файл должен быть расположен в директории inc/img/)
{f} % Обтекание (без обтекания)
{h} % Положение рисунка (см. figure из пакета float)
{1\textwidth} % Ширина рисунка
{Схема алгоритма перебора всех возможных путей} % Подпись рисунка

\includeimage
{ants_1_saved} % Имя файла без расширения (файл должен быть расположен в директории inc/img/)
{f} % Обтекание (без обтекания)
{h} % Положение рисунка (см. figure из пакета float)
{1\textwidth} % Ширина рисунка
{Схема муравьиного алгоритма} % Подпись рисунка


\includeimage
{update_pher} % Имя файла без расширения (файл должен быть расположен в директории inc/img/)
{f} % Обтекание (без обтекания)
{h} % Положение рисунка (см. figure из пакета float)
{0.8\textwidth} % Ширина рисунка
{Схема алгоритма обновления феромонов} % Подпись рисунка

\includeimage
{calc_probs} % Имя файла без расширения (файл должен быть расположен в директории inc/img/)
{f} % Обтекание (без обтекания)
{h} % Положение рисунка (см. figure из пакета float)
{1\textwidth} % Ширина рисунка
{Схема алгоритма расчета вероятностей перехода муравья в вершины} % Подпись рисунка

\includeimage
{choose_path} % Имя файла без расширения (файл должен быть расположен в директории inc/img/)
{f} % Обтекание (без обтекания)
{h} % Положение рисунка (см. figure из пакета float)
{1\textwidth} % Ширина рисунка
{Схема алгоритма выбора следующей вершины для перехода} % Подпись рисунка

\section{Оценка трудоемкости}
Задача коммивояжера является $NP$-трудной, и точный переборный алгоритм ее решения имеет факториальную сложность $O(n!)$~\cite{salesman}. 

Сложность муравьиного алгоритма равна $O(t_{max} \cdot m \cdot n^2)$, то есть она зависит от времени жизни колонии, количества городов и количества муравьев в колонии~\cite{ulyanov}. 

В разработанной реализации количество муравьев равно количеству городов, следовательно трудоемкость муравьиного алгоритма равна $O(t_{max} \cdot n^3)$.

\section*{Вывод}
В данной части работы были рассмотрены схемы муравьиного алгоритма и алгоритма полного перебора.
