\chapter{Конструкторская часть}
В данной части работы будут рассмотрены схемы алгоритмов сортировок, а также приведен расчет их трудоемкости.





\section{Разработка алгоритмов}
На рисунке \ref{img:svd} приведены схемы алгоритма сингулярного разложения.

\includeimage
{svd} % Имя файла без расширения (файл должен быть расположен в директории inc/img/)
{f} % Обтекание (без обтекания)
{H} % Положение рисунка (см. figure из пакета float)
{0.8\textwidth} % Ширина рисунка
{Схема алгоритма сингулярного разложения} % Подпись рисунка






\section{Оценка трудоемкости реализации сингулярного разложения}
В данной части работы, будет произведен расчет трудоемкости для каждой составляющей сингулярного разложения.
\subsection{Расчет транспонированной матрицы}
Матрица размерами $N \times M$.

\begin{equation}
	f_{matrix\_transpose} = f_{resize} + f_{outer\_for}.
\end{equation}

\begin{equation}
	f_{outer\_for} = 3 + N \cdot (f_{resize} + 3 + M \cdot 5)).
\end{equation}

Тогда:
\begin{equation}
	\begin{gathered}
		f_{matrix\_transpose} = f_{resize} + 3 + N \cdot (f_{resize} + 3 + 5M) = f_{resize} + 3N \cdot f_{resize} + \\ + 9N + 15MN= f_{resize} \cdot (3N + 1) + 9N + 15MN.
	\end{gathered}
\end{equation}

\subsection{Расчет произведения матриц}

Матрицы размерами $N \times M$ и $M \times K$.

\begin{equation}
	f_{matrix\_by\_matrix} = f_{resize} + f_{1\_outer\_for}.
\end{equation}

\begin{equation}
	f_{1\_outer\_for} = 3 + N \cdot (f_{resize} + f_{2\_outer\_for}).
\end{equation}

\begin{equation}
	f_{2\_outer\_for} = 3 + M \cdot (3 +  f_{inner\_for}).
\end{equation}

\begin{equation}
	f_{inner\_for} = 3 + M \cdot (2 + 1 + 2 + 2  + 2) = 3 + 9M.
\end{equation}

Тогда:

\begin{equation}
	\begin{gathered}
		f_{matrix\_by\_matrix} = f_{resize} + f_{1\_outer\_for} =
		f_{resize} + 3 \cdot \\ \cdot N (f_{resize} + 3 + M \cdot (3 + 3 + 9M)) = f_{resize} + 3 \cdot N (f_{resize} + 3 + \\ + 6M + 9M^2) = f_{resize} + 3N \cdot f_{resize} + 9N + 18NM + 27NM^2 = \\ = f_{resize} \cdot (3N + 1) + 9N + 18NM + 27NM^2.
	\end{gathered}
\end{equation}

\section{Получение эрмитовой матрицы}

Размер вектора равен $N$;

\begin{equation}
	f_{get\_hermitian\_matrix} = f_{resize} + f_{for\_1} + (2 + 1 + 2 + 1) + f_{for\_2} + f_{for\_3}.
\end{equation}

\begin{equation}
	f_{for\_1} = 3 + N \cdot (1 + f_{resize}) = 3 + N + N \cdot f_{resize}.
\end{equation}

\begin{equation}
	\begin{gathered}
		f_{for\_2} = 3 + (N - 1) \cdot (2 + 1 + 1 + 2 + 1) = 3 + (N - 1) \cdot 7 =\\=3 + 7N - 7 = 7N - 4.
	\end{gathered}
\end{equation}

\begin{equation}
	f_{for\_3} = 3 + N \cdot (2 + 1) = 3 + 3N.
\end{equation}

Тогда:

\begin{equation}
	\begin{gathered}
		f_{get\_hermitian\_matrix} = f_{resize} + f_{for\_1} + (2 + 1 + 2 + 1) +\\+ f_{for\_2} + f_{for\_3} =\\=f_{resize} + 3 + N + N \cdot f_{resize} + 6 + 7N - 4 + 3 + 3N =\\=
		f_{resize} \cdot (1 + N) + 8 + 11N.
	\end{gathered}
\end{equation}

\section{Получение обратной эрмитовой матрицы}

Имеет аналогичную трудоемкость, что и $get\_hermitian\_matrix$ в силу смены порядка операций.

\section{Решение системы с помощью метода Гаусса-Жордана}

Размер матрицы $N \times M$, размер вектора $K$.

\begin{equation}
	f_{jordan\_gaussian\_transform} = f_{for} + f_{while} + 3.
\end{equation}

\begin{equation}
	f_{for} = 3 + (N - 1) \cdot (6 + f_{inner\_for\_1} + f_{inner\_for\_2}).
\end{equation}

\begin{equation}
	f_{inner\_for\_1} = 3 + M \cdot (2 + 2 + 2 + f_{swap}) = 3 + 6M + M \cdot f_{swap}.
\end{equation}

\begin{equation}
	\begin{gathered}
		f_{inner\_for\_2} = 3 + N \cdot (1 + 2 + 3 + M \cdot (2 + 2 + 1 + 2 + 1 + 2 + 2)) =\\=
		3 + N \cdot (6 + 12M) = 3 + 6N + 12NM.
	\end{gathered}
\end{equation}

\begin{equation}
	f_{while} = N \cdot (2 + 1 + 2 + 1) = 6N.
\end{equation}

Тогда:

\begin{equation}
	\begin{gathered}
		f_{jordan\_gaussian\_transform} = 3 + (N - 1) \cdot (6 + 3 + 6M + M \cdot f_{swap} +\\+ 3 + 6N+ 12NM)  + 6N + 3 = 6 + 6N + (N - 1) \cdot (12 + 6M +\\+ M \cdot f_{swap} + 6N + 12NM) = 6 + 6N + 12N - 12 + 6NM - 6M +\\+ M \cdot f_{swap} \cdot (N - 1) + 6N^2 - 6N + 12N^2M - 12NM =\\= -6 + 12N - 6M - 6NM + 6N^2 + 12N^2M + M \cdot f_{swap} \cdot (N - 1).
	\end{gathered}
\end{equation}

\section{Получение обратной диагональной матрицы}

Размер матрицы $N \times M$.

\begin{equation}
	\begin{gathered}
		f_{get\_inverse\_diagonal\_matrix} =\\= f_{resize} + 3 + N \cdot (2 + f_{resize} + 2 + 1 + 1 + 2) =\\= f_{resize} + 3 + N \cdot (8 + f_{resize}) =\\
		=f_{resize} \cdot (1 + N) + 3 + 8N.
	\end{gathered}
\end{equation}

\section{Получение матриц меньшей размерности}

Размер матрицы $N \times M$.

\begin{equation}
	f_{get\_reduced\_matrix\_best} = 1 + 2 \cdot f_{resize} + 1 + 2 + 1 + 2 = 7 + 2 \cdot f_{resize}.
\end{equation}

\begin{equation}
	\begin{gathered}
		f_{get\_reduced\_matrix\_worst} = 1 + f_{resize} + 2 + 2 + N \cdot (1 + f_{resize} + 2 +\\+ M \cdot (2 + 1 + 1 + 2 + 1)) = 1 + f_{resize} + 4 + N \cdot (3 + f_{resize} +\\+ 7M) =
		5 + f_{resize} + 3N + N \cdot f_{resize}. + 7NM = f_{resize} \cdot (1 + N) + 5 +\\+ 3N + 7NM.
	\end{gathered}
\end{equation}



\section{Получение сингулярных чисел и векторов}

Размер матрицы $N \times M$, размер вектора $K$.

\begin{equation}
	\begin{gathered}
		f_{compute\_evd} = f_{first\_if} + 3 + N + 1 + 2 + J \cdot (11 + N + N \cdot (3 + 2 +\\+ N \cdot 8) + 2 + N \cdot 4 + 1 + f_{sec\_if} + 1) + f_{third\_if} + f_{fourth\_if} =\\=
		f_{first_if} + 6 + N + J \cdot (11 + N + N \cdot (5 + 8N) + 4 + 4N) + f_{third\_if} +\\+ f_{fourth\_if}.
	\end{gathered}
\end{equation}

\begin{equation}
	\begin{gathered}
		f_{first\_if\_best} = 2;\\
		f_{first\_if\_worst} = 2 + 1 = 3.
	\end{gathered}
\end{equation}

\begin{equation}
	\begin{gathered}
		f_{sec\_if\_best} = 1;\\
		f_{sec\_if\_worst} = 1 + 8 + 4 = 13.
	\end{gathered}
\end{equation}

\begin{equation}
	\begin{gathered}
		f_{third\_if\_best} = 1;\\
		f_{third\_if\_worst} = 1 + 2 + N \cdot (1 + 2 + 8M) +\\+ f_{jordan\_gaussian\_transform} + f_{get\_hermitian\_matrix} +\\+ 2 \cdot f_{matrix\_by\_matrix} + f_{get\_hermitian\_matrix\_inverse} +\\+ f_{get\_reduced\_matrix} = 3 + 3N + 8NM -6 + 12N - 6M - 6NM +\\+ 6N^2 + 12N^2M + M \cdot f_{swap} \cdot (N - 1) + f_{resize} \cdot (1 + N) + 8 + 11N +\\+ 2 \cdot f_{resize} * (3N + 1) + 18N + 36NM + 54NM^2 +\\+ f_{resize} \cdot (1 + N) + 8 + 11N + f_{resize} \cdot (1 + N) + 5 + 3N + 7NM =\\= 18 + 58N - 6M + 45NM + 6N^2 + 12N^2M + 54NM^2 +\\+ M \cdot f_{swap} \cdot (N - 1) + f_{resize} \cdot (9N + 5).
	\end{gathered}
\end{equation}

\begin{equation}
	\begin{gathered}
		f_{fourth\_if\_best} = 1;\\
		f_{fourth\_if\_worst} = 2 + K \cdot (4 + N \cdot (3 + 6N) +\\+ f_{jordan\_gaussian\_transform} + 9 + 7N) = 2 + K \cdot (13 + 10N + 6N^2 -\\- 6 + 12N - 6M - 6NM + 6N^2 + 12N^2M + M \cdot f_{swap} \cdot (N - 1)) =\\= 2 + K (7 + 22N - 6M - 6NM + 12N^2 + 12N^2M + M \cdot f_{swap} \cdot\\\cdot (N-1)) = 2 + 7K + 22KN - 6KM - 6KNM =12KN^2 +\\+ 12KN^2M + KM \cdot f_{swap} \cdot (N - 1).
	\end{gathered}
\end{equation}

Т.~о. в худшем и лучшем случаях имеем:

\begin{equation}
	\begin{gathered}
		f_{compute\_evd\_best} = f_{first\_if} + 6 + N + J \cdot (11 + N + N \cdot (5 + 8N) + \\
		+ 4 + 4N + f_{sec\_if}) + f_{third\_if} +\\+ f_{fourth\_if} = 2 + 6 + N + J \cdot (15 + 10N + 8N^2 + 1) + 1 + 1 = \\
		= 10 + N + 16J + 10JN + 8JN^2.
	\end{gathered}
\end{equation}

\begin{equation}
	\begin{gathered}
		f_{compute\_evd\_worst} = f_{first\_if} + 6 + N + J \cdot (11 + N + N \cdot (5 + 8N) +\\+ 4 + 4N + f_{sec\_if}) + f_{third\_if} + f_{fourth\_if} = 
		3 + 6 + N + J \cdot (15 +\\+ 10N + 8N^2 + 13) + 18 + 58N - 6M + 45NM + 6N^2 + 12N^2M +\\+ 54NM^2 + M \cdot f_{swap} \cdot (N - 1) + f_{resize} \cdot (9N + 5)  + 12KN^2 +\\+ 12KN^2M + KM \cdot f_{swap} \cdot (N - 1) = 27 + 59N -6M +\\+ 12KN^2M + KM \cdot f_{swap} \cdot (N - 1) + 45NM + 54NM^2 + 12KN^2 +\\+
		f_{resize} \cdot (9N + 5).
	\end{gathered}
\end{equation}

\section{Сингулярное разложение}

Размер матрицы $N \times M$, размер вектора $K$.

\begin{equation}
	\begin{gathered}
		f_{svd} = 2 \cdot f_{matrix\_transpose} + 4 \cdot f_{matrix\_by\_matrix} +\\+ 
		f_{compute\_evd} + 2 + 5K + f_{get\_inverse\_diagonal\_matrix}.
	\end{gathered}
\end{equation}

\begin{equation}
	\begin{gathered}
		f_{svd\_best} = 2 \cdot f_{resize} \cdot (3N + 1) + 18N + 30MN + 4 \cdot  f_{resize} \cdot\\\cdot (3N + 1) + 36N + 72NM + 108NM^2 + 
		10 + N +\\+ 16J + 10JN + 8JN^2 + 2 + 5K + f_{resize} \cdot (1 + N) + 3 + 8N =\\=
		15 + 63N + 102NM + 16J + 10JN + 8JN^2 + 5K +\\+ f_{resize} \cdot (7 + 19N).
	\end{gathered}
\end{equation}

\begin{equation}
	\begin{gathered}
		f_{svd\_worst} = 2 \cdot f_{resize} \cdot (3N + 1) + 18N + 30MN + 4 \cdot\\\cdot f_{resize} \cdot (3N + 1) + 36N + 72NM + 108NM^2 +\\+ 
		27 + 59N -6M + 12KN^2M +\\+ KM \cdot f_{swap} \cdot (N - 1) + 45NM + 54NM^2 + 12KN^2 +\\+
		f_{resize} \cdot (9N + 5) + 2 + 5K + f_{resize} \cdot (1 + N) + 3 +\\+ 8N = f_{resize} \cdot (7 + 19N) + 32 + 113N + 147NM + 162NM^2 +\\+ 12KN^2 + 5K + KM \cdot f_{swap} \cdot (N - 1) + 12KN^2M.
	\end{gathered}
\end{equation}





\section*{Вывод}

В данном разделе была построена схема алгоритма расчета сингулярного разложения и была выведена трудоемкость для лучшего и худшего случаев сингулярного разложения.









