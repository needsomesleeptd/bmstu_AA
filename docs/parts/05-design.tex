\chapter{Конструкторская часть}
В данной части работы будет рассмотрен псевдокод линейного и бинарного алгоритмов поиска.





\section{Псевдокоды рассматриваемых алгоритмов}

В листинге~\ref{alg:bin_search} рассмотрен псевдокод алгоритма бинарного поиска, входными данными являются:
\begin{enumerate}
	\item $array$~--- массив элементов;
	\item $N$~--- число элементов в массиве;
	\item $target$~--- число для поиска в массиве.
\end{enumerate}

\begin{algorithm}
\caption{Псевдокод алгоритма бинарного поиска.}\label{alg:bin_search}
\begin{algorithmic}
	\Require {Последовательность упорядочена по не убыванию.}
	\State {$left \gets 0$}
	\State {$right \gets N - 1$}
	\While {$left <= right$}
		\State {$middle \gets (left + right)/2$}
		\If{$array[middle] \gets target$}
			\State \Return {$middle$}
		\EndIf
		\If {$array[middle] < target$}
			\State {$left \gets middle + 1$}
		\Else
			\State {$right \gets middle - 1$}
		\EndIf
	\EndWhile
	\State \Return {$-1$}
\end{algorithmic}
\end{algorithm}









\textbf{Вывод}

В данном разделе на основе теоретических данных были построены схемы
требуемых алгоритмов, а также для каждого алгоритма сортировки были выведены трудоемкости худшего и лучшего случаев.









