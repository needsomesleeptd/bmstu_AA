\chapter{Конструкторская часть}
В данной части работы будут рассмотрены схемы алгоритмов сортировок, а также схемы параллельной и последовательной обработкой конвейра.








\section{Схемы алгоритмов}
На рисунках~\ref{img:Linear}--\ref{img:handler_3} приведены схемы алгоритмов линейной обработки заявок и конвейерной обработки заявок, также представлена схема
каждого этапа обработки заявок.

\includeimage
{Linear} % Имя файла без расширения (файл должен быть расположен в директории inc/img/)
{f} % Обтекание (без обтекания)
{h} % Положение рисунка (см. figure из пакета float)
{0.9\textwidth} % Ширина рисунка
{Схема линейного алгоритма обработки заявок} % Подпись рисунка



\includeimage
{main_thre} % Имя файла без расширения (файл должен быть расположен в директории inc/img/)
{f} % Обтекание (без обтекания)
{h} % Положение рисунка (см. figure из пакета float)
{1\textwidth} % Ширина рисунка
{Схема алгоритма конвейерной обработки заявок} % Подпись рисунка


\includeimage
{handler_1} % Имя файла без расширения (файл должен быть расположен в директории inc/img/)
{f} % Обтекание (без обтекания)
{h} % Положение рисунка (см. figure из пакета float)
{0.8\textwidth} % Ширина рисунка
{Схема алгоритма 1 этапа обработки заявки (сортировка копии массива)} % Подпись рисунка

\includeimage
{handler_2} % Имя файла без расширения (файл должен быть расположен в директории inc/img/)
{f} % Обтекание (без обтекания)
{h} % Положение рисунка (см. figure из пакета float)
{0.8\textwidth} % Ширина рисунка
{Схема алгоритма 2 этапа обработки заявки (сортировка массива с использованием дополнительных потоков)} % Подпись рисунка

\includeimage
{handler_3} % Имя файла без расширения (файл должен быть расположен в директории inc/img/)
{f} % Обтекание (без обтекания)
{h} % Положение рисунка (см. figure из пакета float)
{0.8\textwidth} % Ширина рисунка
{Схема алгоритма 3 этапа обработки заявки (запись массива в файл)} % Подпись рисунка






\section*{Вывод}

В данном разделе на основе теоретических данных были построены схемы
требуемых алгоритмов, а также для каждого алгоритма сортировки были выведены трудоемкости худшего и лучшего случаев.









