\chapter{Конструкторская часть}
В данной части работы будут рассмотрены схемы алгоритмов сортировок, а также приведен расчет их трудоемкости.





\subsection{Схемы алгоритмов}
\includeimage
{koktail} % Имя файла без расширения (файл должен быть расположен в директории inc/img/)
{f} % Обтекание (без обтекания)
{h} % Положение рисунка (см. figure из пакета float)
{1\textwidth} % Ширина рисунка
{Схема алгоритма сортировки перемешиванием} % Подпись рисунка



\includeimage
{block} % Имя файла без расширения (файл должен быть расположен в директории inc/img/)
{f} % Обтекание (без обтекания)
{h} % Положение рисунка (см. figure из пакета float)
{1\textwidth} % Ширина рисунка
{Схема алгоритма блочной сортировки} % Подпись рисунка


\includeimage
{radix} % Имя файла без расширения (файл должен быть расположен в директории inc/img/)
{f} % Обтекание (без обтекания)
{h} % Положение рисунка (см. figure из пакета float)
{1\textwidth} % Ширина рисунка
{Схема алгоритма поразрядной сортировки} % Подпись рисунка




\subsection{Структуры данных}
Для реализации выбранных алгоритмов были использованы следующие структуры данных:
\begin{enumerate}
	\item Матрица --- массив векторов типа \texttt{int};
	\item Строка --- массив типа \texttt{wchar};
	\item Длина строки --- целое значение типа \texttt{size\_t}.
\end{enumerate}


\textbf{Вывод}

В данной части работы были описаны и разработаны алгоритмы  поиска расстояний Левенштейна и Дамерау-Левенштейна.









