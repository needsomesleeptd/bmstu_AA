\chapter{Конструкторская часть}
В данной части работы будут рассмотрены схемы алгоритмов различных реализаций сортировки слиянием.

\section{Требования к программному обеспечению}

К программе предъявлены ряд требований:

\begin{itemize}
	\item иметь интерфейс для выбора действий;
	\item динамически выделять память под массив данных;
	\item работа с массивами и <<нативными>> потоками.
\end{itemize}

\section{Требования к временным замерам}
Процессорное время --- это время, которое потратил процессор  на выполнение задачи. В данной работе, при использовании многопоточности возможно
ожидание одними потоками выполнения других потоков. Поток не выполняет
никаких действий в это время, поэтому простой не влияет на процессорное время, однако это влияет на результирующее реальное время, поэтому для корректного сравнения различных реализаций  по времени работы стоит замерять и сравнивать реальное время
выполнения реализаций алгоритмов.

\section{Схемы алгоритмов}

На рисунках \ref{img:mergeSort}--\ref{img:merge} приведены схемы алгоритмов различных вариаций алгоритма сортировки слиянием

\includeimage
{mergeSort} % Имя файла без расширения (файл должен быть расположен в директории inc/img/)
{f} % Обтекание (без обтекания)
{H} % Положение рисунка (см. figure из пакета float)
{1\textwidth} % Ширина рисунка
{Схема алгоритма сортировки слиянием, при использовании 1 потока} % Подпись рисунка

\includeimage
{mergeSortMultiThread} % Имя файла без расширения (файл должен быть расположен в директории inc/img/)
{f} % Обтекание (без обтекания)
{H} % Положение рисунка (см. figure из пакета float)
{1\textwidth} % Ширина рисунка
{Схема алгоритма сортировки слиянием, при использовании нескольких потоков} % Подпись рисунка

\includeimage
{merge} % Имя файла без расширения (файл должен быть расположен в директории inc/img/)
{f} % Обтекание (без обтекания)
{H} % Положение рисунка (см. figure из пакета float)
{0.9\textwidth} % Ширина рисунка
{Схема алгоритма слияния отсортированных последовательностей} % Подпись рисунка

\newpage

Алгоритм слияния отсортированных последовательностей используется во всех рассматриваемых версиях алгоритма, в случае использования нескольких потоков, слияние будет происходить в отдельном потоке.

При использовании нескольких потоков (схема на рис.~\ref{img:mergeSortMultiThread}), число доступных потоков делится на 2 при каждом шаге рекурсии, в таком случае число доступных потоков поровну разделяется при каждом разбиении массива на подмассивы для слияния.

\section*{Вывод}

В данном разделе  были построены схемы рассматриваемых алгоритмов.

