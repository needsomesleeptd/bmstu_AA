\chapter{Аналитическая часть}
В данной части работы будут рассмотрены бинарный и линейный алгоритмы поиска.






\section{Линейный поиск}
При использовании линейного поиска рассматриваются все элементы массива до встречи элемента, совпадающего с требуемым. Таким образом в худшем случае (отсутствия элемента в массиве) необходимо рассмотреть все элементы массива. Алгоритм имеет асимптотическую оценку~$O(n)$~\cite{book_knut}.



\section{Бинарный поиск}
При использовании данного алгоритма предполагается упорядоченность данных, т.~е. элементы массива упорядочены либо по не возрастанию либо по не убыванию. Изначально необходимо сравнить искомый элемент $k$ с элементом массива, находящимся в середине, результат сравнения поможет определить в какой половине массива продолжить поиск, далее возможно применение той же процедуры к выбранному подмассиву и т.~д. После не более чем 
$log_{2}{N}$ сравнений либо ключ будет найден, либо будет установлено его отсутствие. Такая процедура иногда называется <<логарифмическим поиском>> или <<методом деления пополам>>, но наиболее
употребительный термин~---~бинарный поиск~\cite{book_knut}.









\section*{Вывод}
В данной части были рассмотрены идеи линейного и бинарного алгоритмов поиска.
