\chapter{Аналитическая часть}
В данной части работы будут рассмотрены основы конвейерной обработки данных, а также будет описан алгоритм сортировки слиянием.

\section{Конвейерная обработка данных}
Конвейеризация (или конвейерная обработка) в общем случае основана на разделении подлежащей исполнению функции на более мелкие части, называемые ступенями, и выделении для каждой из них отдельного блока аппаратуры.
Производительность при этом возрастает благодаря тому, что одновременно на различных ступенях конвейера выполняются несколько команд.
Конвейерная обработка такого рода широко применяется во всех современных быстродействующих процессорах~\cite{conveyor_desc}.

\section{Сортировка слиянием}
Алгоритм сортировки слиянием (англ. merge sort) является эффективным и стабильным алгоритмом сортировки, который применяет принцип <<разделяй и властвуй>> для упорядочивание элементов в массиве \cite{merge-sort}.

Алгоритм состоит из нескольких этапов~\cite{merge-sort}.
\begin{enumerate}
	\item Разделение ---  исходный массив разделяется на две равные (или почти равные) половины.
	Это делается путем нахождения середины массива и создания двух новых массивов, в которые будут скопированы элементы из левой и правой половин.
	
	\item Рекурсивная сортировка: каждая половина массива рекурсивно сортируется с помощью алгоритма сортировки слиянием.
	Этот шаг повторяется до тех пор, пока размер каждой половины не станет равным 1.
	
	\item Слияние: отсортированные половины массива объединяются в один отсортированный массив.
	Для этого создается новый массив, в который будут последовательно добавляться элементы из левой и правой половин.
	При добавлении элементов выбирается наименьший элемент из двух половин и добавляется в новый массив.
	Этот шаг повторяется до тех пор, пока все элементы не будут добавлены в новый массив.
\end{enumerate}
В результате выполнения этих шагов получается отсортированный массив, который содержит все элементы исходного массива.



\section*{Вывод}
В данной части работы была рассмотрена конвейерная обработка данных и алгоритм сортировки слиянием, а также описаны конкретные операции, выполняющиеся в данной работе при конвейерной обработке.
