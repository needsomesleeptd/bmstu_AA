\chapter{Аналитическая часть}
В данной части работы будет описан алгоритм сингулярного разложения матриц.






\section{Сингулярное разложение матриц}
Сингулярное разложение $А$ обозначают
$SVD(A)$, опишем его в виде равенства~\cite{SVD}:
\begin{equation}
	A = USV^{T}.
	\label{eq:SVD_def}
\end{equation}
Матрица $S$ всегда диагональная, ее коэффициенты –-- неотрицательные вещественные числа $\sigma_{1}, \cdots, \sigma_{n}$, расположенные на главной диагонали матрицы, называются сингулярными числами, сингулярные числа являются корнем из собственных чисел матриц. Столбцы матрицы $V$ называются правыми сингулярными векторами и всегда ортогональны 
друг другу. Столбцы матрицы $U$ называются левыми сингулярными векторами и также ортогональны друг другу. Матрицы $U$ и $V$ являются унитарными, т. е сумма квадратов значений каждого столбца матриц равняется единице. Их можно использовать в качестве нового базиса системы координат для представления данных, записанных в матрице $A$~\cite{SVD}.

\section{Алгоритм сингулярного разложения матриц}
Для разложения, необходимо получить сингулярные числа некоторый матрицы, являющейся  произведением исходной матрицы $A$ и ее транспонированной копии $A^{T}$. После чего необходимо найти собственные числа результата произведения, из которых будут получены сингулярные числа~\cite{SVD,pers_val}.

После того как сингулярные числа получены возможно получение матрицы $S$, путем расстановки сингулярных чисел по диагонали матрицы в порядке убывания~\cite{SVD_algo}.

Правые сингулярные векторы будут получены с помощью полученных сингулярных чисел и матрицы $AA^{T}$, пусть дано сингулярное число $\sigma_{1}$ и матрица $AA^{T}$~см. выражение~(\ref{eq:r_eigenvectors_1}). После вычисления значения выражения~(\ref{eq:r_eigenvectors_2}), будет получена матрица коэффициентов системы для 3 неизвестных~(\ref{eq:r_eigenvectors_3}). Поле ее решения будет получен правый сингулярный вектор для сингулярного числа~$\sigma_{1}$~(\cite{SVD_algo}).
\begin{equation}
	\sigma_{1}=15.4, A^{T}A=\begin{pmatrix}
	10 & 5 & 2\\ 
	5 & 6  &5\\
		2 & 5 & 5
	\end{pmatrix}
	\label{eq:r_eigenvectors_1}
\end{equation}

\begin{equation}
	A^{T}A - \sigma_{1}I = \begin{pmatrix}
		10 - 15.4 & 5 & 2\\ 
		5 & 6 -15.4  &5\\
		2 & 5 & 5 - 15.4
	\end{pmatrix}
	\label{eq:r_eigenvectors_2}
\end{equation}

\begin{equation}
	\begin{pmatrix}
		-5.43 - 15.4 & 5 & 2\\ 
		5 & 6 -9.43  &5\\
		2 & 5 & 5 - 10.43
	\end{pmatrix} = 0
	\label{eq:r_eigenvectors_3}
\end{equation}

После получения сингулярных векторов, необходимо их нормализовать и построчно записать  в матрицу, таким образом получая матрицу $V$.

После получения $S$,$V$, получение левых сингулярных векторов возможно с использованием выражения~(\ref{eq:l_eigenvectors}). Таким образом все составляющие сингулярного разложения были получены~\cite{SVD_algo}.
\begin{equation}
	U = AV^{T}S^{-1}
	\label{eq:l_eigenvectors}
\end{equation}















\section*{Вывод}
В данной части было рассмотрено представление сингулярного разложения и алгоритм его получения.
