\chapter{Аналитическая часть}
В данной части работы будут рассмотрены алгоритмы сортировок: перемешиванием, блочная и поразрядная. Также будет определена решаемая задача.






\section{Сортировка перемешиванием}
Алгоритм состоит из последовательных про проходов по массиву в 2 направлениях (от начала массива к его концу и наоборот). При каждом проходе 
происходит попарное сравнение ближайших значений, в случае если их порядок противоречит возрастающему, значения меняются местами и позиция,
после которой произошло изменение порядка запоминается в переменной $left$ в случае прохода от начала до конца массива и в переменной $right$ иначе.
Все следующие проходы по массиву происходят от позиции $left$ до $right$ пока данные позиции не равны. Таким образом при каждом проходе устанавливается
наибольшее и наименьшее значения в индексах массива от $left$ до $right$ \cite{conctail_sort,article_sorts}.

\section{Поразрядная сортировка}
Смысл данной сортировки в том, что данные делятся сначала по разрядам и сортируются внутри каждого разряда.
Сам алгоритм происходит в несколько этапов.
\begin{enumerate}
	\item Алгоритм инициализирует индекс рассматриваемого разряда в числах;
	\item после чего получает значение данного разряда каждого из чисел с помощью остатка деления на основание системы счисления;
	\item затем полученные цифры сортируются;
	\item элементы расставляются в соответствии со своими цифрами.
\end{enumerate}
Данный алгоритм повторяется пока индекс рассматриваемого разряда не будет больше числа всех разрядов в числе \cite{book_knut,article_sorts}.

\section{Блочная сортировка}
Идея заключается в  разбиении входных данных на <<блоки>> одинакового размера, после чего данные в блоках сортируются и результаты сортировок объединяются.
Отсортированная последовательность получается путём последовательного перечисления элементов каждого блока.
Для деления данных на блоки, алгоритм предполагает, что значения  распределены равномерно, и распределяет элементы по блокам равномерно. Например, предположим, что данные имеют значения в диапазоне от 1 до 100 и алгоритм использует 10 блоков. Алгоритм помещает элементы со значениями 1‑10 в первый блок, со значениями 11‑20 — во второй, и т.д.
Если элементы распределены равномерно, в каждый блок попадает примерно одинаковое число элементов. Если в списке $N$ элементов, и алгоритм использует $N$ блоков, в каждый блок попадает всего один или два элемента, поэтому возможно отсортировать элементы за конечное число шагов \cite{article_sorts}.






\textbf{Вывод}
В данной части были рассмотрены идеи поразрядной, блочной сортировки и сортировки перемешиванием.
