\chapter{Аналитическая часть}
В данной части работы будет описана задача коммивояжера,  а также будут описаны способы ее решения~--- метод полного перебора и метод на основе муравьиного алгоритма.

\section{Задача коммивояжера}

Задача коммивояжера занимает особое место в комбинаторной оптимизации и  исследовании  операций. Исторически она 
была одной из тех задач, которые послужили толчком для развития 
этих направлений. В данной задаче рассматривается $n$ городов и матрица попарных расстояний между ними. 
Требуется найти такой порядок посещения городов, чтобы  суммарное пройденное расстояние было минимальным, каждый город посещался ровно один раз и коммивояжер вернулся в тот город, с которого начал свой маршрут.  
Другими словами, во взвешенном полном графе требуется найти гамильтонов цикл минимального веса~\cite{salesman}. 

\section{Решение с использованием полного перебора}
Суть данного решения состоит в переборе всех возможных вариантов замкнутых путей и в выборе кратчайшего из них. Данное решение имеет оценку в $O(n!)$, что затрудняет его использование в графах с большим количеством вершин.

\section{Решение с использованием муравьиного алгоритма}
При поиске путей до пищи муравьи общаются друг с другом с помощью феромонов.
В случае если путь длинный, феромоны испарятся и последующие сородичи предпочтут иной путь с большим числом феромонов, таким образом наибольшее число феромона будет оставлено на кратчайших путях~\cite{shtovba}.

Муравей имеет следующие органы чувств:
\begin{enumerate}[label=\arabic*)]
	\item зрение --- муравей способен определить длину ребра;
	\item память --- муравей способен запомнить пройденный маршрут;
	\item обоняние --- муравей способен чуять феромон.
\end{enumerate}

Введем функцию~(\ref{eq:d_func}), характеризующую привлекательность ребра, определяемую благодаря зрению.

\begin{equation}
	\label{eq:d_func}
	\eta_{ij} = 1 / D_{ij},
\end{equation}
где $D_{ij}$ — расстояние от текущего пункта $i$ до заданного пункта $j$.

Вероятность перехода из пункта $i$ в пункт $j$ определяется по формуле~(\ref{eq:prob}).
\begin{equation}
	\label{eq:prob}
	P_{i,j}={\frac {(\tau_{i,j}^{\alpha})(\eta_{i,j}^{\beta })}{\sum (\tau_{i,j}^{\alpha})(\eta_{i,j}^{\beta})}},
\end{equation}
где:
\begin{enumerate}
	\item $\tau_{i,j}$ --- количество феромонов на ребре от $i$ до $j$;
	\item $\eta_{i,j}$ --- привлекательность пути от $i$ до $j$;
	\item $\alpha$ --- параметр влияния расстояния;
	\item $\beta$ --- параметр влияния феромона.
\end{enumerate}

При $\alpha = 0$ будет выбран ближайший город, что соответствует жадному алгоритму в классической теории оптимизации. Если $\beta=0$, работает лишь феромонное усиление, что влечет за собой быстрое вырождение маршрутов к одному субоптимальному решению~\cite{shtovba}.

После завершения пути, ночью, происходит обновление феромона на пройденных путях по формуле~(\ref{eq:upd_ph}), в случае, если $p$~--- коэффициент испарения феромона, $N$~--- количество феромонов, $Q$~--- некоторая константа порядка длины путей, $L_{k}$~--- длина пути муравья с номером $k$~\cite{shtovba}.
\begin{equation}\label{eq:upd_ph}
	\tau_{ij}(t+1) = (1-p)\tau_{ij}(t) + \Delta \tau_{ij},~~\Delta \tau_{ij} =
	\displaystyle\sum_{k=1}^N \tau^k_{ij},
\end{equation}
где
\begin{equation}\label{eq:3}
	\Delta \tau^k_{ij} = \begin{cases}
		\frac{Q}{L_k}, & \quad \textrm{ребро посещено $k$-ым муравьем,} \\
		0, & \quad \textrm{иначе.}
	\end{cases}
\end{equation}

Поскольку вероятность (\ref{eq:prob}) перехода в заданную точку не должна быть равна нулю, необходимо обеспечить неравенство $\tau_{ij} (t)$ нулю посредством введения дополнительного минимально возможного значения феромона $\tau_{min}$ и в случае, если $\tau_{ij} (t+1)$ принимает значение, меньшее $\tau_{min}$, откатывать феромон до этой величины.

\section*{Вывод}
В данной части работы были рассмотрены идеи, необходимые для разработки и реализации двух рассматриваемых алгоритмов решения задачи коммивояжера: алгоритма полного перебора и муравьиного алгоритма.
