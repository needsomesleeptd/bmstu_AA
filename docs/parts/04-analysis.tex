\chapter{Аналитическая часть}
\label{sec:targets}
Целью данной лабораторной работы является описание и исследование алгоритмов поиска расстояний Левенштейна и Дамерау-Левенштейна.


Для поставленной цели необходимо выполнить следующие задачи.
\begin{enumerate}[label={\arabic*)}]
	\item Исследовать расстояние Левенштейна.
	\item Разработка алгоритмов поиска расстояний Левенштейна, Дамерау-Левенштейна.
	\item Создать программное обеспечение, реализующее следующие алгоритмы.
	\item Провести исследование, затрачиваемого процессорного времени и памяти при различных реализациях алгоритмов.
	\item Провести сравнительный анализ алгоритмов.
\end{enumerate}

\section{Расстояние Левенштейна}


Расстояние Левенштейна~\cite{levenshtein} (редакционное расстояние, дистанция редактирования) --- метрика, измеряющая разность между двумя последовательностями символов.

Вводятся операции нескольких типов:
\begin{enumerate}[label=\arabic*.]
	\item I (англ. insert) --- операция вставки символа.
	\item D (англ. delete) --- операция удаления символа.
	\item R (англ.replace) --- операция замены символа.
\end{enumerate}
Также введем символ $\lambda$, обозначающий пустой символ строки, не входящий ни в одну из рассматриваемых строк.

Будем считать стоимость каждой вышеизложенной операции равной~1:
\begin{itemize}[label=---]
	\item $D(a, b) = 1$, $a \neq b$, в противном случае замена не происходит;
	\item $D(\lambda, b) = 1$;
	\item $D(a, \lambda) = 1$.
\end{itemize}


Также обозначим совпадение символов как M (англ. match), таким образом $D(a,a) = 0$.
Существует  проблема взаимного выравнивания строк, в случае когда строки различной длины существует множество способов сопоставить соответствующие символы.

Решим проблему введением рекуррентной формулы, обозначим:
\begin{enumerate}[label*=\arabic*.]
	\item $L1$ --- длина $S_{1}$.
	\item $L2$ --- длина $S_{2}$.
	\item $S_{1}[1...i]$ --- подстрока $S_{1}$ длиной $i$, начиная с первого символа.
	\item $S_{2}[1...j]$ --- подстрока $S_{2}$ длиной $j$, начиная с первого символа.
\end{enumerate}



Расстояние Левенштейна между двумя строками $S_{1}$ и $S_{2}$ длиной $M$ и $N$ соответственно рассчитывается по рекуррентной формуле:
\begin{small}
\begin{equation}
	\label{eq:lev}
	D(S_{1}[1..i],S_{2}[1..j]) = \left\{ \begin{array}{ll}
	 j, & \textrm{$\mbox{если }i = 0$}\\
	 i, & \textrm{$\mbox{если }j = 0, i > 0$}\\
	min(D(S_{1}[1..i], S_{2}[1.. j - 1])+1,\\
	\qquad D(S_{1}[1..i - 1], S_{2}[1..j]) + 1, &\textrm{$\mbox{если }i>0, j>0$}\\
	\qquad D(S_{1}[1..i - 1], S_{2}[1..j - 1]) + m(S_{1}[i],S_{2}[j]),
	  \end{array} \right,
	\end{equation}
\end{small}
где сравнение символов строк $S_{1}$ и $S_{2}$ рассчитывается как:
\begin{equation}
	\label{eq:m}
	m(a, b) = \begin{cases}
		0 &\text{если a = b,}\\
		1 &\text{иначе.}
	\end{cases}
\end{equation}


В расстоянии Дамерау-Левенштейна вводится еще одна операция, обозначим ее как $S$ (англ. swap) --- данная операция применима, только  если $S_{1}[i] = S_{2}[j - 1]$
и $S_{1}[i - 1] = S_{2}[j]$. Рекуррентная формула  данной метрики выглядит следующим образом:
\begin{tiny}
\begin{equation}
	\label{eq:DL}
	D(S_{1}[1\cdots i],S_{2}[1 \cdots j]) = \left\{ \begin{array}{ll}
		j, & \textrm{$\mbox{если }i = 0$}\\
		i, & \textrm{$\mbox{если }j = 0, i > 1$}\\ 
		min(D(S_{1}[1..i], S_{2}[1.. j - 1]) + 1,\\
		\qquad D(S_{1}[1..i - 1], S_{2}[1..j]) + 1,\\
		\qquad D(S_{1}[1..i - 1], S_{2}[1..j - 1]) + \\
		\qquad+\left[ 
		\begin{array}{ccc}
			0, & \textrm{$\mbox{если }S_{1}[i] == S_{2}[j],$}\\
			1, & \textrm{иначе}
		\end{array} 
		\right.\\
		\qquad \left[
		\begin{array}{ccc}
			D(S_{1}[1..i - 2], S_{2}[1..j - 2]) + 1), & \textrm{$\mbox{если }i > 1, j > 1,$}\\
			& \textrm{$S_{1}[i] == S_{2}[j - 1],$}\\
			& \textrm{$S_{1}[i - 1] == S_{2}[j]$}\\
			+{\infty}, & \textrm{иначе}
		\end{array}
	\right.\\
		
		min(D(S_{1}[1..i], S_{2}[1..j - 1]) + 1,\\
		\qquad D(S_{1}[1..i - 1], S_{2}[1..j]) + 1, \\
		\qquad D(S_{1}[1..i - 1], S_{2}[1..j - 1]) + \\
		\qquad+\left[ 
		\begin{array}{ccc}
			0, & \textrm{$\mbox{если }S_{1}[i] == S_{2}[j],$}\\
			1, & \textrm{иначе}
		\end{array} 
		\right.), &\textrm{$\mbox{если }i>0, j>0$}
	\end{array} \right.
\end{equation}
\end{tiny}
Идея в том, что после замены пары символов местами, полученные пары были поэлементно равны друг другу.

\textbf{Вывод:}

В данном разделе были рассмотрены цели и задачи работы, а также введены необходимые обозначения для поиска расстояний Левенштейна и Дамерау-Левенштейна
и выведены рекуррентные отношения для поиска их значения.

