\chapter{Аналитическая часть}

Целью данной лабораторной работы является описание и исследование алгоритмов поиска расстояний Левенштейна и Дамерау-Левенштейна.


Для поставленной цели необходимо выполнить следующие задачи.
\begin{enumerate}[label={\arabic*)}]
	\item Исследовать расстояние Левенштейна.
	\item Разработка алгоритмов поиска расстояний Левенштейна, Дамерау-Левенштейна.
	\item Создать программное обеспечение, реализующее следующие алгоритмы.
	\item Провести исследование, затрачиваемого процессорного времени и памяти при различных реализациях алгоритмов.
	\item Провести сравнительный анализ алгоритмов.
\end{enumerate}

\section{Расстояние Левенштейна}


Расстояние Левенштейна~\cite{levenshtein} (редакционное расстояние, дистанция редактирования) --- метрика, измеряющая разность между двумя последовательностями символов.

Вводятся операции нескольких типов:
\begin{enumerate}[label=\arabic*]
	\item I (англ. insert) --- операция вставки символа.
	\item D (англ. delete) --- операция удаления символа.
	\item R (англ.replace) --- операция замены символа.
\end{enumerate}
Также введем символ $\lambda$, обозначающий пустой символ строки, не входящий ни в одну из рассматриваемых строк.

Будем считать стоимость каждой вышеизложенной операции равной 1:
\begin{itemize}[label=---]
	\item $D(a, b) = 1$, $a \neq b$, в противном случае замена не происходит;
	\item $D(\lambda, b) = 1$;
	\item $D(a, \lambda) = 1$.
\end{itemize}


Также обозначим совпадение символов как M (англ. match), таким образом $D(a,a) = 0$.
Существует  проблема взаимного выравнивания строк, в случае когда строки различной длины существует множество способов сопоставить соответствующие символы.

Решим проблему введением рекуррентной формулы, обозначим:
\begin{enumerate}[label*=\arabic*.]
	\item $L1$ --- длина $S_{1}$.
	\item $L2$ --- длина $S_{2}$.
	\item $S_{1}[1...i]$ --- подстрока $S_{1}$ длиной $i$, начиная с первого символа.
	\item $S_{2}[1...j]$ --- подстрока $S_{2}$ длиной $j$, начиная с первого символа.
\end{enumerate}



Расстояние Левенштейна между двумя строками $S_{1}$ и $S_{2}$ длиной $M$ и $N$ соответственно рассчитывается по рекуррентной формуле:
\begin{equation}
	\label{eq:lev}
	D(i,j) = \left\{ \begin{array}{ll}
	 j, & \textrm{$\mbox{если }i = 0$}\\
	 i, & \textrm{$\mbox{если }j = 0, i > 0$}\\
	min(D(S1[1..i], S2[1.. j - 1])+1,\\
	\qquad D(S1[1..i - 1], S2[1..j]) + 1, &\textrm{$\mbox{если }i>0, j>0$}\\
	\qquad D(S1[1..i - 1], S2[1..j - 1]) + m(S1[i],S2[j]),
	  \end{array} \right,
	\end{equation}
где сравнение символов строк $S_1$ и $S_2$ рассчитывается как:
\begin{equation}
	\label{eq:m}
	m(a, b) = \begin{cases}
		0 &\text{если a = b,}\\
		1 &\text{иначе.}
	\end{cases}
\end{equation}


В расстоянии Дамерау-Левенштейна вводится еще одна операция, обозначим ее как $S$ (англ. swap) --- данная операция применима, только  если $S_{1}[i] = S_{2}[j - 1]$
и $S_{1}[i - 1] = S_{2}[j]$. Рекуррентная формула  данной метрики выглядит следующим образом:
\begin{equation}
	\label{eq:damerau}
	D(i,j) = \left\{ \begin{array}{ll}
		j, & \textrm{$\mbox{если }i = 0$}\\
		i, & \textrm{$\mbox{если }j = 0, i > 1$}\\ 
		min(D(S1[1..i], S2[1.. j - 1]) + 1,\\
		\qquad D(S1[1..i - 1], S2[1..j]) + 1,\\
		\qquad D(S1[1..i - 1], S2[1..j - 1]) + \\
		\qquad+\left[ 
		\begin{array}{ccc}
			0, & \textrm{$\mbox{если }S1[i] == S2[j],$}\\
			1, & \textrm{иначе}
		\end{array} 
		\right.\\
		\qquad \left[
		\begin{array}{ccc}
			D(S1[1..i - 2], S2[1..j - 2]) + 1), & \textrm{$\mbox{если }i > 1, j > 1,$}\\
			& \textrm{$S1[i] == S2[j - 1],$}\\
			& \textrm{$S1[i - 1] == S2[j]$}\\
			+{\infty}, & \textrm{иначе}
		\end{array}
	\right.\\
		
		min(D(S1[1..i], S2[1..j - 1]) + 1,\\
		\qquad D(S1[1..i - 1], S2[1..j]) + 1, \\
		\qquad D(S1[1..i - 1], S2[1..j - 1]) + \\
		\qquad+\left[ 
		\begin{array}{ccc}
			0, & \textrm{$\mbox{если }S1[i] == S2[j],$}\\
			1, & \textrm{иначе}
		\end{array} 
		\right.), &\textrm{$\mbox{если }i>0, j>0$}
	\end{array} \right.
\end{equation}
Идея в том, что после замены пары символов местами, полученные пары были поэлементно равны друг другу.



\subsection{Нерекурсивный алгоритм нахождения расстояния Левенштейна}

Рекурсивная реализация алгоритма Левенштейна малоэффективна по времени при больших $M$ и $N$, так как множество промежуточных значений  Для оптимизации можно использовать итерационную реализацию заполнения матрицы промежуточными значениями $D(i,j)$.

В качестве структуры данных для хранения промежуточных значений можно использовать матрицу, имеющую размеры:
\begin{equation}
	\label{eq:lev-m-size}
	(N + 1) \times (M + 1)
\end{equation}

Значения в ячейке $[i, j]$ равно значению $D(S_{1}[1...i], S_{2}[1...j])$. Первый элемент матрицы заполнен нулем. Всю таблицу заполнять в соответствии с формулой~(\ref{eq:L}). 

Однако матричный алгоритм является малоэффективным по памяти по сравнению с рекурсивным при больших $M$ и $N$, т.к. множество промежуточных значений $D(i,j)$ хранится в памяти после их использования. Для оптимизации по памяти рекурсивного алгоритма нахождения расстояния Левенштейна можно использовать кеш, т.е. пару строк, содержащую значения $D(i,j)$, вычисленные в предыдущей итерации, алгоритма и значения $D(i,j)$, вычисляемые в текущей итерации.

\section{Расстояние Дамерау-Левенштейна}
Расстояние Дамерау-Левенштейна, названное в честь ученых Фредерика Дамерау и Владимир Левенштейна, --- это мера разницы двух строк символов, определяемая как минимальное количество операций вставки, удаления, замены и транспозиции (перестановки двух соседних символов), необходимых для перевода одной строки в другую. Является модификацией расстояния Левенштейна: к трем базовым операциям добавляется операция транспозиции $T$ (от англ. transposition).

Расстояние Дамерау-Левенштейна может быть вычислено по рекуррентной формуле:

\begin{equation}
	\label{eq:DL}
	D(i, j) = 
	\begin{cases}
		0, &\text{i = 0, j = 0,}\\
		i, &\text{j = 0, i > 0,}\\
		j, &\text{i = 0, j > 0,}\\
		min \begin{cases}
			D(i, j - 1) + 1,\\
			D(S_{1}[1...i-1], S_{2}[1...j]) + 1,\\
			D(S_{1}[1...i-1], S_{2}[1...j-1]) + m(S_{1}[i], S_{2}[j]), \\
			D(S_{1}[1...i-2], S_{2}[1...j-2]) + 1, \\
		\end{cases}
		& \begin{aligned}
			& \text{если i > 1, j > 1}, \\
			& S_{1}[i] = S_{2}[j - 1], \\
			& S_{1}[i - 1] = S_{2}[j], \\
		\end{aligned}\\
		min \begin{cases}
			D(i, j - 1) + 1,\\
			D(S_{1}[1...i-1], S_{2}[1...j]) + 1, \\
			D(S_{1}[1...i-1], S_{2}[1...j-1]) + m(S_{1}[i], S_{2}[j]), \\
		\end{cases}
		 & \text{иначе.}
	\end{cases}
\end{equation}


В данном разделе были рассмотрены алгоритмы динамического программирования --- алгоритмы нахождения расстояний Левенштейна и Дамерау-Левенштейна, формулы которых задаются рекуррентно, а следовательно, данные алгоритмы могут быть реализованы рекурсивно и итеративно. На вход алгоритмам поступают две строки, которые могут содержать как русские, так и английские буквы, также будет предусмотрен ввод пустых строк.
