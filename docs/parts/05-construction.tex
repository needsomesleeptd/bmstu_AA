\chapter{Конструкторская часть}

Рассмотрим различные реализации алгоритмов поиска расстояния Левенштейна и Дамерау-Левенштейна для строк $S_{1},S_{2}$, каждая имеет
длину $N и M$ соответственно.


\section{Реализации алгоритмов}

\subsection{Использование матрицы}
\label{subsec:matrix_math_desc}
Вводится матрица $M$ с $N + 1$ на $M + 1$ элементами. Номер строки матрицы $i$ соответствует длине подстроки $S_{1}$, номер столбца $j$ --- длине подстроки $S_{2}$.
Значение $M[i][j]$ соответствует расстоянию Левенштейна между подстроками, $S_{1}[i] \space \text{и} \space S_{2}[j]$ соответственно. После чего используется формула \ref{eq:lev} или
\ref{eq:DL} для расчета значения в данной клетке, таким образом для получения значения, необходимо рассмотреть диагональную, верхнюю и нижнюю клетки (в \ref{eq:DL} также рассматривается клетка $M[i-2][j-2]$).


\subsection{Использование рекурсии}
\label{subsec:recurse_math_desc}
При использовании рекуррентной формулы возможно использование рекурсии.
На вход подаются 2 строки, а также их длины на данный момент, получив строки с длинами $i$ и $j$, необходимо рассмотреть 
те же строки с длинами $(i-1,j-1)$,$(i,j-1)$,$(i-1,j)$ и расчитать их стоимость  с помощью формул \ref{eq:lev} и \ref{eq:DL} (в \ref{eq:DL} также рассматривается строка $(i-2,j-2)$). 
После чего необходимо выбрать вариант с наименьшей стоимость из возможных.

\subsection{Использование рекурсии с мемоизацией}
\label{subec:memorysation_descr}
Заметим, что метод, описанный в пункте \ref{subsec:recurse_math_desc}, можно ускорить, если запоминать уже посчитанные значения, и использовать их в дальнейшем.
Таким образом вводится матрциа $M$, аналогичная матрице из пункта \ref{subsec:matrix_math_desc}, все ячейки которой заполняются $+\infty$, после чего
перед вычислением текущего значения происходит проверка: в случае, если значение $M[i][j] = +\infty$, вычисляется текущее значение и заносится в матрицу, иначе
вычисления значения не происходит и значение сразу берется из ячейки $M[i][j]$.

\subsection{Схемы алгоритмов}
%TODO - insert block schemes
\begin{figure}[H]
	\centering
	\includesvg[height=1.3\textwidth,width=1\textwidth]{../img/lab_01_AA_table_Lev-lev.svg}
	\caption{Схема алгоритма поиска расстояния Левенштейна с помощью матрицы}
	\label{fig:leb_algo_matrix}
\end{figure}

\begin{figure}[H]
	\centering
	\includesvg[height=1.3\textwidth,width=1\textwidth]{../img/lab_01_AA_table_Lev-Damerau-lev.svg}
	\caption{Схема алгоритма поиска расстояния Дамерау-Левенштейна с помощью матрицы}
	\label{fig:dam_algo_matrix}
\end{figure}

\begin{figure}[H]
	\centering
	\includesvg[height=1.3\textwidth,width=1\textwidth]{../img/lab_01_AA_table_Lev-Damerau-mem.svg}
	\caption{Схема алгоритма поиска расстояния Дамерау-Левенштейна с помошью рекурсии с мемоизацией}
	\label{fig:dam_algo_rec_mem}
\end{figure}

\begin{figure}[H]
	\centering
	\includesvg[height=1.3\textwidth,width=1\textwidth]{../img/lab_01_AA_table_Lev-Damerau-recurse.svg}
	\caption{Схема алгоритма поиска расстояния Дамерау-Левенштейна с помошью рекурсии}
	\label{fig:dam_algo_rec}
\end{figure}


\subsection{Структуры данных}
Для реализации выбранных алгоритмов были использованы следующие структуры данных:
\begin{enumerate}[label*=\arabic*.]
	\item Матрица --- массив векторов типа int.
	\item Строка --- массив типа wchar.
	\item Длина строки --- целое значение типа size\_t.
\end{enumerate}


\textbf{Вывод}

На данном этапе были рассмотрены алгоритмы и записаны их схемы, также были описаны используемые структуры данных.







