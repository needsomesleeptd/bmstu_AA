\chapter*{\hfill{\centering  ВВЕДЕНИЕ}\hfill}

\addcontentsline{toc}{chapter}{ВВЕДЕНИЕ}

Разработчики архитектуры компьютеров издавна прибегали к методам проектирования, известным под общим названием <<совмещение операций>>, при котором аппаратура компьютера в любой момент времени выполняет одновременно более одной базовой операции~\cite{conveyor}.

Этот метод включает в себя, в частности, такое понятие, как конвейеризация.
Конвейеры широко применяются программистами для решения трудоемких задач, которые можно разделить на этапы, а также в большинстве современных быстродействующих процессоров~\cite{conveyor}.

В качестве операций, выполняющихся на конвейере в данной работе, взяты следующие:
\begin{enumerate}
	\item сортировка слиянием копии массива, предоставленного в заявке;
	\item сортировка слиянием исходного массива с использованием 4 дополнительных потоков;
	\item запись значений отсортированного массива в файл, соответствующий номеру заявки.
\end{enumerate}

Целью данной работы является исследование быстродействия метода конвейерной обработки данных.

В рамках выполнения работы необходимо решить следующие задачи: 
\begin{enumerate}
	\item описать организацию конвейерной обработки данных;
	\item описать алгоритмы обработки данных, которые будут использоваться в текущей лабораторной работе;
	\item реализовать программу, выполняющую конвейерную обработку с количеством лент не менее трех  и программу обрабатывающую заявки последовательно;
	\item провести сравнительный анализ времени работы реализаций.
\end{enumerate}



