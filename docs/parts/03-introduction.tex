\chapter*{\hfill{\centering  ВВЕДЕНИЕ}\hfill}
\addcontentsline{toc}{chapter}{ВВЕДЕНИЕ}

Многопоточность представляет собой способность процессора выполнять несколько задач или потоков одновременно, что обеспечивается операционной системой. Этот подход отличается от многопроцессорности, где каждый процессор выполняет отдельную задачу, поскольку в случае многопоточности ресурсы одного или нескольких ядер используются несколькими потоками или процессами \cite{muti-thread}.

При последовательной реализации алгоритма только одно ядро процессора используется для выполнения программы. Однако при использовании параллельных вычислений --- многопоточности --- разные ядра могут одновременно решать независимые вычислительные задачи, что приводит к ускорению общего решения задачи.
В случае многопроцессорных систем, включающих несколько физических процессоров, подход многопоточности направлен на максимальное использование ресурсов каждого ядра, позволяя осуществлять параллельную обработку на уровне потоков и инструкций. Иногда эти подходы объединяются в системах с несколькими многопоточными процессорами или ядрами для повышения общей производительности \cite{muti-thread}.


Целью данной лабораторной работы является получение навыков организации параллельного выполнения операций.

Для достижения поставленной цели необходимо выполнить следующие задачи.
\label{sec:targets}
\begin{enumerate}
	\item Описать алгоритм сортировки слиянием.
	\item Разработать версии  приведенного алгоритма, при использовании 1 потока и нескольких потоков.
	\item Определить средства программной реализации.
	\item Реализовать разработанные алгоритмы.
	\item Выполнить замеры процессорного времени работы различных реализаций алгоритма.
	\item Провести анализ времени получения отсортированных данных.
\end{enumerate}