\chapter*{Введение}
\addcontentsline{toc}{chapter}{Введение}

В последние два десятилетия при оптимизации сложных систем исследователи все чаще применяют природные механизмы поиска наилучших решений. 
Один из таких механизмов --- это муравьиные алгоритмы, представляющие собой новый перспективный метод оптимизации, базирующийся на моделировании поведения колонии муравьев~\cite{shtovba}. 

Первый вариант муравьиного алгоритма был предназначен для приближенного решения задачи коммивояжера~\cite{ershov}. 

Целью данной работы является исследование методов решения задачи коммивояжера двумя способами: полным перебором и с помощью муравьиного алгоритма.

Для поставленной цели необходимо выполнить следующие задачи.
\begin{enumerate}
	\item Описать задачу коммивояжера.
	\item Описать методы решения задачи коммивояжера --- метод полного перебора и метод на основе муравьиного алгоритма.
	\item Привести схемы муравьиного алгоритма и алгоритма, позволяющего решить задачу коммивояжера методом полного перебора.
	\item Разработать и реализовать программный продукт, позволяющий решить задачу коммивояжера исследуемыми методами.
	\item Сравнить по времени метод полного перебора и метод на основе муравьиного алгоритма.
	\item Описать и обосновать полученные результаты в отчете о выполненной лабораторной работе.
\end{enumerate}

Выданный индивидуальный вариант для выполнения лабораторной работы:
\begin{itemize}
	\item неориентированный граф;
	\item без элитных муравьев;
	\item незамкнутый маршрут;
	\item карта перемещения по Африке.
\end{itemize}