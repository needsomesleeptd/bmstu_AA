\chapter*{\hfill{\centering  ВВЕДЕНИЕ}\hfill}

\addcontentsline{toc}{chapter}{ВВЕДЕНИЕ}

Упорядочение данных чрезвычайно важно в программировании. Практически сортировка и поиск в той или иной
мере присутствуют во всех приложениях; в частности, при обработке больших объемов данных эффективность именно этих операций определяет эффективность, а иногда и работоспособность всей системы.
Под сортировкой понимается упорядочивание элементов последовательности по какому-либо признаку \cite{book_shagbazyan}.
Можно сказать, что достаточно четкие представления об этой области нужны при решении любой задачи на ЭВМ как
обязательные элементы искусства программирования \cite{book_knut}.


Целью данной лабораторной работы является описание и исследование трудоемкости алгоритмов сортировки.

\label{sec:targets}
Для поставленной цели необходимо выполнить следующие задачи.
\begin{enumerate}
	\item Описать алгоритмы сортировки;
	\item создать программное обеспечение, реализующее следующие алгоритмы сортировки;
	\begin{itemize}
		\item перемешиванием;
		\item блочная;
		\item поразрядная;
	\end{itemize}
	\item оценить трудоемкости сортировок;
	\item замерить время реализации;
	\item провести анализ затрат работы программы по времени, выяснить влияющие на них характеристики;
\end{enumerate}




