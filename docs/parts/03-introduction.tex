\chapter*{\hfill{\centering  ВВЕДЕНИЕ}\hfill}

\addcontentsline{toc}{chapter}{ВВЕДЕНИЕ}

При работе со словами, необходимо их сравнивать, причем необходима конкретная метрика, которая покажет, насколько посимвольно одно слово отличается от другого.
Одной из таких метрик является Расстояние Левенштейна.
Данное расстояние является метрикой, измеряющей по модулю разность между двумя последовательностями символов \cite{levenshtein}. 

Впервые задачу поставил в 1965 году советский математик Владимир Левенштейн при изучении последовательностей 0 -- 1, впоследствии более общую задачу для произвольного алфавита связали с его именем.

Расстояние Левенштейна активно используется и по сей день:
\begin{itemize}[label=---]
	\item исправление ошибок в слове(в поисковых системах, базах данных, при вводе текста, при автоматическом распознавании отсканированного текста или речи);
	\item сравнение текстовых файлов утилитой \texttt{diff};
	\item для сравнения геномов, хромосом и белков в биоинформатике.
\end{itemize}

Данная метрика была модифицирована Фредриком Дамерау, путем введения операции перестановки соседних символов \cite{Damerau}.

\label{sec:targets}
Целью данной лабораторной работы является описание и исследование алгоритмов поиска расстояний Левенштейна и Дамерау---Левенштейна.


Для поставленной цели необходимо выполнить следующие задачи.
\begin{enumerate}
	\item Исследовать расстояние Левенштейна;
	\item разработать алгоритмы поиска расстояний Левенштейна, Дамерау---Левенштейна;
	\item создать программное обеспечение, реализующее следующие алгоритмы;
	\item провести исследование, затрачиваемого процессорного времени и памяти при различных реализациях алгоритмов;
	\item провести сравнительный анализ алгоритмов.
\end{enumerate}


