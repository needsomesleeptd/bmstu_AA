\chapter*{\hfill{\centering  ВВЕДЕНИЕ}\hfill}

\addcontentsline{toc}{chapter}{ВВЕДЕНИЕ}

Разработчики архитектуры компьютеров издавна прибегали к методам проектирования, известным под общим названием <<совмещение операций>>, при котором аппаратура компьютера в любой момент времени выполняет одновременно более одной базовой операции.

Этот метод включает в себя, в частности, такое понятие, как
конвейеризация. Конвейры широко применяются программистами для решения трудоемких задач, которые можно разделить на этапы, а также в
большинстве современных быстродействующих процессоров~\cite{conveyor}.


Целью данной работы является получение навыков организации конвейерной обработки данных.

В рамках выполнения работы необходимо решить следующие задачи: 
\begin{enumerate}
	\item описать организацию конвейерной обработки данных;
	\item описать алгоритмы обработки данных, которые будут использоваться в текущей лабораторной работе;
	\item реализовать программу, выполняющую конвейерную обработку с количеством лент не менее трех в однопоточной и многопоточной реализаций;
	\item провести сравнительный анализ времени работы реализаций.
\end{enumerate}



