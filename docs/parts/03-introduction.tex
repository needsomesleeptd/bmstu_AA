\chapter*{\hfill{\centering  ВВЕДЕНИЕ}\hfill}

\addcontentsline{toc}{chapter}{ВВЕДЕНИЕ}

Во многих областях человеческой деятельности информацию часто представляют в форме матриц.
Матрица --- это регулярный числовой массив. В некоторых задачах необходимо уменьшить 
число хранимых данных, не потеряв важную информацию из матрицы.
При анализ данных для снижения размерности хранимых данных и понижения <<шума>> возможно использование сингулярного разложения матриц (SVD, от англ. Singular Value Decomposition)~\cite{SVD}.


Целью данной лабораторной работы является описание алгоритма сингулярного разложения и исследование его трудоемкости.

\label{sec:targets}
Для поставленной цели необходимо выполнить следующие задачи.
\begin{enumerate}
	\item Описать алгоритм сингулярного разложения.
	\item Разработать алгоритм сингулярного разложения.
	\item Создать программное обеспечение, реализующее алгоритм сингулярного разложения.
	\item Оценить трудоемкость реализации сингулярного разложения.
\end{enumerate}




