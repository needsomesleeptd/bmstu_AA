\chapter{Исследовательская часть}

В данном разделе будут приведены примеры работы программ, постановка эксперимента и сравнительный анализ алгоритмов на основе полученных данных.


\section{Технические характеристики}

Технические характеристики устройства, на котором выполнялись замеры времени, представлены далее.

\begin{enumerate}
	\item Процессор	Intel(R) Core(TM) i7-9750H CPU @ 2.60GHz, 2592 МГц, ядер: 6, логических процессоров: 12.
	\item Оперативная память: 16 ГБайт.
	\item Операционная система: Майкрософт Windows 10 Pro \cite{windows}.
	\item Использованная подсистема: WSL2 \cite{WSL2}.
\end{enumerate}

При замерах времени ноутбук был включен в сеть электропитания и был нагружен только системными приложениями.



\section{Демонстрация работы программы}

На рисунке \ref{img:work_ex} представлена демонстрация работы разработанного приложения для алгоритмов сортировок.
\includeimage
{work_ex} % Имя файла без расширения (файл должен быть расположен в директории inc/img/)
{f} % Обтекание (без обтекания)
{H} % Положение рисунка (см. figure из пакета float)
{1\textwidth} % Ширина рисунка
{Пример работы программы} % Подпись рисунка



\section{Временные характеристики}

Результаты исследования замеров по времени приведены в таблице \ref{t:timings}.

\begin{table}[ht]
	\centering
	\caption{Полученная таблица замеров по времени различных реализаций алгоритмов сортировки}
	\begin{tabular}{|c|c|c|c|}
		\hline
		n     & Поразрядная(мс) & Блочная(мс) & Перемешиванием(мс) \\ \hline
		1000  & 0.15692         & 1.2194      & 2.03               \\ \hline
		2000  & 0.31811         & 4.5549      & 8.2541             \\ \hline
		3000  & 0.47956         & 10.05       & 18.767             \\ \hline
		4000  & 0.62436         & 17.109      & 32.384             \\ \hline
		5000  & 0.78017         & 26.401      & 50.514             \\ \hline
		6000  & 0.93926         & 37.91       & 72.854             \\ \hline
		7000  & 1.0944          & 51.295      & 99.256             \\ \hline
		8000  & 1.2494          & 66.616      & 129.05             \\ \hline
		9000  & 1.6587          & 97.898      & 191.14             \\ \hline
		10000 & 1.5698          & 104.19      & 203.08             \\ \hline
	\end{tabular}
	\label{t:timings}
\end{table}

Для таблицы \ref{t:timings} расчеты проводились с шагом 1000, сортировки производились 1000 раз, после чего результат усредняется. В качестве сортируемых значений использовались  числа от 1 до 10000000. Размерность блоков определялась как $\frac{n}{2} + 2$. Данные генерировались из равномерного распределения.
По таблице \ref{t:timings} были построены графики \ref{img:all-sorts-cmp} -- \ref{img:slow-sorts-cmp}.

\includeimage
{all-sorts-cmp} % Имя файла без расширения (файл должен быть расположен в директории inc/img/)
{f} % Обтекание (без обтекания)
{H} % Положение рисунка (см. figure из пакета float)
{1\textwidth} % Ширина рисунка
{Сравнение реализаций сортировок по времени с использованием логарифмической шкалы} % Подпись рисунка


\includeimage
{slow-sorts-cmp} % Имя файла без расширения (файл должен быть расположен в директории inc/img/)
{f} % Обтекание (без обтекания)
{H} % Положение рисунка (см. figure из пакета float)
{1\textwidth} % Ширина рисунка
{Сравнение реализаций блочной сортировки и сортировки перемешиванием} % Подпись рисунка

\section*{Вывод}
В результате анализа таблицы \ref{t:timings}, было получено, что при 10000 элементов реализация алгоритма  блочной сортировки требует в 1.95 раз больше  времени, чем сортировка перемешиванием, поразрядная сортировка при 10000 элементах требует в 65 раз меньше времени, чем блочная сортировка . Ввиду того, что сортируемые числа неотрицательные и имеют малое число разрядов поразрядная сортировка оказалась самой эффективной по временным затратам, блочная сортировка показала лучший результат по сравнению с сортировкой перемешиваем благодаря равномерному распределению данных.


