\chapter{Исследовательская часть}

В данном разделе будут приведены примеры работы программ, проведены замеры и сделан сравнительный анализ алгоритмов на основе полученных данных.


\section{Технические характеристики}

Технические характеристики устройства, на котором выполнялись замеры времени, представлены далее.

\begin{enumerate}
	\item Процессор	Intel(R) Core(TM) i7-9750H CPU 2592 МГц, ядер: 6, логических процессоров: 12.
	\item Оперативная память: 16 ГБ.
	\item Операционная система: Майкрософт Windows 10 Pro~\cite{windows}.
	\item Использованная подсистема: WSL2~\cite{WSL2}.
\end{enumerate}

При замерах времени ноутбук был включен в сеть электропитания и был нагружен только системными приложениями.



\section{Демонстрация работы программы}

На рисунке \ref{img:work_ex} представлена демонстрация работы разработанного приложения для алгоритмов сортировок.
\includeimage
{work_ex} % Имя файла без расширения (файл должен быть расположен в директории inc/img/)
{f} % Обтекание (без обтекания)
{H} % Положение рисунка (см. figure из пакета float)
{1\textwidth} % Ширина рисунка
{Пример работы программы} % Подпись рисунка



\section{Временные характеристики}

Результаты замеров времени получения результатов от числа заявок, приведен в таблице~\ref{t:timings_req}, размер сортируемых массивов был равен 9000.В качестве сортируемых значений использовались  числа от 1 до 10000000. Данные генерировались из равномерного распределения, время замеров приведено в секундах. По таблице~\ref{t:timings_req}, был получен график~\ref{img:req_cmp}.

\begin{table}[H]
	\centering
	\caption{Замеры по времени получения результата реализаций различных способов обработки заявок от числа заявок}
	\begin{tabular}{|c|c|c|}
		\hline
		Заявки & Последовательная обработка (с) & Конвейерная обработка (c) \\ \hline
		10     & 0.138498                       & 0.132937              \\ \hline
		20     & 0.276232                       & 0.270415              \\ \hline
		30     & 0.419785                       & 0.375741              \\ \hline
		40     & 0.560126                       & 0.53251               \\ \hline
		50     & 0.706062                       & 0.620642              \\ \hline
		60     & 0.836043                       & 0.734683              \\ \hline
		70     & 0.957982                       & 0.866383              \\ \hline
		80     & 1.10419                        & 0.988661              \\ \hline
		90     & 1.23092                        & 1.11383               \\ \hline
		100    & 1.39349                        & 1.27446               \\ \hline
		110    & 1.52715                        & 1.41736               \\ \hline
		120    & 2.09141                        & 1.93223               \\ \hline
	\end{tabular}
	\label{t:timings_req}
\end{table}

\includeimage
{req_cmp} % Имя файла без расширения (файл должен быть расположен в директории inc/img/)
{f} % Обтекание (без обтекания)
{H} % Положение рисунка (см. figure из пакета float)
{1\textwidth} % Ширина рисунка
{Сравнение реализаций обработок заявок по времени получения результата в зависимости от числа заявок} % Подпись рисунка

Результаты замеров времени получения результатов от числа сортируемых элементов, приведен в таблице~\ref{t:timings_n}, число заявок было равно 100, столбец $n$ в данном случае описывает число сортируемых элементов. качестве сортируемых значений использовались  числа от 1 до 10000000, время замеров приведено в секундах. Данные генерировались из равномерного распределения. По таблице~\ref{t:timings_n}, был получен график~\ref{img:n_cmp}.

\begin{table}[H]
	\centering
	\caption{Замеры по времени получения результата реализаций различных способов обработки заявок от числа элементов в массиве}
	\begin{tabular}{|c|c|c|}
	\hline
	n     & Последовательная обработка (с) & Конвейерная обработка (с) \\ \hline
	1000  & 0.167468                       & 0.108578                  \\ \hline
	2000  & 0.291374                       & 0.224196                  \\ \hline
	3000  & 0.440222                       & 0.362134                  \\ \hline
	4000  & 0.557684                       & 0.453164                  \\ \hline
	5000  & 0.769051                       & 0.63144                   \\ \hline
	6000  & 0.884096                       & 0.750595                  \\ \hline
	7000  & 0.987792                       & 0.851104                  \\ \hline
	8000  & 1.16747                        & 1.10913                   \\ \hline
	9000  & 1.33448                        & 1.30057                   \\ \hline
	10000 & 1.62122                        & 1.49942                   \\ \hline
	11000 & 1.65712                        & 1.57924                   \\ \hline
	12000 & 1.80122                        & 1.73396                   \\ \hline
	13000 & 1.97171                        & 1.85373                   \\ \hline
	14000 & 2.08562                        & 2.01101                   \\ \hline
	15000 & 2.12513                        & 1.96112                   \\ \hline
	\end{tabular}
	\label{t:timings_n}
\end{table}

\includeimage
{n_cmp} % Имя файла без расширения (файл должен быть расположен в директории inc/img/)
{f} % Обтекание (без обтекания)
{H} % Положение рисунка (см. figure из пакета float)
{1\textwidth} % Ширина рисунка
{Сравнение реализаций обработок заявок по времени получения результата в зависимости от числа элементов в массиве} % Подпись рисунка

\section*{Вывод}
В результате анализа таблицы~\ref{t:timings_req}, можно сделать вывод что конвейерная обработка заявок позволяет быстрее получать результат: при 110 заявках, при использовании реализации с конвейером данные были получены за 1.41 сек, что в 1.08 раз быстрее, чем при использовании последовательной обработки.

При увелечении числа сортируемых элементов, реализация конвейерной обработки быстрее получает результат: при 15000 сортируемых элементах результат был получен за 1.96 секунд, что в 1.09 раз быстрее чем последовательная обработка.
