\chapter{Исследовательская часть}

\section{Технические характеристики}

Технические характеристики устройства, на котором выполнялись замеры по времени, представлены далее.

\begin{enumerate}
	\item Процессор	Intel(R) Core(TM) i7-9750H CPU @ 2.60GHz, 2592 МГц, ядер: 6, логических процессоров: 12.
	\item Оперативная память: 16 ГБайт.
	\item Операционная система: Майкрософт Windows 10 Pro.
\end{enumerate}

При замерах времени ноутбук был включен в сеть электропитания и был нагружен только системными приложениями.

\section{Демонстрация работы программы}

На рисунке \ref{img:demonstration} представлена демонстрация работы разработанного программного обеспечения, 
показаны результаты вычислений расстояний Дамерау--Левенштейна и Левенштейна между словами <<fds>>,<<asd>>, заметим
что в случае поднятого флага вывода матриц происходит вывод матриц, использованных в расчетах.
\clearpage
\begin{figure}[H]
	\centering
	\includegraphics[height=0.7\textheight]{../img/programm_work.png}
	\caption{Демонстрация работы программы}
	\label{img:demonstration}
\end{figure}

\clearpage

\section{Временные характеристики}

Результаты эксперимента замеров по времени приведены в таблице \ref{t:timings}, каждый столбец таблице имеет заголовок в формате:
<<Алгоритм расчета-соответствующее расстояние>>, под <<Дамерау>> имеется в виду расстояние Дамерау--Левенштейна, под словом
<<Мемоизация>> имеется в виду  рекурсия с мемоизацией, описание которой приведено в разделе \ref{subec:memorysation_descr}.

Заметим, что некоторые поля в данной таблице
имеют значение <<->>, это обусловлено тем, что дальнейший расчет значений столбца <<Рекурсия-Дамерау>> окажется слишком
долгим, полученных данных достаточно для проведения исследования.

Замеры проводились на одинаковых длин строк от 1 до 100 с шагом 1, для получения достоверных результатов замеры 
времени для каждой пары строк проводились 100 раз, после чего усреднялись. Все результаты вычислений приведены в миллисекундах.
\begin{table}[!ht]
    \centering
	\label{t:timings}
	\tiny
	\caption{Полученная таблица временных замеров различных реализаций поиска редакционных расстояний}
    \begin{tabular}{|l|l|l|l|l|}
    \hline
        Длина слова & Матрица-Дамерау & Рекурсия-Дамерау & Мемоизация-Дамерау & Матрица-Левенштейн \\ \hline
        1 & 0.00134 & 0.00113 & 0.00126 & 0.00688 \\ \hline
        2 & 0.0014 & 0.00133 & 0.00156 & 0.00135 \\ \hline
        3 & 0.00172 & 0.0021 & 0.00194 & 0.00158 \\ \hline
        4 & 0.00214 & 0.00583 & 0.00408 & 0.00206 \\ \hline
        5 & 0.00262 & 0.02679 & 0.00322 & 0.00249 \\ \hline
        6 & 0.00466 & 0.13082 & 0.00419 & 0.00306 \\ \hline
        7 & 0.0028 & 0.48749 & 0.00346 & 0.00352 \\ \hline
        8 & 0.00301 & 2.3975 & 0.00494 & 0.00286 \\ \hline
        9 & 0.00457 & 13.881 & 0.00517 & 0.00425 \\ \hline
        10 & 0.0046 & 74.181 & 0.00657 & 0.00422 \\ \hline
        11 & 0.00523 & 416.8 & 0.0074 & 0.00476 \\ \hline
        12 & 0.0063 & 2423.0 & 0.00881 & 0.00569 \\ \hline
        13 & 0.00632 & - & 0.00876 & 0.00575 \\ \hline
        14 & 0.00854 & - & 0.00999 & 0.00658 \\ \hline
        15 & 0.00824 & - & 0.0115 & 0.0075 \\ \hline
        16 & 0.00909 & - & 0.01305 & 0.00923 \\ \hline
        17 & 0.01041 & - & 0.01559 & 0.00956 \\ \hline
        18 & 0.01202 & - & 0.016 & 0.01014 \\ \hline
        19 & 0.0126 & - & 0.01818 & 0.01148 \\ \hline
        20 & 0.01339 & - & 0.01958 & 0.0122 \\ \hline
        21 & 0.01464 & - & 0.02243 & 0.01338 \\ \hline
        22 & 0.01753 & - & 0.02401 & 0.01522 \\ \hline
        23 & 0.01769 & - & 0.02708 & 0.016 \\ \hline
        24 & 0.01889 & - & 0.02867 & 0.01709 \\ \hline
        25 & 0.02133 & - & 0.03004 & 0.01849 \\ \hline
        26 & 0.02241 & - & 0.03286 & 0.02148 \\ \hline
        27 & 0.02427 & - & 0.03635 & 0.02216 \\ \hline
        28 & 0.02539 & - & 0.03736 & 0.02485 \\ \hline
        29 & 0.0272 & - & 0.04016 & 0.02652 \\ \hline
        30 & 0.02931 & - & 0.04404 & 0.02675 \\ \hline
        31 & 0.03108 & - & 0.04581 & 0.02993 \\ \hline
        32 & 0.03355 & - & 0.05012 & 0.03049 \\ \hline
        33 & 0.03501 & - & 0.05191 & 0.03281 \\ \hline
        34 & 0.0407 & - & 0.05685 & 0.03382 \\ \hline
        35 & 0.03886 & - & 0.05863 & 0.03544 \\ \hline
        36 & 0.04309 & - & 0.06307 & 0.03907 \\ \hline
        37 & 0.04323 & - & 0.06525 & 0.03922 \\ \hline
        38 & 0.04734 & - & 0.07587 & 0.04526 \\ \hline
        39 & 0.05079 & - & 0.07475 & 0.04505 \\ \hline
        40 & 0.0554 & - & 0.08139 & 0.04885 \\ \hline
        41 & 0.05543 & - & 0.08094 & 0.05126 \\ \hline
        42 & 0.05758 & - & 0.09296 & 0.05472 \\ \hline
        43 & 0.05906 & - & 0.09024 & 0.05349 \\ \hline
        44 & 0.06341 & - & 0.09139 & 0.05691 \\ \hline
        45 & 0.06622 & - & 0.09554 & 0.05951 \\ \hline
        46 & 0.068 & - & 0.10273 & 0.06117 \\ \hline
        47 & 0.0712 & - & 0.10483 & 0.06496 \\ \hline
        48 & 0.07377 & - & 0.11548 & 0.06771 \\ \hline
        49 & 0.07676 & - & 0.11508 & 0.06909 \\ \hline
        50 & 0.0806 & - & 0.11882 & 0.07561 \\ \hline
        51 & 0.08402 & - & 0.12272 & 0.0767 \\ \hline
        52 & 0.08809 & - & 0.12774 & 0.07858 \\ \hline
        53 & 0.09262 & - & 0.13187 & 0.08221 \\ \hline
        54 & 0.09183 & - & 0.13909 & 0.08462 \\ \hline
        55 & 0.09647 & - & 0.14234 & 0.08714 \\ \hline
        56 & 0.1006 & - & 0.14796 & 0.09048 \\ \hline
        57 & 0.1028 & - & 0.15549 & 0.09485 \\ \hline
        58 & 0.10585 & - & 0.16104 & 0.09738 \\ \hline
        59 & 0.10826 & - & 0.16298 & 0.10099 \\ \hline
        60 & 0.11546 & - & 0.16882 & 0.10566 \\ \hline
        61 & 0.12044 & - & 0.17882 & 0.10939 \\ \hline
        62 & 0.12153 & - & 0.1808 & 0.11003 \\ \hline
        63 & 0.12612 & - & 0.18707 & 0.11303 \\ \hline
        64 & 0.129 & - & 0.19677 & 0.11822 \\ \hline
        65 & 0.1343 & - & 0.19808 & 0.13005 \\ \hline
        66 & 0.14127 & - & 0.21004 & 0.1253 \\ \hline
        67 & 0.14692 & - & 0.20936 & 0.13454 \\ \hline
        68 & 0.1679 & - & 0.22829 & 0.14849 \\ \hline
        69 & 0.15841 & - & 0.22867 & 0.14192 \\ \hline
        70 & 0.15921 & - & 0.24836 & 0.14623 \\ \hline
        71 & 0.15988 & - & 0.2375 & 0.14523 \\ \hline
        72 & 0.16827 & - & 0.24643 & 0.15342 \\ \hline
        73 & 0.17037 & - & 0.25365 & 0.156 \\ \hline
        74 & 0.17445 & - & 0.26027 & 0.15706 \\ \hline
        75 & 0.1839 & - & 0.26633 & 0.16341 \\ \hline
        76 & 0.18451 & - & 0.2745 & 0.16571 \\ \hline
        77 & 0.19255 & - & 0.28234 & 0.17876 \\ \hline
        78 & 0.19624 & - & 0.28753 & 0.17891 \\ \hline
        79 & 0.20694 & - & 0.29764 & 0.18215 \\ \hline
        80 & 0.20866 & - & 0.30539 & 0.18861 \\ \hline
        81 & 0.20868 & - & 0.31365 & 0.18863 \\ \hline
        82 & 0.21407 & - & 0.32003 & 0.19693 \\ \hline
        83 & 0.22075 & - & 0.32734 & 0.20323 \\ \hline
        84 & 0.25258 & - & 0.34314 & 0.20608 \\ \hline
        85 & 0.23223 & - & 0.34761 & 0.20881 \\ \hline
        86 & 0.23893 & - & 0.36106 & 0.21708 \\ \hline
        87 & 0.25343 & - & 0.37841 & 0.23564 \\ \hline
        88 & 0.26672 & - & 0.3987 & 0.23541 \\ \hline
        89 & 0.2565 & - & 0.37914 & 0.22846 \\ \hline
        90 & 0.25403 & - & 0.39294 & 0.25183 \\ \hline
        91 & 0.26909 & - & 0.39794 & 0.24198 \\ \hline
        92 & 0.27941 & - & 0.39998 & 0.24435 \\ \hline
        93 & 0.28903 & - & 0.42276 & 0.25272 \\ \hline
        94 & 0.28921 & - & 0.43863 & 0.26984 \\ \hline
        95 & 0.31258 & - & 0.45226 & 0.28118 \\ \hline
        96 & 0.31653 & - & 0.46734 & 0.28617 \\ \hline
        97 & 0.32134 & - & 0.47819 & 0.28236 \\ \hline
        98 & 0.30153 & - & 0.45308 & 0.27619 \\ \hline
        99 & 0.31551 & - & 0.47476 & 0.30401 \\ \hline
        100 & 0.46852 & - & 0.61734 & 0.38562 \\ \hline
    \end{tabular}
\end{table}

Имеет смысл сравнить поиск расстояний Левенштейна и Дамерау--Левенштейна при использовании матрицы. Заметим, что при использовании
алгоритма поиска расстояния Дамерау--Левенштейна, при рассмотрении слов длины больше 80 затрачивается больше времени, чем 
при использоавнии алгоритма Левенштейна. Данная закономерность обусловлена дополнительной операцией обмена символов, которую 
необходимо учитывать при поиске расстояния Дамерау-Левенштейна. Графики \ref{plt:time_matrix_cmp,plt:time_mat_rec_cmp,plt:time_rec_cmp} получены при 
помощи анализа данных таблицы \ref{t:timings}.

\begin{figure}[h]
	\centering
	\includesvg[height=0.3\textheight]{../img/matrixCmp.svg}
	\caption{Сравнение по времени нерекурсивных реализаций алгоритмов поиска расстояний Левенштейна и Дамерау-Левенштейна}
	\label{plt:time_matrix_cmp}
\end{figure}

Также необходимо сравнить реализации рекурсивного поиска расстояния Дамерау-Левенштейна с мемоизацией и без, полученный график приведен
на картинке \ref{plt:time_rec_cmp}. Заметим, что рекурсия с мемоизацией, засчет сохранения информации о уже расчитанных расстояниях подстрок
показала себя заметно эффективнее, чем рекурсия без мемоизации. 


\begin{figure}[h]
	\centering
	\includesvg[height=0.3\textheight]{../img/recCmp.svg}
	\caption{Сравнение по времени реализации рекурсивного поиска расстояния Дамерау-Левенштейна с мемоизацией и без}
	\label{plt:time_rec_cmp}
\end{figure}

Стоит отметить, что при сравнении реализации рекурсивного поиска расстояния Дамерау-Левенштейна с  мемоизацией и поиска
с использованием матрицы (см. \ref{plt:time_mat_rec_cmp}), реализация с использованием матрицы показала себя более эффективной
по времени. 

\begin{figure}[h]
	\centering
	\includesvg[height=0.3\textheight]{../img/matRecCmp.svg}
	\caption{Сравнение по времени реализации рекурсивного поиска расстояния Дамерау-Левенштейна с  мемоизацией, с поиском
	расстояния с использоавнием матрицы}
	\label{plt:time_mat_rec_cmp}
\end{figure}


\section{Характеристики по памяти}

Введем следующие обозначения:
\begin{itemize}
	\item$n$ --- длина строки $S_{1}$;
	\item$m$ --- длина строки $S_{2}$;
	\item$size()$ --- функция вычисляющая размер в байтах;
	\item $string$ --- строковый тип;
	\item $int$ --- целочисленный тип;
	\item $size\_t$ --- беззнаковый целочисленный тип.
\end{itemize}

Максимальная глубина стека вызовов при рекурсивной реализации нахождения расстояния Дамерау-Левенштейна равна сумме входящих строк, а на каждый вызов требуется 2 дополнительные переменные, соответственно, максимальный расход памяти равен:
\begin{equation}
	\label{eq:dl_rec_memory}
	(n + m) \cdot (2 \cdot size(string) + 3 \cdot size(int) + 2 \cdot sizeof(size\_t)),
\end{equation}
где:
\begin{itemize}
	\item $2 \cdot size(string)$ --- хранение двух строк;
	\item $2 \cdot size(size\_t)$ --- хранение размеров строк;
	\item $2 \cdot size(int)$ --- дополнительные переменные;
	\item $size(int)$ --- адрес возврата.
\end{itemize}

Для рекурсивного алгоритма c кешированием поиска расстояния Дамерау-Левенштейна будет теоретически схож с расчетом в формуле (\ref{eq:dl_rec_memory}), но также учитывается матрица, соответственно, максимальный расход памяти равен:
\begin{equation}
	\label{eq:dl_hash_memory}
	\begin{aligned}
		(n + m) \cdot (2 \cdot size(string) + 3 \cdot size(int) + 2 \cdot size(size\_t)) + \\
		+ (n + 1) \cdot (m + 1) \cdot size(int)
	\end{aligned}
\end{equation}
Использование памяти при итеративной реализации алгоритма поиска расстояния Левенштейна теоретически равно:
\begin{equation}
	\label{eq:lev_mtr_memory}
	\begin{aligned}
		(n + 1) \cdot (m + 1) \cdot size(int) + 2 \cdot size(string) + 2 \cdot size(size\_t) + \\
		+ size(int **) + (n + 1) \cdot size(int *) + 2 \cdot size(int),
	\end{aligned}
\end{equation}
где 
\begin{itemize}
	\item $2 \cdot size(string)$ --- хранение двух строк;
	\item $2 \cdot size(size\_t)$ --- хранение размеров матрицы;
	\item $(n + 1) \cdot (m + 1) \cdot size(int)$ --- хранение матрицы;
	\item $size(int **) + (n + 1) \cdot size(int *)$ --- указатель на матрицу;
	\item $size(int)$ --- дополнительная переменная для хранения результата;
	\item $size(int)$ --- адрес возврата.
\end{itemize}

Использование памяти при итеративной реализации алгоритма поиска расстояния Дамерау-Левенштейна теоретически равно:
\begin{equation}
	\label{eq:dl_mtr_memory}
	\begin{aligned}
		(n + 1) \cdot (m + 1) \cdot size(int) + 2 \cdot size(string) + 2 \cdot size(size\_t) + \\
		+ size(int **) + (n + 1) \cdot size(int *) + 3 \cdot size(int),
	\end{aligned}
\end{equation}
где 
\begin{itemize}
	\item $2 * size(string)$ --- хранение двух строк;
	\item $2 \cdot size(size\_t)$ --- хранение размеров матрицы;
	\item $(n + 1) \cdot (m + 1) \cdot size(int)$ --- хранение матрицы;
	\item $size(int **) + (n + 1) \cdot size(int *)$ --- указатель на матрицу;
	\item $2 \cdot size(int)$ --- дополнительные переменные;
	\item $size(int)$ --- адрес возврата.
\end{itemize}

По расходу памяти итеративные алгоритмы проигрывают рекурсивным: максимальный размер используемой памяти в итеративном растет как произведение длин строк, в то время как у рекурсивного алгоритма — как сумма длин строк.

По формулам \ref{eq:dl_rec_memory} -- \ref{eq:lev_mtr_memory} затрат по памяти в программе были написаны соответствующие функции для подсчета расходуемой памяти, результаты расчетов, которых представлены в таблице \ref{tbl:memory}, где размеры строк находятся в диапазоне от 10 до 200 с шагом 10.

\clearpage

\begin{table}[ht]
	\small
	\begin{center}
		\begin{threeparttable}
		\caption{Замер памяти для строк, размером от 10 до 200}
		\label{tbl:memory}
		\begin{tabular}{|c|c|c|c|c|}
			\hline
			& \multicolumn{4}{c|}{\bfseries Размер в байтах} \\ \cline{2-5}
			& \multicolumn{1}{c|}{\bfseries Левенштейн}
			& \multicolumn{3}{c|}{\bfseries Дамерау-Левенштейн} \\ \cline{2-5}
			\bfseries Длина (символ) & \bfseries Итеративный & \bfseries Итеративный & \multicolumn{2}{c|}{\bfseries Рекурсивный} \\ \cline{4-5}
			& & & \bfseries Без кеша & \bfseries С кешом
			\csvreader{csv/memory.csv}{}
			{\\\hline \csvcoli & \csvcolii & \csvcoliii & \csvcoliv & \csvcolv} \\
			\hline
		\end{tabular}	
		\end{threeparttable}
	\end{center}
\end{table}

Из данных, приведенных в таблице \ref{tbl:memory}, понятно, что рекурсивные алгоритмы являются более эффективными по памяти, так как используется только память под локальные переменные, передаваемые аргументы и возвращаемое значение, в то время как итеративные алгоритмы затрачивают память линейно пропорционально длинам обрабатываемых строк.

\begin{figure}[h]
	\centering
	\includegraphics[height=0.3\textheight]{img/diag_03.png}
	\caption{Сравнение по памяти алгоритмов поиска расстояния Левенштейна и Дамерау-Левенштейна --- итеративной и рекурсивной реализации}
	\label{plt:memory}
\end{figure}

Из рисунка \ref{plt:memory_1} понятно, что рекурсивная реализация алгоритма поиска расстояния Дамерау-Левенштейна эффективная по памяти, чем итеративная.

\begin{figure}[h]
	\centering
	\includegraphics[height=0.3\textheight]{img/diag_04.png}
	\caption{Сравнение по памяти алгоритмов поиска расстояния Дамерау-Левенштейна --- итеративной и рекурсивной реализации}
	\label{plt:memory_1}
\end{figure}

\clearpage

\section{Вывод}

В данном разделе было произведено сравнение количества затраченного времени и памяти алгоритмов поиска расстояний Левенштейна и Дамерау-Левенштейна. Наименее затратным по времени оказался итеративный алгоритм нахождения расстояния Левенштейна.

Приведенные характеристики показывают, что рекурсивная реализация алгоритма в 21 раз проигрывает по времени. В связи с этим, рекурсивные алгоритмы следует использовать лишь для малых размерностей строк (1 -- 4 символа).

Так как во время печати очень часто возникают ошибки связанные с транспозицией букв \cite{ulianov}, алгоритм поиска расстояния Дамерау-Левенштейна является наиболее предпочтительным, не смотря на то, что он проигрывает по времени и памяти алгоритму Левенштейна.

Рекурсивная реализация алгоритма поиска расстояния Дамерау-Левенштейна будет более затратным по времени по сравнению с итеративной реализацией алгоритма поиска расстояния Дамерау-Левенштейна, но менее затратным по памяти по отношению к итеративному алгоритму Дамерау-Левенштейна.
