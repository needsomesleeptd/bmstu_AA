\chapter{Исследовательская часть}

В данном разделе будут приведены пример работы программы, условия исследования   и сравнительный анализ алгоритмов на основе полученных данных.

\section{Технические характеристики}

Технические характеристики устройства, на котором выполнялись замеры времени, представлены далее.

\begin{enumerate}
	\item Процессор	Intel(R) Core(TM) i7-9750H CPU @ 2.60GHz, 2592 МГц, ядер: 6, логических процессоров: 12.
	\item Оперативная память: 16 ГБайт.
	\item Операционная система: Майкрософт Windows 10 Pro \cite{windows}.
	\item Использованная подсистема: WSL2 \cite{WSL2}.
\end{enumerate}

При замерах времени ноутбук был включен в сеть электропитания и был нагружен только системными приложениями.


\section{Демонстрация работы программы}

На рисунке \ref{img:work_ex} представлена демонстрация работы разработанного приложения для алгоритмов сортировок.
\includeimage
{work_ex} % Имя файла без расширения (файл должен быть расположен в директории inc/img/)
{f} % Обтекание (без обтекания)
{H} % Положение рисунка (см. figure из пакета float)
{1\textwidth} % Ширина рисунка
{Пример работы программы} % Подпись рисунка




\section{Временные характеристики}

Результаты исследования временных замеров  приведены в таблице \ref{t:timings}.

\begin{table}[ht]
	\centering
	\caption{Полученная таблица замеров по времени различных реализаций алгоритмов сортировки}
	\begin{tabular}{|c|c|c|c|}
		\hline
		n     & Один поток(мс) & Несколько потоков(мс) & Число потоков \\ \hline
		70000 & 86.912                  & 87.369                                  & 0             \\ \hline
		70000 & 86.912                  & 2.7701                                  & 2             \\ \hline
		70000 & 86.912                  & 2.9471                                  & 4             \\ \hline
		70000 & 86.912                  & 2.7181                                  & 8             \\ \hline
		70000 & 86.912                  & 2.899                                   & 16            \\ \hline
		70000 & 86.912                  & 2.5946                                  & 32            \\ \hline
		70000 & 86.912                  & 2.9484                                  & 64            \\ \hline
		70000 & 86.912                  & 3.1136                                  & 96            \\ \hline
	\end{tabular}
	\label{t:timings}
\end{table}
Для таблицы \ref{t:timings} расчеты проводились с шагом изменения числа потоков 2, сортировки производились 100 раз, после чего результат усредняется. В качестве сортируемых значений использовались  числа от 1 до 10000000. Данные генерировались из равномерного распределения.
Значение $n$ определяет число элементов в массиве, значения из столбца <<Один поток>> показывают результаты замеров  работы реализации при использовании 1 потока, значение  из столбца <<Несколько потоков>> показывают результаты замеров работы многопоточногой реализации. <<Число потоков>> определяет число вспомогательных потоков, для получения результата при многопоточной реализации алгоритма.


По таблице \ref{t:timings} был построен график \ref{img:threads-cmp}. 

\includeimage
{threads-cmp} % Имя файла без расширения (файл должен быть расположен в директории inc/img/)
{f} % Обтекание (без обтекания)
{H} % Положение рисунка (см. figure из пакета float)
{1\textwidth} % Ширина рисунка
{Сравнение реализаций сортировок по времени с использованием логарифмической шкалы} % Подпись рисунка


Также были проведены замеры количества времени, необходимого для получения результата
и числа потоков, используемых при этом, данные замеры приведены в таблице \ref{t:timings-threads}.
Обозначения столбцов и условия получения данных аналогичны таблице \ref{t:timings}. По таблице  \ref{t:timings-threads}, был построен график \ref{diff-threads}.


\begin{table}[ht]
	\centering
	\caption{Полученная таблица замеров по времени для различного числа потоков}
	\begin{tabular}{|c|c|c|c|}
		\hline
		n     & Несколько потоков(мс) & Число потоков \\ \hline
		60000 & 2.0737                            & 2             \\ \hline
		60000 & 2.1153                            & 4             \\ \hline
		60000 & 2.1402                            & 8             \\ \hline
		60000 & 2.1318                            & 12            \\ \hline
		60000 & 2.169                             & 16            \\ \hline
		60000 & 2.1903                            & 20            \\ \hline
		60000 & 2.1193                            & 24            \\ \hline
		60000 & 2.1299                            & 28            \\ \hline
		60000 & 2.1435                            & 32            \\ \hline
		60000 & 2.1078                            & 36            \\ \hline
		60000 & 2.1282                            & 40            \\ \hline
		60000 & 2.1187                            & 44            \\ \hline
		60000 & 2.144                             & 48            \\ \hline
		60000 & 2.1525                            & 52            \\ \hline
		60000 & 2.1608                            & 56            \\ \hline
		60000 & 2.2299                            & 60            \\ \hline
		60000 & 2.1467                            & 64            \\ \hline
		60000 & 2.1435                            & 68            \\ \hline
		60000 & 2.1273                            & 72            \\ \hline
		60000 & 2.1394                            & 76            \\ \hline
		60000 & 2.1606                            & 80            \\ \hline
		60000 & 2.1907                            & 84            \\ \hline
		60000 & 2.198                             & 88            \\ \hline
		60000 & 2.1616                            & 92            \\ \hline
		60000 & 2.3131                            & 96            \\ \hline
		60000 & 2.5066                            & 100           \\ \hline
	\end{tabular}
	\label{t:timings-threads}
\end{table}


\includeimage
{diff-threads} % Имя файла без расширения (файл должен быть расположен в директории inc/img/)
{f} % Обтекание (без обтекания)
{H} % Положение рисунка (см. figure из пакета float)
{1\textwidth} % Ширина рисунка
{Зависимость времени получения результата от числа используемых потоков} % Подпись рисунка


Также была получена таблица \ref{t:timings-n}, в которой в многопоточной реализации было использовано 2 вспомогательных потока. По таблице \ref{t:timings-n} был получен график \ref{img:diff-time}. 
Обозначения столбцов и условия получения данных аналогичны таблице \ref{t:timings}.
\begin{table}[!ht]
	\centering
	\caption{Полученная таблица замеров по времени для различного числа элементов массива, при 2 вспомогательных потоках}
	\begin{tabular}{|c|c|c|}
		\hline
		n     & Один поток(мс) & Несколько потоков(мс) \\ \hline
		10000 & 11.149                  & 0.37046                           \\ \hline
		20000 & 23.616                  & 0.76908                           \\ \hline
		30000 & 34.857                  & 1.0772                            \\ \hline
		40000 & 49.326                  & 1.4374                            \\ \hline
		50000 & 60.081                  & 1.7727                            \\ \hline
		60000 & 70.217                  & 2.0737                            \\ \hline
		70000 & 86.912                  & 2.7701                            \\ \hline
	\end{tabular}
	\label{t:timings-n}
\end{table}


\includeimage
{diff-time} % Имя файла без расширения (файл должен быть расположен в директории inc/img/)
{f} % Обтекание (без обтекания)
{H} % Положение рисунка (см. figure из пакета float)
{1\textwidth} % Ширина рисунка
{Зависимость времени получения результата от числа элемнтов массива,график представлеен с использованием логарифмической шкалы} % Подпись рисунка


\section*{Вывод}
В результате анализа таблицы \ref{t:timings}, было получено, что при использовании 2 вспомогательных потоков при сортировке 70000 элементов требуется в 31 раз меньше времени для получения результата. 

Из таблицы  \ref{t:timings-threads}  следует, что увлечение числа используемых потоков, не всегда дает прирост к скорости получения результатов, при превышении числа логических ядер заметно увлечение времени для получения результата. Наихудший результат был получен при использовании 100 потоков (2.5 мс), данный результат в 1.2 раз хуже результата при использовании 2 потоков.

Из таблицы \ref{t:timings-n}, можно сделать выводы, что при увеличении числа элементов в массиве
время получения результата с использованием нескольких потоков также увеличивается. При увлечении числа сортируемых элементов с 10000 до 70000, время получения результата при использовании одного потока  увеличилось в 7.8 раз.
При увлечении числа сортируемых элементов с 10000 до 70000, время получения результата при использовании 2 вспомогательных потоков увеличилось в 7.4 раза.


