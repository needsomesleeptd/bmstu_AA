\chapter{Исследовательская часть}


В данном разделе будет описано исследование зависимости среднего числа генерируемых кадров от числа и типа примитивов на сцене. Также будет описаны технические характеристики устройства, на котором проводились замеры и приведен анализ полученных результатов.

\section{Технические характеристики}

Технические характеристики устройства, на котором выполнялись замеры времени, представлены далее.

\begin{enumerate}
	\item Процессор	Intel(R) Core(TM) i7-9750H CPU @ 2.60GHz, 2592 МГц, ядер: 6, логических процессоров: 12.
	\item Оперативная память: 16 ГБайт.
	\item Операционная система: Майкрософт Windows 10 Pro \cite{windows}.
	\item Использованная подсистема: WSL2 \cite{WSL2}.
\end{enumerate}

При замерах времени ноутбук был включен в сеть электропитания и был нагружен только системными приложениями.

\section{Параметризация муравьиного алгоритма}
В качестве исходной матрицы для параметаризации была использована матрица 
В результате параметаризации была получена таблицы~\ref{t:params_1}--\ref{t:params_3}, приведенные в приложении, таблицы имеют различный разброс расстояний: 1000, 100, 10 соответственно.
В данных таблице
столбцы обозначают соответственно:
\begin{enumerate}
	\item $\alpha$ --- параметр $\alpha$ при вычислении вероятности перехода в новый город;
	\item $eva$ --- коэффициент испарения феромонов;
	\item $days$ --- время жизни колонии;
	\item $optim$ --- результат решения полным перебором;
	\item $delta$ --- разница между решением полным перебором и решением муравьиного алгоритма.
\end{enumerate}
Замеры проводились 10 раз и выбирался результат с максимальным $delta$ от результата перебора.
Для рассматриваемых таблиц~\ref{t:params_1}--\ref{t:params_3} соответственно использовались матрицы расстояний~(\ref{eq:kd1}--\ref{eq:kd3}).

\begin{equation}
	\label{eq:kd1}
	K_{1} =\begin{bmatrix}
	 0 & 3434 & 934 & 3096 & 5374 & 3486 & 1740 & 1328 & 2672 & 2547 \\
		3434 & 0 & 4315 & 4726 & 3236 & 699 & 5121 & 2290 & 6050 & 5950 \\
		934 & 4315 & 0 & 3468 & 5892 & 4410 & 1161 & 2084 & 1997 & 1831 \\
		3096 & 4726 & 3468 & 0 & 7728 & 4259 & 2888 & 3917 & 3239 & 3330 \\
		5374 & 3236 & 5892 & 7728 & 0 & 3904 & 7021 & 4088 & 7889 & 7717 \\
		3486 & 699 & 4410 & 4259 & 3904 & 0 & 5069 & 2539 & 5974 & 5899 \\
		1740 & 5121 & 1161 & 2888 & 7021 & 5069 & 0 & 3061 & 934 & 831 \\
		1328 & 2290 & 2084 & 3917 & 4088 & 2539 & 3061 & 0 & 3987 & 3851 \\
		2672 & 6050 & 1997 & 3239 & 7889 & 5974 & 934 & 3987 & 0 & 205 \\
		2547 & 5950 & 1831 & 3330 & 7717 & 5899 & 831 & 3851 & 205 & 0 \\
	\end{bmatrix}
\end{equation}

\begin{equation}
	\label{eq:kd2}
K_{2} = \begin{bmatrix}
	0 & 48 & 72 & 36 & 34 & 89 & 85 & 8 & 22 & 11 \\
	48 & 0 & 33 & 23 & 78 & 61 & 16 & 1 & 92 & 31 \\
	72 & 33 & 0 & 100 & 51 & 9 & 66 & 58 & 36 & 23 \\
	36 & 23 & 100 & 0 & 41 & 43 & 45 & 51 & 31 & 67 \\
	34 & 78 & 51 & 41 & 0 & 33 & 22 & 38 & 32 & 63 \\
	89 & 61 & 9 & 43 & 33 & 0 & 46 & 71 & 9 & 32 \\
	85 & 16 & 66 & 45 & 22 & 46 & 0 & 41 & 51 & 78 \\
	8 & 1 & 58 & 51 & 38 & 71 & 41 & 0 & 100 & 23 \\
	22 & 92 & 36 & 31 & 32 & 9 & 51 & 100 & 0 & 33 \\
	11 & 31 & 23 & 67 & 63 & 32 & 78 & 23 & 33 & 0 \\
\end{bmatrix}
\end{equation}

\begin{equation}
	\label{eq:kd3}
K_{3} = \begin{bmatrix}
	0 & 1 & 9 & 0 & 9 & 5 & 2 & 10 & 7 & 10 \\
	1 & 0 & 9 & 10 & 9 & 1 & 9 & 1 & 9 & 7 \\
	9 & 9 & 0 & 4 & 8 & 7 & 4 & 6 & 8 & 5 \\
	0 & 10 & 4 & 0 & 9 & 0 & 1 & 7 & 3 & 1 \\
	9 & 9 & 8 & 9 & 0 & 8 & 9 & 8 & 7 & 5 \\
	5 & 1 & 7 & 0 & 8 & 0 & 1 & 4 & 8 & 2 \\
	2 & 9 & 4 & 1 & 9 & 1 & 0 & 9 & 8 & 9 \\
	10 & 1 & 6 & 7 & 8 & 4 & 9 & 0 & 4 & 10 \\
	7 & 9 & 8 & 3 & 7 & 8 & 8 & 4 & 0 & 9 \\
	10 & 7 & 5 & 1 & 5 & 2 & 9 & 10 & 9 & 0 \\
\end{bmatrix}
\end{equation}

\section{Проведение замеров}






