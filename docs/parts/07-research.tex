\chapter{Исследовательская часть}

В данном разделе будут приведены примеры работы программ, постановка исследования и сравнительный анализ алгоритмов на основе полученных данных.


\section{Технические характеристики}

Технические характеристики устройства, на котором выполнялись замеры времени, представлены далее.

\begin{enumerate}
	\item Процессор	Intel(R) Core(TM) i7-9750H ЦПУ 2592 МГц, ядер: 6, логических процессоров: 12.
	\item Оперативная память: 16 ГБ.
	\item Операционная система: Майкрософт Windows 10 Pro \cite{windows}.
	\item Использованная подсистема: WSL2 \cite{WSL2}.
\end{enumerate}

При замерах времени ноутбук был включен в сеть электропитания и был нагружен только системными приложениями.



\section{Демонстрация работы программы}

На рисунке \ref{img:work_ex} представлена демонстрация работы разработанного приложения для алгоритмов поиска.
\includeimage
{work_ex} % Имя файла без расширения (файл должен быть расположен в директории inc/img/)
{f} % Обтекание (без обтекания)
{H} % Положение рисунка (см. figure из пакета float)
{1\textwidth} % Ширина рисунка
{Пример работы программы} % Подпись рисунка



\section{Число сравнений}

В результате проведения замеров по числу сравнений были получены столбчатые диаграммы~\ref{img:bin_cmp}--\ref{img:lin_ind}, при выполнении замеров элементы массива были отсортированы по возрастанию и принимали значения от 0 до 1000. Результат числа сравнений под индексом $-1$ является результатом поиска элемента, отсутствующего в рассматриваемом массиве.

\includeimage
{bin_cmp} % Имя файла без расширения (файл должен быть расположен в директории inc/img/)
{f} % Обтекание (без обтекания)
{H} % Положение рисунка (см. figure из пакета float)
{1\textwidth} % Ширина рисунка
{Диаграмма зависимости числа сравнений от индекса массива при использовании бинарного поиска} % Подпись рисунк

\includeimage
{bin_ind} % Имя файла без расширения (файл должен быть расположен в директории inc/img/)
{f} % Обтекание (без обтекания)
{H} % Положение рисунка (см. figure из пакета float)
{1\textwidth} % Ширина рисунка
{Диаграмма зависимости числа сравнений от индекса массива при использовании бинарного поиска} % Подпись рисунк

\includeimage
{lin_ind} % Имя файла без расширения (файл должен быть расположен в директории inc/img/)
{f} % Обтекание (без обтекания)
{H} % Положение рисунка (см. figure из пакета float)
{1\textwidth} % Ширина рисунка
{Диаграмма зависимости числа сравнений от индекса массива при использовании линейного поиска поиска} % Подпись рисунк




\section*{Вывод}
Из диаграмм~\ref{img:bin_cmp}--\ref{img:lin_ind}, можно сделать вывод, что максимальное число сравнений при использовании реализации бинарного поиска равно 10, при использовании реализации линейного поиска --- 512, таким образом бинарный поиск при поиске в массиве длинной 512 в худшем случае для обоих алгоритмов (искомый элемент --- последний элемент массива) требует в 51.2 раза меньше сравнений, ввиду возможности рассмотрения подмассивов. Однако, при использовании бинарного поиска стоит учитывать условие упорядоченности массива.


