\chapter{Исследовательская часть}

В данном разделе будут приведены пример работы программы, условия исследования   и сравнительный анализ алгоритмов на основе полученных данных.

\section{Технические характеристики}

Технические характеристики устройства, на котором выполнялись замеры времени, представлены далее.

\begin{enumerate}
	\item Процессор	Intel(R) Core(TM) i7-9750H CPU @ 2.60GHz, 2592 МГц, ядер: 6, логических процессоров: 12.
	\item Оперативная память: 16 ГБайт.
	\item Операционная система: Майкрософт Windows 10 Pro \cite{windows}.
	\item Использованная подсистема: WSL2 \cite{WSL2}.
\end{enumerate}

При замерах времени ноутбук был включен в сеть электропитания и был нагружен только системными приложениями.


\section{Демонстрация работы программы}

На рисунке~\ref{img:work_ex} представлена демонстрация работы разработанного приложения для алгоритмов сортировок.
\includeimage
{work_ex} % Имя файла без расширения (файл должен быть расположен в директории inc/img/)
{f} % Обтекание (без обтекания)
{H} % Положение рисунка (см. figure из пакета float)
{1\textwidth} % Ширина рисунка
{Пример работы программы} % Подпись рисунка




\section{Временные характеристики}

Результаты исследования временных замеров  приведены в таблице~\ref{t:timings}.

\begin{table}[ht]
	\centering
	\caption{Полученная таблица замеров по времени различных реализаций алгоритмов сортировки}
	\begin{tabular}{|c|c|c|c|}
		\hline
	n       &  Один поток(мс) & Несколько потоков(мс) & Число потоков \\ \hline
	50000 & 58.886                  & 58.933                            & 0           \\ \hline
	50000 & 58.886                  & 46.289                            & 2           \\ \hline
	50000 & 58.886                  & 52.604                            & 4           \\ \hline
	50000 & 58.886                  & 54.353                            & 6           \\ \hline
	50000 & 58.886                  & 58.822                            & 8           \\ \hline
	50000 & 58.886                  & 57.853                            & 10          \\ \hline
	50000 & 58.886                  & 61.572                            & 12          \\ \hline
	50000 & 58.886                  & 64.308                            & 14          \\ \hline
	50000 & 58.886                  & 64.89                             & 16          \\ \hline
	50000 & 58.886                  & 67.216                            & 18          \\ \hline
	50000 & 58.886                  & 67.029                            & 20          \\ \hline
	50000 & 58.886                  & 68.892                            & 22          \\ \hline
	50000 & 58.886                  & 70.15                             & 24          \\ \hline
	50000 & 58.886                  & 72.656                            & 26          \\ \hline
	50000 & 58.886                  & 73.546                            & 28          \\ \hline
	50000 & 58.886                  & 72.768                            & 30          \\ \hline
	50000 & 58.886                  & 73.515                            & 32          \\ \hline
	50000 & 58.886                  & 75.199                            & 34          \\ \hline
	50000 & 58.886                  & 75.746                            & 36          \\ \hline
	50000 & 58.886                  & 76.953                            & 38          \\ \hline
	50000 & 58.886                  & 79.806                            & 40          \\ \hline
	50000 & 58.886                  & 77.823                            & 42          \\ \hline
	50000 & 58.886                  & 79.456                            & 44          \\ \hline
	50000 & 58.886                  & 81.63                             & 46          \\ \hline
	50000 & 58.886                  & 81.089                            & 48          \\ \hline
	\end{tabular}
	\label{t:timings}
\end{table}
Для таблицы~\ref{t:timings} расчеты проводились с шагом изменения числа дополнительных потоков 2, сортировки производились 100 раз, после чего результат усредняется. В качестве сортируемых значений использовались  числа от 1 до 10000000. Данные генерировались из равномерного распределения.
Значение $n$ определяет число элементов в массиве, значения из столбца <<Один поток>> показывают результаты замеров  работы реализации при использовании 1 потока, значение  из столбца <<Несколько потоков>> показывают результаты замеров работы многопоточной реализации. <<Число потоков>> определяет число вспомогательных потоков, для получения результата при многопоточной реализации алгоритма. Результаты замеров времени приведен в миллисекундах.


По таблице~\ref{t:timings} был построен график~\ref{img:threads-cmp}. 

\includeimage
{threads-cmp} % Имя файла без расширения (файл должен быть расположен в директории inc/img/)
{f} % Обтекание (без обтекания)
{h} % Положение рисунка (см. figure из пакета float)
{1\textwidth} % Ширина рисунка
{Сравнение реализаций сортировок слиянием по времени с использованием логарифмической шкалы} % Подпись рисунка










Также была получена таблица~\ref{t:timings-n}, в которой в многопоточной реализации был использован 1 вспомогательный поток.~По таблице~\ref{t:timings-n} был получен график~\ref{img:diff-time}. 
Обозначения столбцов и условия получения данных аналогичны таблице~\ref{t:timings}.

\begin{table}[H]
	\centering
	\caption{Полученная таблица замеров по времени многопоточной реализации алгоритма сортировки слиянием для различного числа элементов массива, при 1 вспомогательном потоке}
	\begin{tabular}{|c|c|c|}
		\hline
		n       & Один поток(мс) & Несколько потоков(мс) \\ \hline
		10000 & 11.193                  & 7.3873                            \\ \hline
		20000 & 23.173                  & 18.97                             \\ \hline
		30000 & 33.781                  & 28.963                            \\ \hline
		40000 & 47.825                  & 44.313                            \\ \hline
		50000 & 58.886                  & 54.198                            \\ \hline
		60000 & 69.215                  & 66.343                            \\ \hline
		70000 & 83.156                  & 80.055                            \\ \hline
	\end{tabular}
	\label{t:timings-n}
\end{table}




\includeimage
{diff-time} % Имя файла без расширения (файл должен быть расположен в директории inc/img/)
{f} % Обтекание (без обтекания)
{h} % Положение рисунка (см. figure из пакета float)
{1\textwidth} % Ширина рисунка
{Зависимость времени получения результата от числа элементов массива при различных реализациях сортировок} % Подпись рисунка


\section*{Вывод}
В результате анализа таблицы \ref{t:timings}, было получено, что при использовании 6 вспомогательных потоков при сортировке 50000 элементов требуется в 1.27 раз меньше времени для получения результата, чем при использовании одного потока. 

Из таблицы \ref{t:timings-n}, можно сделать выводы, что при увеличении числа элементов в массиве
время получения результата с использованием нескольких потоков также увеличивается. При увлечении числа сортируемых элементов с 10000 до 70000, время получения результата при использовании одного потока  увеличилось в 7.4 раза.
При увлечении числа сортируемых элементов с 10000 до 70000, время получения результата при использовании 1 вспомогательного потока увеличилось в 10.8 раз.


