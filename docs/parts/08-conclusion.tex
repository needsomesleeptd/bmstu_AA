\chapter*{\hfill{\centering  ЗАКЛЮЧЕНИЕ}\hfill}
\addcontentsline{toc}{chapter}{ЗАКЛЮЧЕНИЕ}

В результате исследования можно сделать вывод, что при малых размерностях матриц реализация алгоритма полного перебора тратит меньше времени на получение результата.
При размерности матрицы 8, реализации алгоритма полного перебора, необходимо 0.09 секунд для получения результата, что в 3.13 меньше времени при сравнении с муравьиным алгоритмом.
Начиная с размерности 9, реализации алгоритма полного перебора необходимо 1.15 секунд, что в 1.45 раз больше, чем при использовании муравьиного алгоритма.
Таким образом, выбор алгоритма зависит от размера матрицы расстояний, введенной в конкретной задаче.

При увлечении числа дней существования колонии отличия результата муравьиного алгоритма от оптимального уменьшается, независимо от разброса расстояний. От значений $\alpha$ и $eva$ зависимостей не обнаружено, что говорит о том, что значения параметров должны регулироваться для каждой конкретной задачи отдельно. 

Поставленная цель: разработка программного обеспечения, решающего задачу полным перебором и с помощью муравьиного алгоритма, была достигнута.

Для поставленной цели были выполнены все задачи.
\begin{enumerate}
	\item Описать задачу коммивояжера.
	\item Описать методы решения задачи коммивояжера --- метод полного перебора и метод на основе муравьиного алгоритма.
	\item Привести схемы муравьиного алгоритма и алгоритма, позволяющего решить задачу коммивояжера методом полного перебора.
	\item Разработать и реализовать программный продукт, позволяющий решить задачу коммивояжера исследуемыми методами.
	\item Сравнить по времени метод полного перебора и метод на основе муравьиного алгоритма.
	\item Описать и обосновать полученные результаты в отчете о выполненной лабораторной работе.
\end{enumerate}