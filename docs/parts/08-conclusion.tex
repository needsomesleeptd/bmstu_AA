\chapter*{\hfill{\centering  ЗАКЛЮЧЕНИЕ}\hfill}
\addcontentsline{toc}{chapter}{ЗАКЛЮЧЕНИЕ}

В результате исследования было определено, что при использовании реализации конвейерной обработки заявок время получения результата сократилось, относительно последовательной обработки заявок при увелечении числа заявок и при увелечении числа элементов массива.
При обработке 110 заявок при 9000 сортируемых элементах результат был получен в 1.08 раз быстрее, при обработке 100 заявок при 15000 сортируемых элементов, результат был получен в 1.09 раз быстрее, данный результат был получен ввиду параллельной обработки заявок на различных этапах различными потоками.





Поставленная цель: исследование быстродействия метода конвейерной обработки данных, была достигнута.

Для поставленной цели были выполнены все поставленные задачи:
\begin{enumerate}
	\item описать организацию конвейерной обработки данных;
	\item описать алгоритмы обработки данных, которые будут использоваться в текущей лабораторной работе;
	\item реализовать программу, выполняющую конвейерную обработку с количеством лент не менее трех  и программу обрабатывающую заявки последовательно;
	\item провести сравнительный анализ времени работы реализаций.
\end{enumerate}

