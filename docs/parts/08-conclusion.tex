\chapter*{\hfill{\centering  ЗАКЛЮЧЕНИЕ}\hfill}
\addcontentsline{toc}{chapter}{ЗАКЛЮЧЕНИЕ}

В результате исследования было определено, что время алгоритмов нахождения расстояний Левенштейна и Дамерау-Левенштейна растет с ускорением при увеличении длин строк.
Лучшие показатели по времени дает реализация алгоритма нахождения расстояния Дамерау-Левенштейна с использованием матрицы.
При анализе таблицы  \ref{t:timings}, было установлено, что данная реализация требует в 294.662 раз меньше времени для получения расстояния, чем реализация с использованием рекурсии без мемоизации и в 1.4 раз меньше времени при использовании рекурсии с мемоизацией, при длине слов в 12 символов.
При этом требования к памяти в итеративной реализации с использованием матрицы растут пропорционально произведению длин введенных строк. Из таблицы \ref{t:memory}, можно сделать выводы, что при длине строк в 100 символов, реализации с использованием матрицы 
необходимо в 1.7 раз больше памяти чем реализации с помощью рекурсии без мемоизации, реализации с помощью рекурсии с мемоизацией необходимо в 1.7 больше памяти чем реализации с помощью матрицы.


Цели данной лабораторной работы были достигнуты, а именно описание и исследование особенностей задач динамического программирования на алгоритмах Левенштейна и Дамерау-Левенштейна.

Для достижения поставленной целей были выполнены следующие задачи.
\begin{enumerate}[label={\arabic*)}]
	\item Описаны алгоритмы поиска расстояния Левенштейна и \newline Дамерау-Левенштейна.
	\item Создано программное обеспечение, реализующее следующие алгоритмы.
	\begin{itemize}[label=---]
		\item нерекурсивный метод поиска расстояния Левенштейна;
		\item нерекурсивный метод поиска расстояния Дамерау-Левенштейна;
		\item рекурсивный метод поиска расстояния Дамерау-Левенштейна;
		\item рекурсивный с кешированием метод поиска расстояния Дамерау-Левенштейна.
	\end{itemize}
	\item Выбраны инструменты для замера процессорного времени выполнения реализаций алгоритмов.
	\item Проведены анализ затрат работы программы по времени и по памяти, выяснены влияющие на них характеристики. 
\end{enumerate}
