\chapter*{\hfill{\centering  ЗАКЛЮЧЕНИЕ}\hfill}
\addcontentsline{toc}{chapter}{ЗАКЛЮЧЕНИЕ}

В результате исследования было получено, что при использовании 2 вспомогательных потоков при сортировке 70000 элементов требуется в 31 раз меньше времени для получения результата, так как слияние подмассивов наибольших размеров происходит в различных потоках. Также было выявлено, что увлечение числа используемых потоков не всегда дает прирост к скорости получения результатов,при превышении числа логических ядер, заметно увлечение времени для получения результата, ввиду необходимости частого переключения контекста потока. Наихудший результат был получен при использовании 100 потоков (2.5 мс), данный результат в 1.2 раз хуже результата при использовании 2 потоков. Также можно сделать вывод, что при увеличении числа элементов в массиве
время получения результата с использованием нескольких потоков также увеличивается. При увлечении числа сортируемых элементов с 10000 до 70000, время получения результата при использовании одного потока  увеличилось в 7.8 раз.
При увлечении числа сортируемых элементов с 10000 до 70000, время получения результата при использовании 2 вспомогательных потоков увеличилось в 7.4 раза.


Поставленная цель: изучение принципов и получение навыков организации параллельного выполнения операций, была достигнута.

Для поставленной цели были выполнены все поставленные задачи.
\label{sec:targets}
\begin{enumerate}
	\item Описать алгоритм сортировки слиянием.
	\item Разработать версии  приведенного алгоритма, при использовании 1 потока и нескольких потоков.
	\item Определить средства программной реализации.
	\item Реализовать разработанные алгоритмы.
	\item Выполнить замеры процессорного времени работы различных реализаций алгоритма.
	\item Провести анализ времени получения отсортированных данных.
\end{enumerate}

 

