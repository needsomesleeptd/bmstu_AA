\chapter*{\hfill{\centering  ЗАКЛЮЧЕНИЕ}\hfill}
\addcontentsline{toc}{chapter}{ЗАКЛЮЧЕНИЕ}

В результате исследования что максимальное число сравнений при использовании реализации бинарного поиска равно 10, при использовании реализации линейного поиска --- 512, таким образом бинарный поиск при поиске в массиве длинной 512 в худшем случае для обоих алгоритмов (искомый элемент --- последний элемент массива) требует в 51.2 раза меньше сравнений, ввиду возможности разделения массива пополам и рассмотрения значений только из подмассивов. Однако стоит учитывать условие упорядоченности массива, при использовании бинарного поиска.


Поставленная цель: описание метода бинарного и классического поиска и их сравнения по числу сравнений элементов, была достигнута.
Для поставленной цели были выполнены все задачи.
\begin{enumerate}
	\item Описать алгоритмы поиска.
	\item Создать программное обеспечение, реализующее линейный и бинарный алгоритмы поиска.
	\item Замерить число сравнений различных алгоритмов.
	\item Провести анализ полученных результатов.
\end{enumerate}
