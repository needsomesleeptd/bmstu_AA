\chapter*{Заключение}
\addcontentsline{toc}{chapter}{Заключение}

В результате исследования было определено, что время алгоритмов нахождения расстояний Левенштейна и Дамерау-Левенштейна растет в геометрической прогрессии при увеличении длин строк.
Лучшие показатели по времени дает матричная реализация алгоритма нахождения расстояния Дамерау-Левенштейна и его рекурсивная реализация с кешем, использование которых приводит к 21-кратному превосходству по времени работы уже на длине строки в 4 символа за счет сохранения необходимых промежуточных вычислений.
При этом итеративная реализации с использованием матрицы занимают довольно много памяти при большой длине строк. 

Цель данной лабораторной работы были достигнуты, а именно описание и исследование особенностей задач динамического программирования на алгоритмах Левенштейна и Дамерау-Левенштейна.

Для достижения поставленной целей были выполнены следующие задачи.
\begin{enumerate}[label={\arabic*)}]
	\item Описаны алгоритмы поиска расстояния Левенштейна и \newline Дамерау-Левенштейна;
	\item Создано программное обеспечение, реализующее следующие алгоритмы.
	\begin{itemize}[label=---]
		\item нерекурсивный метод поиска расстояния Левенштейна;
		\item нерекурсивный метод поиска расстояния Дамерау-Левенштейна;
		\item рекурсивный метод поиска расстояния Дамерау-Левенштейна;
		\item рекурсивный с кешированием метод поиска расстояния Дамерау-Левенштейна.
	\end{itemize}
	\item Выбраны инструменты для замера процессорного времени выполнения реализаций алгоритмов.
	\item Проведены анализ затрат работы программы по времени и по памяти, выяснить влияющие на них характеристики. 
\end{enumerate}
\end{document}