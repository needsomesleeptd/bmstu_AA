\chapter*{\hfill{\centering  ЗАКЛЮЧЕНИЕ}\hfill}
\addcontentsline{toc}{chapter}{ЗАКЛЮЧЕНИЕ}

В результате исследования было определено, что  реализация алгоритма  блочной сортировки требует в 1.95 раз больше  времени, чем сортировка перемешиванием, поразрядная сортировка требует в 65 раз меньше времени, чем блочная сортировка. Ввиду того, что сортируемые числа неотрицательные и имеют малое число разрядов поразрядная сортировка оказалась самой эффективной по временным затратам, блочная сортировка показала лучший результат по сравнению с сортировкой перемешиваем благодаря равномерному распределению данных.



Цели данной лабораторной работы были достигнуты, а именно описание и исследование особенностей задач динамического программирования на алгоритмах Левенштейна и Дамерау-Левенштейна.

Целью данной лабораторной работы является изучение и исследование трудоемкости алгоритмов сортировки.

Для поставленной цели были выполнены следующие задачи.
\begin{enumerate}
	\item Описать алгоритмы сортировки;
	\item создать программное обеспечение, реализующее следующие алгоритмы сортировки;
	\begin{itemize}
		\item перемешиванием;
		\item блочная;
		\item поразрядная;
	\end{itemize}
	\item оценить трудоемкости сортировок;
	\item замерить время реализации;
	\item провести анализ затрат работы программы по времени, выяснить влияющие на них характеристики;
\end{enumerate}
