\chapter{Выполнения задания}

\section{Выбор языка программирования}
Для выполнения домашнего задания был выбран язык \texttt{C++}.

\section{Код программы}

Код программы приведен в листинге \ref{lst:merge} в приложении.




\section{Графовые модели программы}

\subsection{Граф управления}

На рисунке \ref{img:GU} представлен граф управления.

\includeimage
{GU} % Имя файла без расширения (файл должен быть расположен в директории inc/img/)
{f} % Обтекание (без обтекания)
{H} % Положение рисунка (см. figure из пакета float)
{0.5\textwidth} % Ширина рисунка
{Граф управления} % Подпись рисунка



\subsection{Информационный граф}

На рисунке \ref{img:IG} представлен информационный граф.

\includeimage
{IG} % Имя файла без расширения (файл должен быть расположен в директории inc/img/)
{f} % Обтекание (без обтекания)
{H} % Положение рисунка (см. figure из пакета float)
{1\textwidth} % Ширина рисунка
{Информационный граф} % Подпись рисунка

\subsection{Информационная история}
Рассмотрим следующий массив: \texttt{a = [7, 4, 2, 1]}.

На рисунках \ref{img:OI_1}--\ref{img:OI_2} представлена операционная история программы для этого массива, а также и информационная история \ref{img:II_1}--\ref{img:II_2}.

\includeimage
{II_1} % Имя файла без расширения (файл должен быть расположен в директории inc/img/)
{f} % Обтекание (без обтекания)
{H} % Положение рисунка (см. figure из пакета float)
{1\textwidth} % Ширина рисунка
{Информационная история (начало)} % Подпись рисунка

\includeimage
{II_2} % Имя файла без расширения (файл должен быть расположен в директории inc/img/)
{f} % Обтекание (без обтекания)
{H} % Положение рисунка (см. figure из пакета float)
{1\textwidth} % Ширина рисунка
{Информационная история (продолжение)} % Подпись рисунка


\subsection{Операционная история}
На рисунках \ref{img:OI_1}--\ref{img:OI_2} представлена операционная история программы для данного массива.

\includeimage
{OI_1} % Имя файла без расширения (файл должен быть расположен в директории inc/img/)
{f} % Обтекание (без обтекания)
{H} % Положение рисунка (см. figure из пакета float)
{0.8\textwidth} % Ширина рисунка
{Операционная  история (начало)} % Подпись рисунка

\includeimage
{OI_2} % Имя файла без расширения (файл должен быть расположен в директории inc/img/)
{f} % Обтекание (без обтекания)
{H} % Положение рисунка (см. figure из пакета float)
{1\textwidth} % Ширина рисунка
{Операционная  история (продолжение)} % Подпись рисунка


\section*{Вывод}
Сортировка разделяет массив на непересекающиеся подмассивы, а затем
объединяет их, возможно организовать объединение непересекающихся подмассивов  в отдельных потоках.

