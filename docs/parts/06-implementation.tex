\chapter{Технологический раздел}

В данной части работы будут описаны средства реализации программы, а также листинги и  модульные  тесты.



	
\section{Средства реализации}
Алгоритмы для данной лабораторной работы были реализованы на языке C++, при использовании компилятора gcc версии 10.5.0, так как в стандартной библиотеке приведенного языка
присутствует функция \texttt{clock}, которая  позволяет получить количество тиков с времени начала выполнения программы, при делении полученного значения на макропеременную \texttt{CLOCKS\_PER\_SEC}, возможно получение значения времени в секундах~\cite{cpp-time}.

Для создания потоков и работы с ними был использован класс \texttt{thread} из стандартной библиотеки выбранного языка~\cite{std-thread}.
\includelistingpretty
{thead-example.cpp} % Имя файла с расширением (файл должен быть расположен в директории inc/lst/)
{c++} % Язык программирования (необязательный аргумент)
{Пример работы с классом thread} % Подпись листинга

В листинге \ref{lst:thead-example.cpp}, приведен работы с описанным классом, каждый объект класса представляет собой поток операционной системы, что позволяет нескольким функциям выполняться параллельно~\cite{std-thread}. 

Поток начинает свою работу после создания объекта, соответствующего класса, запуская функцию, приведенную в его конструкторе~\cite{std-thread}.
В данном примере будет запущен 1 поток, который выполнит функцию \texttt{foo},
которая выведет число 10 на экран.


\section{Реализация алгоритмов}
Листинги~\ref{lst:merge.cpp}--\ref{lst:mergeSortMultiThread.cpp} исходных кодов программ приведены в приложении. 

\section{Тестирование}

В таблице \ref{t:mod_tests} приведены модульные  тесты для разработанных алгоритмов сортировки. Все приведенные массивы были отсортированы с помощью сортировки слиянием, в столбце <<Один поток>>
показаны результаты использования реализации с 1 потоком,  в столбце <<Несколько потоков>> показаны результаты использования реализации с несколькими потоками. Все тесты пройдены успешно.
\begin{table}[ht]
	\small
	\begin{center}
		\begin{threeparttable}
			\caption{Модульные тесты}
			\label{t:mod_tests}
			\begin{tabular}{|c|c|c|c|c|}
				\hline
				\bfseries Массив
				& \bfseries Размер
				& \bfseries Ожидаемый р-т
				& \multicolumn{2}{c|}{\bfseries Фактический результат} \\ \cline{4-5}
				& & & \bfseries Один поток & \bfseries  Несколько потоков \\
				\hline
				1 2 3 4  & 4 & 1 2 3 4 & 1 2 3 4 & 1 2 3 4 \\
				\hline
				4 3 2 1 & 4 & 4 3 2 1 & 4 3 2 1 & 4 3 2 1 \\
				\hline
				3 5 1 6 & 4 & 1 3 5 6  & 1 3 5 6 & 1 3 5 6 \\
				\hline
				-5 -1 -3 -4 -2 & 5 & -5 -4 -3 -2 -1 & -5 -4 -3 -2 -1 & -5 -4 -3 -2 -1 \\
				\hline
				1 -3 2 9 -9 & 5 & -9 -3 1 2 9  & -9 -3 1 2 91 & -9 -3 1 2 9 \\
				\hline
			\end{tabular}	
		\end{threeparttable}	
	\end{center}
\end{table}


\textbf{Вывод}

 В данной части работы были представлены листинги реализованных алгоритмов и тесты, успешно пройденные программой.
