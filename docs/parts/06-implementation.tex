\chapter{Технологический раздел}

В данной части работы будут описаны средства реализации программы, а также листинги и функциональные тесты.

\section{Средства реализации}
Алгоритмы для данной лабораторной работы были реализованы на языке \texttt{Python}, при использовании компилятора, так как в стандартной библиотеке приведенного языка
присутствует функция \texttt{process\_time}, позволяющая измерять процессорное время~\cite{process_time}.
	
	
	
\section{Реализация алгоритмов}
Листинги исходных кодов программ~\ref{lst:antAlg.py}--\ref{lst:updPher.py} приведены в приложении.
	

\section{Тестирование}
В таблице~\ref{t:func_t}, приведены функциональные тесты программы,
значения в столбце <<Результат программы>>, обозначает минимальное расстояние и порядок обхода городов, значения в данном столбце были получены при использовании обоих алгоритмов. Все тесты были пройдены успешно.
\begin{center}
	\captionsetup{justification=raggedright,singlelinecheck=off}
	\begin{longtable}[c]{|c|c|c|c|c|}
		\caption{Функциональные тесты\label{t:func_t}} \\ \hline
		Матрица смежности & Ожидаемый результат & Результат программы \\
		\hline
		$ \begin{pmatrix}
		 0 &  1  & 5 &  6\\
		1  & 0  & 8  & 8\\
		5  & 8 &  0 & 10\\
		6  & 8 & 10 &  0
		
		\end{pmatrix}$ &
		14, [2, 0, 1, 3] &
		14, [2, 0, 1, 3] \\
		
		$ \begin{pmatrix}
			0 & 1 & 2 \\
			1 & 0 & 3 \\
			2 & 3 & 0	
		\end{pmatrix}$ &
		3, [1, 0, 2] &
		3, [1, 0, 2] \\
		
		$ \begin{pmatrix}
		 0 & 3\\
		 3 & 0
		\end{pmatrix}$ &
		3, [0, 1] &
		3, [0, 1] \\
		\hline
	\end{longtable}
\end{center}
	
\section*{Вывод}

В данной части работы были описаны средства реализации и представлены листинги реализованных алгоритмов и тесты, успешно пройденные программой.