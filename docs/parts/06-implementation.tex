\chapter{Технологический раздел}

В данной части работы будут описаны средства реализации программы, а также листинги, модульные и функциональные тесты.

\section{Средства реализации}
Алгоритмы для данной лабораторной работы были реализованы на языке C++, при использовании компилятора gcc версии 10.5.0, так как в стандартной библиотеке приведенного языка
присутствует функция \texttt{clock\_gettime}, которая (при использовании макропеременной \texttt{CLOCK\_THREAD\_CPUTIME\_ID}) позволяет рассчитать процессорное время конкретного потока \cite{cpp-time}.



\section{Реализация алгоритмов}
Листинги исходных кодов программ  \ref{lst:radix_sort}--\ref{lst:block_sort} приведены в приложении.


\section{Тестирование}

В таблице \ref{tbl:mod_tests} приведены функциональные тесты для разработанных алгоритмов сортировки. Все тесты пройдены успешно.
\begin{table}[ht]
	\small
	\begin{center}
		\begin{threeparttable}
			\caption{Модульные тесты}
			\label{t:mod_tests}
			\begin{tabular}{|c|c|c|c|c|}
				\hline
				\bfseries Массив
				& \bfseries Размер
				& \bfseries Ожидаемый р-т
				& \multicolumn{2}{c|}{\bfseries Фактический результат} \\ \cline{4-5}
				& & & \bfseries Шелла/Пирам. & \bfseries Бусинами \\
				\hline
				41 67 34 0 69 & 5 & 0 34 41 67 69 & 0 34 41 67 69 & 0 34 41 67 69 \\
				\hline
				31 57 24 -10 59 & 5 & -10 24 31 57 59 & -10 24 31 57 59 & - \\
				\hline
				1 2 3 4 5 & 5 & 1 2 3 4 5 & 1 2 3 4 5 & 1 2 3 4 5 \\
				\hline
				100 88 76 65 43 & 5 & 43 65 76 88 100 & 43 65 76 88 100 & 43 65 76 88 100 \\
				\hline
				-59 -33 -66 -100 -31 & 5 & -100 -66 -59 -33 -31 & -100 -66 -59 -33 -31 & - \\
				\hline
			\end{tabular}	
		\end{threeparttable}	
	\end{center}
\end{table}