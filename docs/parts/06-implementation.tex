\chapter{Технологический раздел}

В данной части работы будут описаны средства реализации программы, а также листинги, модульные и функциональные тесты.

\section{Средства реализации}
Алгоритмы для данной лабораторной работы были реализованы на языке C++, при использовании компилятора gcc версии 10.5.0, так как в стандартной библиотеке приведенного языка
присутствует структура данных~---~вектор, которая позволяет хранить данные и которую возможно упорядочить с помощью средств стандартной библиотеки~\cite{cpp-vec}.



\section{Реализация алгоритмов}
Листинги исходных кодов программ  \ref{lst:bin_s.cpp}--\ref{lst:lin_s.cpp} приведены в приложении.


\section{Тестирование}

В таблице~\ref{t:mod_tests} приведены модульные тесты для разработанных алгоритмов поиска, в столбце <<Цель>> находятся искомые значения. Все тесты пройдены успешно.
\begin{table}[ht]
	\small
	\begin{center}
		\begin{threeparttable}
			\caption{Модульные тесты}
			\label{t:mod_tests}
			\begin{tabular}{|c|c|c|c|c|}
				\hline
				\bfseries Массив
				& \bfseries Цель
				& \bfseries Ожидаемый р-т
				& \multicolumn{2}{c|}{\bfseries Фактический результат} \\ \cline{4-5}
				& & & \bfseries Бинарный поиск & \bfseries Линейный поиск \\
				\hline
				1 2 3 4  & 1 & 0 &0 & 0 \\
				\hline
				1 2 3 4 & 4 & 3 & 3 & 3 \\
				\hline
				-9 -3 -2 -1 1 2 3 & -1 & 3  & 3 & 3 \\
				\hline
				-9 -3 -2 -1 1 2 3 & 1 & 4  & 4 & 4 \\
				\hline
				-9 -3 -2 -1 1 2 3 & -3 & 1  & 1 & 1 \\
				\hline
			\end{tabular}	
		\end{threeparttable}	
	\end{center}
\end{table}

\textbf{Вывод}

В данной части работы были представлены листинги реализованных алгоритмов и тесты, успешно пройденные программой.