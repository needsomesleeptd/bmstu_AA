\chapter{Технологический раздел}

В данной части работы будут описаны средства реализации программы, а также листинги, модульные и функциональные тесты.

\section{Средства реализации}
Алгоритмы для данной лабораторной работы были реализованы на языке C++, при использовании компилятора gcc версии 10.5.0, так как в стандартной библиотеке приведенного языка
присутствует функция \texttt{clock\_gettime}, которая (при использовании макропеременной \texttt{CLOCK\_THREAD\_CPUTIME\_ID}) позволяет рассчитать процессорное время конкретного потока \cite{cpp-time}.



\section{Реализация алгоритмов}
Листинги исходных кодов программ  \ref{lst:radix_sort}--\ref{lst:block_sort} приведены в приложении.


\section{Тестирование}

В таблице \ref{tbl:mod_tests} приведены функциональные тесты для разработанных алгоритмов сортировки. Все тесты пройдены успешно.
\begin{table}[ht]
	\small
	\begin{center}
		\begin{threeparttable}
			\caption{Модульные тесты}
			\label{t:mod_tests}
			\begin{tabular}{|c|c|c|c|c|}
				\hline
				\bfseries Массив
				& \bfseries Размер
				& \bfseries Ожидаемый р-т
				& \multicolumn{2}{c|}{\bfseries Фактический результат} \\ \cline{4-5}
				& & & \bfseries Блочная/Перемеш. & \bfseries Поразрядная \\
				\hline
				1 2 3 4  & 4 & 1 2 3 4 & 1 2 3 4 & 1 2 3 4 \\
				\hline
				4 3 2 1 & 4 & 4 3 2 1 & 4 3 2 1 & 4 3 2 1 \\
				\hline
				3 5 1 6 & 4 & 1 3 5 6  & 1 3 5 6 & 1 3 5 6 \\
				\hline
				-5 -1 -3 -4 -2 & 5 & -5 -4 -3 -2 -1 & -5 -4 -3 -2 -1 & -5 -4 -3 -2 -1 \\
				\hline
				1 -3 2 9 -9 & 5 & -9 -3 1 2 9  & -9 -3 1 2 91 & -9 -3 1 2 9 \\
				\hline
			\end{tabular}	
		\end{threeparttable}	
	\end{center}
\end{table}

\textbf{Вывод}
Были представлены листинги реализованных алгоритмов и тесты, успешно пройденные программой.