\chapter{Технологический раздел}

В данной части работы будут описаны средства реализации программы, а также листинги и функциональные тесты.

\section{Средства реализации}
Алгоритмы для данной лабораторной работы были реализованы на языке C++, при использовании компилятора gcc версии 10.5.0, так как в стандартной библиотеке приведенного языка
присутствует функция \texttt{clock\_gettime}, которая  позволяет получить реальное время а также класс \texttt{queue}, упрощающий использование очереди~\cite{cpp-time,queue}.



\section{Реализация алгоритмов}
Листинги исходных кодов программ  \ref{lst:async.cpp}--\ref{lst:atom_queue.hpp} приведены в приложении.


\section{Тестирование}

В таблице \ref{t:tests} приведены функциональные тесты для разработанных алгоритмов, фактический результат был получен на обеих реализациях обработок заявок. Все тесты пройдены успешно.
\begin{table}[ht]
	\small
	\begin{center}
			\caption{Функциональные тесты}
			\label{t:tests}
		\begin{tabular}{|c|c|c|c|}
			\hline
			Число заявок & Размер Массива & Ожидаемый р-т & Фактический р-т \\
			\hline
			2 & -2 & Вывод предупреждения & Вывод предупреждения \\ 	\hline
			-2 & 2 & Вывод предупреждения & Вывод предупреждения \\ \hline
			2 & 2 & Вывод лога & Вывод лога \\ \hline
			80 & 10 &  Вывод лога & Вывод лога \\ \hline
		\end{tabular}
	\end{center}
\end{table}

\section*{Вывод}
В данной части работы были представлены листинги реализованных алгоритмов и тесты, успешно пройденные программой.