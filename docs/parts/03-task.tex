\chapter{Описание задачи}
В данной части работы, будет описано полученное задание, а также смысл графового представления программы.

\section{Задание}

Описать четырьмя графовыми моделями (граф управления, информационный граф, опрерационная история, информационная история) последовательный алгоритм либо фрагмент алгоритма, содержащий от 15 значащих строк кода и от двух циклов, один из которых является вложенным в другой.

\textbf{Вариант 17:} в качестве реализуемого алгоритма~--- сортировка слиянием.

\section{Графовые модели программы}

Программа представлена в виде графа: набор вершин и множество соединяющих их направленных дуг.


Выделяют 2 типа дуг \cite{graph}:
\begin{enumerate}
    \item операционное отношение~--- по передаче управления;
    \item информационное отношение~--- по передаче данных.
\end{enumerate}

Граф управления~--- модель, в который \textbf{вершины}~---операторы, \textbf{дуги}~--- операционные отношения.

Информационный граф~--- модель, в которой \textbf{вершины}: операторы, \textbf{дуги}~--- информационные отношения.

Операционная история~--- модель, в которой \textbf{вершины}: срабатывание операторов, \textbf{дуги}~--- операционные отношения.

Информационная история~--- модель, в которой \textbf{вершины}: срабатывание операторов, \textbf{дуги}~--- информационные отношения.

Графы более компактны, однако менее информативны, чем истории. Истории менее комактны, однако более информативны, чем графы.
