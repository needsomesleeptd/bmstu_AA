\chapter{Конструкторский раздел}



В этом разделе будут рассмотрены различные реализации Классического алгоритма умножения матриц и алгоритм умножения матриц
Винограда


\section{Реализации алгоритмов}

В данной части работы будут рассмотрены различные реализации алгоритмов
умножения матриц и представлены схемы алгоритмов данных реализаций.

На вход алгоритмов подаются матрицы $M1$ и $M2$ с размерностями $n1\times m1$ и $n2\times m2$.



\includeimage
{std} % Имя файла без расширения (файл должен быть расположен в директории inc/img/)
{f} % Обтекание (без обтекания)
{h} % Положение рисунка (см. figure из пакета float)
{0.9\textwidth} % Ширина рисунка
{Стандартный алгоритм расчета произведения матриц} % Подпись рисунка

\includeimage
{vin} % Имя файла без расширения (файл должен быть расположен в директории inc/img/)
{f} % Обтекание (без обтекания)
{h} % Положение рисунка (см. figure из пакета float)
{1\textwidth} % Ширина рисунка
{Алгоритм расчета произведения матриц Винограда} % Подпись рисунка

\includeimage
{strassen} % Имя файла без расширения (файл должен быть расположен в директории inc/img/)
{f} % Обтекание (без обтекания)
{h} % Положение рисунка (см. figure из пакета float)
{1\textwidth} % Ширина рисунка
{Алгоритм расчета произведения матриц Штрассена} % Подпись рисунка

\includeimage
{vin-opt} % Имя файла без расширения (файл должен быть расположен в директории inc/img/)
{f} % Обтекание (без обтекания)
{h} % Положение рисунка (см. figure из пакета float)
{1\textwidth} % Ширина рисунка
{} % Подпись рисунка

\includeimage
{vin-opt2} % Имя файла без расширения (файл должен быть расположен в директории inc/img/)
{f} % Обтекание (без обтекания)
{h} % Положение рисунка (см. figure из пакета float)
{1\textwidth} % Ширина рисунка
{Оптимизированный алгоритм Винограда расчета произведения матриц} % Подпись рисунка



\subsection{Оценка трудоемкости}

Для последующего вычисления трудоемкости необходимо ввести модель вычислений.

\begin{enumerate}
	
	\item Трудоемкость следующих базовых операций единична:
	+, -, =, +=, -=, ==, !=, <, >, <=, >=, [], ++, --, <<, >>.
	
	Операции *, \%, / имеют трудоемкость 2.
	
	\item Трудоемкость цикла for(k = 0; k < N; k++) \{тело цикла\} рассчитывается как:
	\begin{equation}
		\label{for:for}
		f_{for} = f_{\text{инициал.}} + f_{\text{сравн.}} + N(f_{\text{тела}} + f_{\text{инкр.}} + f_{\text{сравн.}})
	\end{equation}
	
	\item Трудоемкость условного оператора \code{if (условие) then A else B} рассчитывается как:
	
	\begin{equation}
		\label{for:if}
		f_{if} = f_{\text{условия}} +
		\begin{cases}
			min(f_A, f_B), & \text{л.с.}\\
			max(f_A, f_B), & \text{х.с.}
		\end{cases}
	\end{equation}
	
\end{enumerate}


Произведем теоретическую оценку трудоемкости алгоритмов умножения матриц.


\section{Структуры данных}

Для реализации алгоритмов, будут использованы следующие типы данных:
Для реализации выбранных алгоритмов были использованы следующие структуры данных:
\begin{enumerate}
	\item Матрица --- массив векторов типа \texttt{int};
	\item размерности матрицы --- целые значение типа \texttt{size\_t}.
\end{enumerate}



\section*{Вывод}

В данном разделе на основе теоретических данных были построены схемы требуемых алгоритмов, выбраны используемые типы данных.
