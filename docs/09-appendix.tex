\begin{appendices}
	\chapter{}
	\begin{lstlisting}[label=lst:std,caption=Реализация стандартного алгоритма расчета произведения матриц]
	Matrix Matrix::dot(const Matrix& other)
	{
		if (this->_m != other._n)
		return Matrix(0, 0);
		
		Matrix mat_res = Matrix(this->_n, other._m);
		for (size_t i = 0; i < this->_n; i++)
		{
			for (size_t j = 0; j < other._m; j++)
			{
				mat_res._table[i][j] = 0;
				for (size_t k = 0; k < other._n; k++)
				mat_res._table[i][j] = mat_res._table[i][j] + this->_table[i][k] * other._table[k][j];
			}
		}
		return mat_res;
	}
	\end{lstlisting}
	
	
	
	\begin{lstlisting}[label=lst:vin,caption=Реализация алгоритма расчета произведения матриц Винограда]
	if (this->_m != other._n)
	return Matrix(0, 0);
	
	size_t n = this->_table.size();
	size_t m = other._table.size();
	size_t t = other._table[0].size();
	
	Matrix mat_res = Matrix(n, t);
	std::vector<int> rowFactor(n);
	std::vector<int> columnFactor(t);
	bool isEvenColumns = (m % 2 == 0);
	for (size_t i = 0; i < n; i++)
	{
		for (size_t j = 0; j < m / 2; j++)
		rowFactor[i] = rowFactor[i] + this->_table[i][2 * j + 1] * this->_table[i][2 * j];
	}
	
	for (size_t i = 0; i < t; i++)
	{
		for (size_t j = 0; j < m / 2; j++)
		columnFactor[i] = columnFactor[i] + other._table[2 * j + 1][i] * other._table[2 * j][i];
	}
	
	for (size_t i = 0; i < n; i++)
	for (size_t j = 0; j < t; j++)
	{
		mat_res._table[i][j] = -rowFactor[i] - columnFactor[j];
		for (size_t k = 0; k < m / 2; k++)
		{
			mat_res._table[i][j] = mat_res._table[i][j] + (this->_table[i][2 * k + 1] + other._table[2 * k][j])
			* (this->_table[i][2 * k] + other._table[2 * k + 1][j]);
			
		}
	}
	
	if (!isEvenColumns)
	for (size_t i = 0; i < n; i++)
	for (size_t j = 0; j < t; j++)
	mat_res._table[i][j] = mat_res._table[i][j] + this->_table[i][m - 1] * other._table[m - 1][j];
	
	return mat_res;
		}
	\end{lstlisting}
	
	\begin{lstlisting}[label=lst:vin_opt,caption=Реализация алгоритма расчета произведения матриц Винограда с оптимизациями]
	
	Matrix Matrix::dot_vin_opt(const Matrix& other)
	{
		if (this->_m != other._n)
		return Matrix(0, 0);
		
		size_t n = this->_table.size();
		size_t m = other._table.size();
		size_t t = other._table[0].size();
		size_t half_m = m / 2;
		
		Matrix mat_res = Matrix(n, t);
		std::vector<int> rowFactor(n);
		std::vector<int> columnFactor(t);
		bool isEvenColumns = (m % 2 == 0);
		for (size_t i = 0; i < n; i++)
		{
			for (size_t j = 0; j < half_m; j++)
			rowFactor[i] += this->_table[i][(j << 1) + 1] * this->_table[i][j << 1];
		}
		
		for (size_t i = 0; i < t; i++)
		{
			for (size_t j = 0; j < half_m; j++)
			columnFactor[i] += other._table[(j << 1) + 1][i] * other._table[j << 1][i];
		}
		
		for (size_t i = 0; i < n; i++)
		for (size_t j = 0; j < t; j++)
		{
			mat_res._table[i][j] = -rowFactor[i] - columnFactor[j];
			for (size_t k = 0; k < half_m; k++)
			{
				mat_res._table[i][j] += (this->_table[i][(k << 1) + 1] + other._table[k << 1][j])
				* (this->_table[i][k << 1] + other._table[(k << 1) + 1][j]);
				
			}
		}
		
		if (!isEvenColumns)
		for (size_t i = 0; i < n; i++)
		for (size_t j = 0; j < t; j++)
		mat_res._table[i][j] += this->_table[i][m - 1] * other._table[m - 1][j];
		
		return mat_res;
	}
	\end{lstlisting}
	
	\begin{lstlisting}[label=lst:stras,caption=Реализация алгоритма расчета произведения матриц Штрассена]
	std::vector<std::vector<int>> strassenMultiply(const std::vector<std::vector<int>>& matrix1,
	const std::vector<std::vector<int>>& matrix2)
	{
		
		size_t n = matrix1.size();
		
		if (n == 1)
		{
			std::vector<std::vector<int>> result(1, std::vector<int>(1, 0));
			result[0][0] = matrix1[0][0] * matrix2[0][0];
			return result;
		}
		
		
		size_t half = n / 2;
		std::vector<std::vector<int>> a11(half, std::vector<int>(half, 0));
		std::vector<std::vector<int>> a12(half, std::vector<int>(half, 0));
		std::vector<std::vector<int>> a21(half, std::vector<int>(half, 0));
		std::vector<std::vector<int>> a22(half, std::vector<int>(half, 0));
		
		std::vector<std::vector<int>> b11(half, std::vector<int>(half, 0));
		std::vector<std::vector<int>> b12(half, std::vector<int>(half, 0));
		std::vector<std::vector<int>> b21(half, std::vector<int>(half, 0));
		std::vector<std::vector<int>> b22(half, std::vector<int>(half, 0));
		
		for (size_t i = 0; i < half; i++)
		{
			for (size_t j = 0; j < half; j++)
			{
				a11[i][j] = matrix1[i][j];
				a12[i][j] = matrix1[i][j + half];
				a21[i][j] = matrix1[i + half][j];
				a22[i][j] = matrix1[i + half][j + half];
				
				b11[i][j] = matrix2[i][j];
				b12[i][j] = matrix2[i][j + half];
				b21[i][j] = matrix2[i + half][j];
				b22[i][j] = matrix2[i + half][j + half];
			}
		}
		
		
		std::vector<std::vector<int>> m1 = strassenMultiply(a11, b12 - b22);
		std::vector<std::vector<int>> m2 = strassenMultiply(a11 + a12, b22);
		std::vector<std::vector<int>> m3 = strassenMultiply(a21 + a22, b11);
		std::vector<std::vector<int>> m4 = strassenMultiply(a22, b21 - b11);
		std::vector<std::vector<int>> m5 = strassenMultiply(a11 + a22, b11 + b22);
		std::vector<std::vector<int>> m6 = strassenMultiply(a12 - a22, b21 + b22);
		std::vector<std::vector<int>> m7 = strassenMultiply(a11 - a21, b11 + b12);
		
		
		std::vector<std::vector<int>> c11 = m5 + m4 - m2 + m6;
		std::vector<std::vector<int>> c12 = m1 + m2;
		std::vector<std::vector<int>> c21 = m3 + m4;
		std::vector<std::vector<int>> c22 = m5 + m1 - m3 - m7;
		
		
		std::vector<std::vector<int>> result(n, std::vector<int>(n, 0));
		for (size_t i = 0; i < half; i++)
		{
			for (size_t j = 0; j < half; j++)
			{
				result[i][j] = c11[i][j];
				result[i][j + half] = c12[i][j];
				result[i + half][j] = c21[i][j];
				result[i + half][j + half] = c22[i][j];
			}
		}
		
		return result;
	}
	\end{lstlisting}
	
	
	
	\begin{lstlisting}[label=lst:stras_meth,caption= Метод расчета произведения алгоритмом штрассена с расширением исходной матрицы]
	Matrix Matrix::dot_shtrassen(const Matrix& other)
	{
		if (this->_m != other._n)
		return Matrix(0, 0);
		size_t max_dim = std::max(std::max(this->_n, this->_m), other._m);
		size_t max_dim2d = pow(2, ceil(log2(max_dim)));
		Matrix matrix1_scaled = this->shtrassen_extend(max_dim2d);
		Matrix matrix2_scaled = other.shtrassen_extend(max_dim2d);
		std::vector<std::vector<int>> vals = strassenMultiply(matrix1_scaled._table, matrix2_scaled._table);
		Matrix mat(this->_n, other._m);
		mat._table = vals;
		return mat;
	}
	\end{lstlisting}
	
	\begin{lstlisting}[label=lst:stras_meth,caption= Метод расширения матрицы до квадартной матрицы с заданной размерностью]
	Matrix Matrix::shtrassen_extend(std::size_t new_dim) const
	{
		
		Matrix res_matrix = Matrix(new_dim, new_dim);
		for (size_t i = 0; i < this->_n; i++)
		for (size_t j = 0; j < this->_m; j++)
		res_matrix._table[i][j] = this->_table[i][j];
		
		return res_matrix;
	}
	\end{lstlisting}
	
	
\end{appendices}